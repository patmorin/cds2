\documentclass{beamer}
\usepackage[T1]{fontenc}
\usepackage[utf8]{inputenc}
\usepackage{graphicx}
\usepackage[normalem]{ulem}
\usepackage{xmpmulti}
% \usepackage[dvipsnames]{xcolor}

\newcommand{\emphh}[1]{\textcolor{blue}{\emph{#1}}}
\newcommand{\hilite}[1]{\emphh{#1}}
\title{Connected Dominating Sets in Triangulations}

\author{}
\author{%
  Prosenjit~Bose \and
  Vida~Dujmović \and
  Hussein~Houdrouge \and
  Pat~Morin \and
  Saeed~Odak}
\date{}

\setbeameroption{hide notes} % Only slides
%\setbeameroption{show only notes} % Only notes
% \setbeameroption{show notes on second screen=right} % Both


% \DeclareMathOperator{\tw}{tw}
% \DeclareMathOperator{\td}{td}
% \DeclareMathOperator{\wcol}{wcol}
% \DeclareMathOperator{\lvr}{\chi_{\ell-\mathrm{vr}}}
% \DeclareMathOperator{\pcn}{\chi_{p}}

\begin{document}

\begin{frame}
  % \begin{center}
    \maketitle
  % \end{center}
\end{frame}



\begin{frame}
  \frametitle{SEFENOMAP}
  \uncover<3->{\textbf{Theorem (Angelini, Evans, Frati, and Gudmundsson 2016):}  If $|G_1|=n$ and $|G_2|\le n/2$ then SEFENOMAP has a solution.}

  \begin{center}
    \begin{tabular}{cc}
    \includegraphics[page=1,scale=.8]{figs/sefenomap} & \includegraphics[page=2,scale=.8]{figs/sefenomap} \\
    $G_1$ & $G_2$
  \end{tabular}
    \only<2->{\includegraphics[page=3,scale=.8]{figs/sefenomap}}
  \end{center}
  \only<4>{\textbf{Solution:} Find induced outerplane graph $G_1[S]$ with $|S|=|V(G_2)|$}
\end{frame}

\begin{frame}
  \frametitle{Free Sets}

  $S\subseteq V(G)$ is a \emphh{free set} if, for every $|S|$-point set $P$, $G$ has a \emphh{straight-line} non-crossing drawing with all vertices in $S$ mapped to points in $P$.\\[1em]

  \uncover<2->{\textbf{Theorem (Bose-Dujmović-Da~Lozza-Hurtado-Langerman-Mchedlidze--M-Roselli-Rote-Wood 2009--2021):} Every $n$-vertex planar graph $G$ has a free set of size $\Omega(\sqrt{n})$.\\[1em]

  \textbf{Theorem (Ravsky-Verbistky 2011):}  There exists $n$-vertex planar $G$ with no free set of size $\Omega(n^{0.98583})$.}

  % \uncover<2->{$S$ is a free-set iff and only if $S$ is contained in a \emphh{proper good curve}}
  \begin{center}
    \uncover<3->{\includegraphics[page=2,scale=0.8]{figs/1-proper}}
  \end{center}
  % \uncover<2->{$S\subseteq V(G)$ is a \emphh{one-bend free set} if, for every $|S|$-point set $P$, $G$ has a \emphh{one-bend} non-crossing drawing with all vertices in $S$ mapped to points in $P$.}
\end{frame}

\begin{frame}
  \frametitle{One-Bend Free Sets}

  $S\subseteq V(G)$ is a \emphh{one-bend free set} if, for every $|S|$-point set $P$, $G$ has a \emphh{one-bend} non-crossing drawing with all vertices in $S$ mapped to points in $P$.\\[1em]
  % $S\subseteq V(G)$ is a \emphh{free set} if, for every $|S|$-point set $P$, $G$ has a \emphh{straight-line} non-crossing drawing with all vertices in $S$ mapped to points in $P$.\\[1em]

  % \textbf{Theorem (Here):} Every $n$-vertex planar graph $G$ has a one-bend free set of size at least $11n/21$.\\[2em]

  \textbf{Theorem (Goldner-Harary):}  There exists $n$-vertex planar $G$ with no free set of size greater than $10n/11$.

  \uncover<2->{\textbf{Observation:} $S\subseteq V(G)$ is a one-bend free set iff $S$ is a free set in a $1$-subdvision of $G$}

  \only<1-2>{\includegraphics[page=9]{figs/proper_good}}%
  \only<3>{\includegraphics[page=7]{figs/proper_good}}%
  \only<4->{\includegraphics[page=8]{figs/proper_good}}%
\end{frame}


\begin{frame}
  \frametitle{(Connected) Dominating Sets}

  \begin{itemize}
    \item $X\subseteq V(G)$ is a \emphh{dominating set} of $G$ if $N[X]=V(G)$
    \begin{center}
      \only<1>{\includegraphics[page=10]{figs/walkthrough}}%
      \only<2>{\includegraphics[page=11]{figs/walkthrough}}%
    \end{center}
    \item $X$ is \emphh{connected} if $G[X]$ is connected
  \end{itemize}
\end{frame}


\begin{frame}
  \frametitle{Connected Dominating Sets and Lush Spanning Trees}

  \begin{center}
    $X$ is a connected dominating set of $G$ \\[1em]
    $\Updownarrow$ \\[1em]
    $G$ has a spanning tree where vertices in $V(G)\setminus X$ are leaves.
    \only<1>{\includegraphics[page=11]{figs/walkthrough}}%
    \only<2>{\includegraphics[page=12]{figs/walkthrough}}%
    \only<3>{\includegraphics[page=9]{figs/walkthrough}}%
  \end{center}
\end{frame}

\begin{frame}
  \frametitle{Triangulations}

  \textbf{Theorem (Albertson, Berman, Hutchinson, and Thomassen 1990):}  Every triangulation has a \hilite{homemorphically irreducible} spanning tree.\\[2em]

  \uncover<2->{\textbf{Corollary:} Every $n$-vertex triangulation has a spanning tree with at least $n/2$ leaves.\\[2em]}

  \uncover<3->{\textbf{Corollary:} Every $n$-vertex triangulation has a connected dominating set of size at least $n/2$.}
\end{frame}

\begin{frame}
  \frametitle{Induced Outerplane Graphs}

  \begin{center}
    \only<1>{\includegraphics[page=13]{figs/walkthrough}}%
    \only<2->{\includegraphics[page=14]{figs/walkthrough}}%
  \end{center}

  \uncover<2->{\textbf{Observation:} If $X$ is a connected dominating set then $G-X$ is an induced outerplace graph with $n-|X|$ vertices.}\\[2em]

  \uncover<3->{Better solution to SEFENOMAP if $|X|< n/2$}
\end{frame}

\begin{frame}
  \frametitle{$2$-Proper Good Curves}

  \begin{center}
    \only<1>{\includegraphics[page=1]{figs/proper_good}}%
    \only<2>{\includegraphics[page=2]{figs/proper_good}}%
    \only<3>{\includegraphics[page=3]{figs/proper_good}}%
    \only<4>{\includegraphics[page=4]{figs/proper_good}}%
    \only<5>{\includegraphics[page=5]{figs/proper_good}}%
    \only<6>{\includegraphics[page=6]{figs/proper_good}}%
    % \only<7>{\includegraphics[page=7]{figs/proper_good}}%
    % \only<8>{\includegraphics[page=8]{figs/proper_good}}%
  \end{center}

  \uncover<7->{\textbf{Observation:} If $X$ is a connected dominating set then $V(G)\setminus X$ is a one-bend free set.}
\end{frame}


\begin{frame}
  \frametitle{The Benchmark: $n/3$}

  \begin{center}
    \includegraphics[height=.7\textheight]{figs/nover3}
  \end{center}
\end{frame}


\begin{frame}
  \frametitle{Proper Good Curves}

  Free sets correspond to vertices on a \emphh{proper good curves}:

  One-bend free sets correspond to vertices on a \emphh{$2$-proper good curve}.

\end{frame}

\begin{frame}
  \frametitle{Main Result}

  \textbf{Theorem:}  Every $n$-vertex triangulation $G$ has a connected dominating set of size at most $10n/21=\overline{0.476190}n$.\\[2em]

  \textbf{Corollary:}  Every $n$-vertex triangulation $G$ has a spanning tree with at least $11n/21=\overline{0.523809}n$ leaves.\\[2em]

  \textbf{Corollary:}  Every $n$-vertex triangulation $G$ has an induced outerplane graph $G[Y]$ with at least $11n/21=\overline{0.523809}n$ vertices.\\[2em]

  \textbf{Corollary:}  A bound of $11n/21$ for SEFENOMAP.\\[2em]

  \textbf{Corollary:}  Every $n$-vertex planar graph has a one-bend free set of size $11n/21$.
\end{frame}

\begin{frame}
  \frametitle{NaïveGreedy Strategy}

  \begin{tabular}{p{.6\textwidth}c}

    $\textsc{NaïveGreedy}(G)$:\newline
    $X\gets\{v_0\}$ \newline
    Repeat until $N[X]=V(G)$: \newline
    --- Let $v\in N(X)$ maximize $|N(v)\setminus N(X)|$ \newline
    --- Set $X\gets X\cup\{v\}$
    &
    \raisebox{-\height}{%
      \only<1>{\includegraphics[width=.4\textwidth,page=1]{figs/walkthrough}}%
      \only<2>{\includegraphics[width=.4\textwidth,page=2]{figs/walkthrough}}%
      \only<3>{\includegraphics[width=.4\textwidth,page=3]{figs/walkthrough}}%
      \only<4>{\includegraphics[width=.4\textwidth,page=4]{figs/walkthrough}}%
      \only<5>{\includegraphics[width=.4\textwidth,page=5]{figs/walkthrough}}%
      \only<6-7>{\includegraphics[width=.4\textwidth,page=6]{figs/walkthrough}}%
      \only<8>{\includegraphics[width=.4\textwidth,page=7]{figs/walkthrough}}%
      \only<9>{\includegraphics[width=.4\textwidth,page=8]{figs/walkthrough}}%
      % \only<10>{\includegraphics[width=.4\textwidth,page=9]{figs/walkthrough}}%
    }
  \end{tabular}


  \begin{tabular}{p{.6\textwidth}c}
    \raggedright
    \uncover<4->{
    \textbf{Ideally:} $|N(v)\setminus N(X)|\ge c$ at each step \newline
    Would give $|X|\le n/c$\vspace{1em}
    }

    \uncover<6->{
    \textbf{Problem:} Impossible to guarantee $|N(v)\setminus N(X)|\ge 2$.\vspace{1em}
    }

    & \uncover<7->{\raisebox{-\height}{\includegraphics[width=.4\textwidth]{figs/noguarantee}}}
  \end{tabular}
\end{frame}


\begin{frame}
  \frametitle{$1$-Critical Graphs}

  All inner faces are triangles \newline
  All vertices have inner degree $\le 1$ \\[3em]

  \includegraphics[width=.98\textwidth]{figs/critical}\\[3em]

  \textbf{Lemma:} $\text{\#inner vertices} \le \tfrac{1}{4}\text{\#vertices}$
\end{frame}

\begin{frame}
  \frametitle{A First Attempt}

  \textbf{Theorem:} $\textsc{NaïveGreedy}(G)$ produces a connected dominating set $X$ of size at most $4n/7$.\\[1em]

  \uncover<2->{
  \textit{Proof:} Takes vertices of inner-degree at least $2$ until $G-X$ is $1$-critical then takes $I:=V(G)-N[X]$ additional vertices.
  \begin{center}
    \includegraphics{figs/greedy} \includegraphics[page=6]{figs/walkthrough}
  \end{center}
  \[
     |X| = r + I , \quad 2r + I \le n, \quad I \le (n-r)/4
  \]
  Maximized when $r=3n/7$, $I=n/7$
  }
\end{frame}

\begin{frame}
  \frametitle{Some Lies I've Told}

  \begin{itemize}
    \item Actually a linear program that involves:
    \begin{itemize}
      \item $r$: number of rounds
      \item $x_d$: number of times we choose a vertex of inner-degree $d$
      \item $D$: Number of dominated vertices $\sum_{d\ge 2}d x_d$
      \item $I$: Number of undominated vertices
      \item $B$: Number of \emphh{boundary} vertices  (X---D---I)
    \end{itemize}
    \[
       B \le D - r
    \]
    \[
       D+I \le n
    \]
    \[
      I \le B/3
    \]
    \[
       |X| \le r + I
    \]
  \end{itemize}
\end{frame}

\begin{frame}
  \frametitle{Lessons}

  \begin{itemize}[<+->]
    \item Always taking inner-degree $\ge2$ leads to $|X|\le n/2$
    \item We lose a bit in the final step ($1$-critical graph)
    \item To beat $n/2$, we need to choose vertices of inner-degree $>2$\\[3em]
  \end{itemize}
  \uncover<4->{
  \begin{center}
    But...\\[1em]
    \includegraphics{figs/ring}
  \end{center}
  }
\end{frame}


\begin{frame}
  \frametitle{The Next-Best Thing}

  Take vertices of inner-degree $\ge 2.5$

  \begin{center}
    \only<+>{\includegraphics[page=1]{figs/ring}}%
    \only<+>{\includegraphics[page=2]{figs/ring}}%
    \only<+>{\includegraphics[page=3]{figs/ring}}%
    \only<+>{\includegraphics[page=4]{figs/ring}}%
    \only<+>{\includegraphics[page=5]{figs/ring}}%
    \only<+>{\includegraphics[page=6]{figs/ring}}%
    \only<+>{\includegraphics[page=7]{figs/ring}}%
    \only<+>{\includegraphics[page=8]{figs/ring}}%
  \end{center}
\end{frame}


\begin{frame}
  \frametitle{Final Step: $2$-Critical Graphs}

  \begin{center}
    \only<+>{\includegraphics[page=1]{figs/2-critical}}%
    \only<+>{\includegraphics[page=2]{figs/2-critical}}%
    \only<+->{\includegraphics[page=3]{figs/2-critical}}%
    % \only<+>{\includegraphics[page=4]{figs/2-critical}}%
    % \only<+>{\includegraphics[page=5]{figs/2-critical}}%
    % \only<+>{\includegraphics[page=6]{figs/2-critical}}%
    % \only<+>{\includegraphics[page=7]{figs/2-critical}}%
    % \only<+>{\includegraphics[page=8]{figs/2-critical}}%
  \end{center}
  \uncover<4->{
    \[
      \tfrac{2}{3}\operatorname{boundary}(G-N[X]) + \tfrac{1}{3}\operatorname{interior}(G-N[X]) =: \tfrac{2}{3}R + \tfrac{1}{3}S
    \]
  }
\end{frame}

\begin{frame}
  \frametitle{Analysis}

  \textbf{Theorem:} $\textsc{BestGreedy}(G)$ produces a connected dominating set of size at most $10n/21$.\\[1em]

  \textit{Proof:} Takes vertices of inner-degree at least $2.5$ until $G-X$ is $2$-critical then takes $\tfrac{2}{3}R+\tfrac{1}{3}S$ additional vertices

  % \begin{center}
  %   \includegraphics{figs/greedy} \includegraphics[page=6]{figs/walkthrough}
  % \end{center}
  \begin{tabular}{p{.05\textwidth}c}
  \[
     |X| \le r + \tfrac{2}{3}R + \tfrac{1}{3}S
  \]
  \[
     D \ge 2.5r
  \]
  \[
    D + R + S = n
  \]
  \[
    B \le D - r
  \]
  \[
    R \le B
  \]
  \[
    S \le \tfrac{1}{3}R
  \]
  & \raisebox{-\height}{\includegraphics{figs/dbrs}}
  \end{tabular}
  LP gives $|X|\le 10n/21$.
  \uncover<2->{\hfill{$\Box$}}
  % }
\end{frame}

\begin{frame}
  \frametitle{Conclusions}

  \begin{itemize}[<+->]
    \item Lower Bound is $n/3$ upper bound is $10n/21$.

    \item Better way to get one-bend free sets?

    \item SEFENOMAP where all edges of $G_1$ have small complexity?

    \item Connected dominating sets in edge-maximal beyond-planar classes?
    \begin{itemize}[<+->]
      \item Start with edge-maximal $1$-planar
    \end{itemize}
  \end{itemize}

  \begin{center}
    \uncover<+->{\Huge Thank You!}
  \end{center}
\end{frame}

\end{document}
