\documentclass{article}
\listfiles
\usepackage[OT1]{fontenc}
\usepackage[utf8]{inputenc}
\usepackage{kpfonts}
\usepackage{graphicx}
\usepackage[dvipsnames]{xcolor}
\usepackage{amsthm}
\usepackage[margin=1.25in]{geometry}
\usepackage[sf,bf,small,raggedright,compact]{titlesec}
\usepackage{hyperref}
\definecolor{linkblue}{named}{MidnightBlue}
\hypersetup{colorlinks=true, linkcolor=linkblue,  anchorcolor=linkblue,
        citecolor=linkblue, filecolor=linkblue, menucolor=linkblue,
        urlcolor=linkblue}
\usepackage[capitalize]{cleveref}
\usepackage{paralist}
\usepackage[longnamesfirst,numbers,sort&compress]{natbib}
% \usepackage[normalem]{ulem}

\usepackage{stmaryrd}

\newenvironment{patnote}{\par  \color{red}PM:}{\par}

\newcommand{\pat}[1]{\textcolor{red}{PM: #1}}
\newcommand{\hussein}[1]{\textcolor{purple}{HH: #1}}

\usepackage{listings,newtxtt}
\lstset{basicstyle=\ttfamily, keywordstyle=\bfseries}


\newcommand{\R}{\mathbb{R}}
\newcommand{\pref}[1]{(P\ref{#1})}
\setlength{\parskip}{1ex}

\newtheorem{thm}{Theorem}
\newtheorem{obs}{Observation}
\newtheorem{lem}{Lemma}
\crefname{lem}{Lemma}{Lemmata}
\newtheorem{cor}{Corollary}
\newtheorem{clm}{Claim}
\newenvironment{clmproof}{\noindent\emph{Proof of Claim:}}{\hfill{$\blacksquare$}\par}

\crefname{dm}{}{}
\creflabelformat{dm}{#2(DM#1)#3}
\crefname{dp}{}{}
\creflabelformat{dp}{#2(DP#1)#3}
\crefname{p}{}{}
\creflabelformat{p}{#2(#1)#3}
\crefname{bg}{}{}
\creflabelformat{bg}{#2[bg#1]#3}


\DeclareMathOperator{\corners}{\Gamma}
\DeclareMathOperator{\dist}{d}
\DeclareMathOperator{\pack}{pack}
\DeclareMathOperator{\hit}{hit}

\newcommand{\defin}[1]{\emph{\textcolor{Maroon}{#1}}}


\raggedbottom
\theoremstyle{definition}
\newtheorem{definition}{Definition}[section]
\newtheorem{observation}{Observation}

% \newcommand{\pat}[1]{[\textcolor{red}{PM: #1}]}
% \newcommand{\vida}[1]{[{\color{pink}VD: #1}]}
% \newcommand{\saeed}[1]{[{\color{blue}SO: #1}]}
% \newcommand{\hussein}[1]{[\textcolor{purple}{$H^2$: #1}]}

\title{Connected Dominating Sets in Triangulations}
\author{%
  Prosenjit Bose\thanks{School of Computer Science, Carleton University. Research partially funded by NSERC} \and
  Vida Dujmović\thanks{Department of Computer Science and Electrical Engineering, University of Ottawa. Research partially funded by NSERC}\and
  Hussein Houdrouge\footnotemark[1] \and
  Pat Morin\footnotemark[1] \and
  Saeed Odak\footnotemark[2]}
\date{}



\begin{document}

\maketitle

\begin{abstract}
  We show that every $n$-vertex triangulation has a connected dominating set of size at most $10n/21$.  Equivalently, every $n$ vertex triangulation has a spanning tree with at least $11n/21$ leaves. Prior to the current work, the best known bounds were $n/2$, which follows from work of Albertson, Berman, Hutchinson, and Thomassen (J. Graph Theory \textbf{14}(2):247--258). One immediate consequence of this result is an improved bound for the SEFENOMAP graph drawing problem of Angelini, Evans, Frati, and Gudmundsson (J. Graph Theory \textbf{82}(1):45--64).  As a second application, we show that every $n$-vertex planar graph has a one-bend non-crossing drawing in which some set of at least $11n/21$ vertices is drawn on the $x$-axis.  The main result extends to $n$-vertex triangulations of genus-$g$ surfaces, and implies that these have connected dominating sets of size at most $10n/21+O(\sqrt{gn})$.
\end{abstract}


\section{Introduction}

A set $X$ of vertices in a graph $G$ is a \defin{dominating set} of $G$ if each vertex of $G$ is in $X$ or adjacent to a vertex in $X$.  A dominating set $X$ of $G$ is \defin{connected} if the subgraph $G[X]$ of $G$ induced by the vertices in $X$ is connected.  There is an enormous body of literature on dominating sets. Several books are devoted to the topic \cite{haynes.hedetniemi.ea:domination,haynes.hedetniemi.ea:topics,du.wan:connected,haynes.hedetniemi.ea:vol2}, including a book and book chapter devoted to connected dominating sets \cite{du.wan:connected,chellali.favaron:connected}.  A typical result in the area is of the form: ``Every $n$-vertex graph in some family $\mathcal{G}$ of graphs has a (connected) dominating set of size at most $f(n)$.'' or ``For infinitely many $n$, there exists an $n$-vertex member of $\mathcal{G}$ with no (connected) dominating set of size less than $g(n)$.''

\subsection{Connected Dominating Sets in Triangulations}

A \defin{triangulation} is an edge-maximal planar graph.  \citet{matheson.tarjan:dominating} proved that every $n$-vertex triangulation has a dominating set of size at most $n/3=0.33\overline{3}n$ and that there exists $n$-vertex triangulations with no dominating set of size less than $n/4=0.25n$. The gap between these upper and lower bounds stood for over $20$ years until a recent breakthrough by \citet{spacapan:domination} reduced the upper bound to $17n/53\approx 0.32075471698n$.  This was swiftly followed by an improvement to $2n/7= 0.2\overline{857142}n$ by \citet{christiansen.rotenberg.ea:triangulations}.

In the current paper, we consider connected dominating sets in triangulations.  An easy consequence of the proof used by \citet{matheson.tarjan:dominating} is that $n$-vertex triangulations have connected dominating sets of size at most $2n/3=0.66\overline{6}n$.  A more general result, due to  \citet{kleitman.west:spanning} shows that graphs of minimum-degree $3$ in which each edge is included in a $3$-cycle have connected dominating sets of size at most $2(n-5)/3<0.66\overline{6}n$. \citet{albertson.berman.ea:graphs} prove that every triangulation has a spanning tree $T$ with no vertices of degree $2$, which implies that the number of leaves of $T$ exceeds the number of non-leaves by at least $2$.  In particular, the number of leaves is greater than $(n+2)/2$. Thus, the set $X$ of non-leaf vertices of $T$ is a connected dominating set of size at most $(n-2)/2<0.5n$. We prove:
% \citet{kleitman.west:spanning} also show that graphs of minimum degree $4$ have connected dominating sets of size at most $(3n-8)/5<0.6n$. This result applies to triangulations of minimum degree $4$, which includes all $4$-connected triangulations.  The main result in this paper is the following:

\begin{thm}\label{main_result2}
  For every $n\ge 3$, every $n$-vertex triangulation $G$ has a connected dominating set $X$ of size at most $10n/21= 0.\overline{476190}n$.
\end{thm}

The proof of \cref{main_result2} is constructive and gives an $O(n)$ time algorithm for finding the set $X$.  The best known lower bound for this problem is $n/3 = 0.33\overline{3} n$, obtained from a triangulation that contains $n/3$ vertex-disjoint pairwise-nested triangles $\Delta_1,\ldots,\Delta_{n/3}$.  In order to dominate $\Delta_1\cup \Delta_2$, any dominating set must contain at least two vertices in $\Delta_1 \cup \Delta_2$. In order to dominate $\Delta_{n/3-1}\cup \Delta_{n/3}$, any dominating set must contain two vertices in $\Delta_{n/3-1}\cup\Delta_{n/3}$.  Then, in order to be connected, any connected dominating set must contain a vertex in each of $\Delta_3,\ldots,\Delta_{n/3-2}$.

Connected dominating sets are closely related to spanning trees with many leaves.  If $X$ is a connected dominating set of a graph $G$, then $G$ has a spanning tree in which the vertices of $G-X$ are all leaves.  To see this, start with a spanning tree $T$ of $G[X]$ and then, for each  $v\in V(G)\setminus X$ choose some $x\in N_G(v)\cap X$ and add the edge $xv$ to $T$. Conversely, if $T$ is a spanning tree of $G$ with leaf set $L$, then $X:=V(T-L)$ is a connected dominating set of $G$.  Thus, \cref{main_result2} has the following equivalent statement:

\begin{cor}\label{main_theorem_cor}
  For every $n\ge 3$, every $n$-vertex triangulation $G$ has a spanning tree $T$ with at least $11n/21= 0.\overline{523809}n$ leaves.
\end{cor}

\Cref{main_theorem_cor} makes progress on the maxleaf spanning-tree problem for triangulations, explicitly posed by \citet[Question~4.2]{bradshaw.masarik.ea:robust}.  \Cref{main_theorem_cor} also answers a problem posed by \citet{noguchi.zamfirescu:spanning}, who asked if there exists some $\epsilon >0$ such that, for every sufficiently large $n$, every $n$-vertex $4$-connected triangulation has a spanning tree with at least $(1/2+\epsilon)n$ leaves.  \Cref{main_theorem_cor} gives an affirmative answer to this question (with $\epsilon=1/42$), even without the $4$-connectivity condition.

A \defin{surface triangulation} is a graph $G$ embedded on a surface $\mathcal{S}$ in such a way that each face of the embedding is a topological disc whose boundary is a $3$-cycle in $G$.  The Euler genus of a surface triangulation is the Euler genus of the surface $\mathcal{S}$ on which $G$ is embedded.  Using existing techniques for slicing surface-embedded graphs, we obtain the following generalization of \cref{main_result2,main_theorem_cor}:

\begin{thm}\label{genus_result}
  For every $n\ge 3$, every $n$-vertex Euler genus-$g$ surface triangulation $G$ has a connected dominating set $X$ of size at most $10n/21 +O(\sqrt{gn})= 0.\overline{476190}n + O(\sqrt{gn})$.  Equivalently, $G$ has a spanning tree with at least $11n/21 -O(\sqrt{gn})=0.\overline{523809}n-O(\sqrt{gn})$ leaves.
\end{thm}



\subsection{One-Bend Collinear Sets in Planar Graphs}

\pat{Rewrite this section in terms of one-bend free sets.}

The original motivation for this research was a graph drawing problem, which we now describe.  It is known that every $n$-vertex planar graph $G$ has a non-crossing drawing in the plane with edges of $G$ drawn as line segments and such that some set $Y$ of $\Omega(n^{1/2})$ vertices of $G$ are drawn on the $x$-axis \cite{bose.dujmovic:polynomial,dujmovic:utility}.  The set $Y$ is called a \defin{collinear set}.  For bounded-degree planar graphs, this result can be improved to $|Y|=\Omega(n^{0.8})$ \cite{dujmovic.morin:dual}.  Determining the supremum value of $\alpha$ such that every $n$-vertex planar graph has a collinear set of size $\Omega(n^{\alpha})$ remains a difficult open problem, but it is known that $\alpha \le \log_{23}(22)<0.9859$ \cite{ravsky.verbitsky:collinear}.

We consider a relaxation of this problem in which the edges of the graph can be drawn as a polygonal path consisting of at most two line segments.  Such a drawing is called a \defin{one-bend} drawing of $G$.  A subset $Y$ of $V(G)$ is a \defin{one-bend collinear set} if $G$ has a one-bend drawing in which all vertices of $Y$ appear on the $x$-axis.  We show that, for any spanning tree $T$ of $G$, the leaves of $T$ are a one-bend collinear set of $G$.  Combined with \cref{main_theorem_cor}, this gives:

\begin{thm}\label{one_bend_collinear_thm}
  For every $n\ge 3$, every $n$-vertex planar graph has a one-bend collinear set of size at least $11n/21=0.\overline{523809}n$.
\end{thm}

Note that if all the vertices of $G$ are on the $x$-axis then no edge in a one-bend drawing of $G$ crosses the $x$-axis, so this one bend drawing is a $2$-page book-embedding of $G$. This implies that $G$ is a spanning subgraph of some Hamiltonian triangulation $G^+$.  The Goldner–Harary graph is an $11$-vertex triangulation that is not Hamiltonian. It follows that the graph $G$ obtained by taking $k$ vertex-disjoint copies of the Goldner-Harary graph has $n:=11k$ vertices and has no one-bend collinear set of size greater than $10k=10n/11=0.90\overline{90}n$.

\subsection{Related Work}

\paragraph{Connected Dominating Sets}

The book chapter by \citet{chellali.favaron:connected} surveys combinatorial results on connected-dominating sets, including the result of \citet{kleitman.west:spanning} mentioned above.  Most results of this form focus on graphs with lower bounds on their minimum degree, possibly combined with some other constraints.  For example, the Kleitman-West result relevant to triangulations is about graphs of minimum-degree $3$ in which each edge participates in a $3$-cycle.

\citet{wan.alzoubi.ea:simple} describe an algorithm for finding a connected dominating set in a $K_t$-minor-free graph. When run on an $n$-vertex planar graph, their algorithm produces a connected dominating set of size at most $15\alpha(G^2)-5$, where $\alpha(G^2)$ is the size of the largest \defin{distance-$2$ independent set} in $G$; i.e., the largest subset of $V(G)$ that contains no pair of vertices whose distance in $G$ is less than or equal to $2$.  However, there exists $n$-vertex triangulations $G$ with $\alpha(G^2)=n/4$,\footnote{To create such a triangulation, start with $n/4$ vertex-disjoint copies of $K_4$ embedded so that each copy contributes three vertices to the outer face and has one inner vertex, and then add edges arbitrarily to create a triangulation.  Then the inner vertices of the copies of the original copies of $K_4$ form an independent set in $G^2$.} for which this algorithm does not guarantee an output of size less than $n$.

As discussed above, the result of \citet{albertson.berman.ea:graphs} on spanning-trees without degree-$2$ vertices implies that every $n$-vertex triangulation has a connected dominating set of size $(n-2)/2$.  \citet{chen.ren.ea:homeomorphically} give a significant generalization of this result, which applies to any connected graph $G$ in which the graph induced by the neighbours of each vertex is connected.
 Motivated by  a graph drawing problem (SEFE without mapping) \citet{angelini.evans.ea:sefe} show that every $n$-vertex plane graph $G$ (and therefore every triangulation) contains an induced outerplane graph $G[L]$ with least $n/2$ vertices.  If $X$ is a connected dominating set in a triangulation $G$, then $G-X$ is an outerplane graph.\footnote{In \cref{discussion}, we show that the converse of this statement is almost true: the largest $Y\subset V(G)$ such that $G[Y]$ is outerplane and the smallest connected dominating set $X\subset V(G)$ satisfy $|X|+|Y|=|V(G)|$.}  Our \cref{main_result2} therefore implies an improved result for their graph drawing problem, improving the bound from $n/2$ to $11n/21$, as described in the following theorem:

% \pat{Unlike what we wrote in a previous version, the problem of finding the largest induced outerplane graph $G[Y]$ is not immediately equivalent to finding the smallest connected dominating set.  Any cycle $C$ in $G[Y]$ could be a separating cycle in $G$. One has to show that we can always replace such a cycle $C$ with a cycle $C'$ of at least the same length, but such that $C'$ has fewer vertices of $G$ in its interior.
% For that you need to show that at least one of the following is true: (i) $C$ contains a vertex $v$ in its interior that is adjacent to at least three consecutive vertices $xyz$ of $C$, in which case we replace $y$ with $v$. (ii) the number of vertices in the interior of $C$ that are adjacent to $C$ is at least $|V(C)|$, in which case we replace $C$ with this set of vertices.  Is this something we should include?
% See the last section for the start of a proof....
% }

\begin{thm}\label{sefenomap}
  For every $n$-vertex planar graph $G_1$ and every $\lceil 11n/21\rceil$-vertex planar graph $G_2$, there exists point sets $P_1\supseteq P_2$ with $|P_1|=n$, $|P_2|=\lceil 11n/21\rceil$ and non-crossing embeddings of $G_1$ and $G_2$ such that
  \begin{compactenum}[(a)]
    \item the vertices of $G_i$ are mapped to the points in $P_i$ for each $i\in\{1,2\}$;
    \item the edges of $G_2$ are drawn as line segments; and
    \item each edge $e$ of $G_1$ whose endpoints are both mapped to points in $P_2$ is drawn as a line segment.
  \end{compactenum}
\end{thm}


In the introduction we describe an $n$-vertex triangulation (containing a sequence of nested triangles) for which every connected dominating set has size at least $n/3$.  Since this example contains many separating triangles it is natural to consider the special case of $4$-connected triangulations.  \citet{noguchi.zamfirescu:spanning} describe, for infinitely many values of $n$, $4$-connected $n$-vertex triangulations for which any connected dominating set has at least $n/3$ vertices.

In the current paper, we consider \defin{triangulations}; edge-maximal planar graphs. \citet{hernandez:vigilancia} and \citet{chen.hao.ea:bounds} show that every edge-maximal \emph{outerplanar} graph has a connected dominating set of size at most $\lfloor (n-2)/2\rfloor$. This immediately implies that every Hamiltonian $n$-vertex triangulation (including every $4$-connected $n$-vertex triangulation) has a connected dominating set of size at most $\lfloor (n-2)/2\rfloor$.  For even $n\ge 4$, the bound $(n-2)/2$ for edge-maximal outerplanar graphs is tight, as is easily seen by constructing a graph with two degree-$2$ vertices $s$ and $t$ and $(n-2)/2$ disjoint edges, each of which separates $s$ from $t$.  \citet{zhuang:connected} parameterizes the problem by the number, $k$,  of degree-$2$ vertices, and shows that any $n$-vertex edge-maximal outerplanar graph with $k$ degree-$2$ vertices has a connected dominating set of size at most $\lfloor\min\{(n+k-4)/2,2(n-k)/3\}\rfloor$.

%
%
%
%
% \pat{Add: \newline
% results on edge-maximal outerplanar graphs --- implies $n/2$ bound for Hamiltonian planar graphs (including $4$-connected.) \newline
% connected dominating sets in unit disc graphs \newline
% $2$-bend collinear sets \newline
% $15$-approximation for minimum connected dominating set in planar graphs \newline
% }

\paragraph{Collinear Sets}

It is not difficult to establish that a planar graph $G$ has a one-bend drawing in which all vertices of $G$ are drawn on the $x$-axis if and only if $G$ has a $2$-page book embedding. It is well-known that $2$-page graphs are \defin{subhamiltonian}; each such graph is a spanning subgraph of some Hamiltonian planar graph. As pointed out already, the non-Hamiltonian Goldner-Harary triangulation can be used to construct an $n$-vertex planar graph with no one-bend collinear set larger than $10n/11$.

The $x$-axis is just one example of an $x$-monotone function $f(x)=x$. Unsurprisingly, perhaps, this function turns out to be the most difficult for one-bend graph drawing.
% \citet{DBLP:conf/gd/EverettLLW07} describe a \defin{one-bend universal} $n$-point set $S_n$ such that every $n$-vertex planar graph $G$ has a one-bend drawing with the vertices of $G$ drawn on the points in $S_n$. By construction, the point set $S_n$ is contained in the graph of a strictly concave decreasing function.
\citet{DBLP:journals/comgeo/GiacomoDLW05} show that for \emph{any} strictly concave function $f:[0,1]\to\R$ and any planar graph $G$ there exists a function $x:V(G)\to[0,1]$ such that $G$
% , and any total order $<$ on $V(G)$ there exists a function $x:V(G)\to\R$ such that $x(v) < x(w)$ if and only if $x< w$ and $G$
has a one-bend drawing in which each vertex $v$ of $G$ is drawn at the point $(x(v),f(x(v)))$.\footnote{Their result is actually considerably stronger: For any total order $<_G$ on $V(G)$ they provide a function $x$ such that $x(v) < x(w)$ if and only if $x<_G w$.}
  In our language, every $n$-vertex planar graph $G$ has a \defin{co-$f$-ular} set of size $n$.
% These results are similar in flavour to \cref{one_bend} in that they produce drawings in which a subset of $V(G)$ is drawn on a particular $1$-dimensional curve.

% For every $n \in \mathbb{N}$, Everett et al. \cite{DBLP:conf/gd/EverettLLW07} show that there exists a universal set $\mathcal{U}_n$ of $n$ distinct points in the plane such that every $n$-vertex planar graph admits a 1-bend drawing with vertices on the points of set $\mathcal{U}_n$. Furthermore, Giacomo et al. \cite{DBLP:journals/comgeo/GiacomoDLW05} demonstrate that for a linear ordering $L$ of vertices of a planar triangulation $G$ and strictly convex curve $\lambda$, there is a one-bend plane drawing of $G$ such that vertices of $G$ appear on $\lambda$ with the same order as in $L$.

Another interpretation of the result of \citet{DBLP:journals/comgeo/GiacomoDLW05} is that every strictly concave curve $C_f:=\{(x,f(x)):0\le x\le 1\}$ is \defin{one-bend universal}; for every planar graph $G$ there exists a one-bend drawing of $G$ in which the vertices of $G$ are mapped to points in $C_f$. \citet{DBLP:conf/gd/EverettLLW07} take this a step further and describe a \defin{one-bend universal} $n$-point set $S_n$ such that every $n$-vertex planar graph $G$ has a one-bend drawing with the vertices of $G$ drawn on the points in $S_n$. (In their construction, the points in $S_n$ happen to lie on a strictly concave curve.)
% By construction, the point set $S_n$ is contained in the graph of a strictly concave decreasing function.
With further relaxation on the drawing of the edges, de Fraysseix et al. \cite{DBLP:journals/combinatorica/FraysseixPP90} show that \emph{any} set of $n$ points in the plane is \defin{two-bend universal}; for \emph{any} set of $n$ points in $\R^2$ and any $n$ vertex planar graph $G$ there exist a two-bend drawing of $G$ in which the vertices of $G$ are drawn on the points in $S$.
% .a universal set for the 2-bend drawing of planar graphs.

% Furthermore, They show that the planar embedding problem of any $n$-vertex graph on an $n$ collinear points in $\mathbb{R}^2$ with at most one bend along each edge is NP-complete.

\subsection{Outline}

The remainder of this paper is organized as follows:  In \cref{strategy}, we describe the general strategy we use for finding connected dominating sets in triangulations.  In \cref{warm_up} we show that a simple version of this strategy can be used to obtain a connected dominating set of size at most $4n/7= 0.\overline{571428}n$.
% , which is already better than the best known bounds for triangulations.
In \cref{full_result} we show that a more careful construction leads to a proof of \cref{main_result2}.  In \cref{bounded_genus} we prove \cref{genus_result}.  In \cref{one_bend}, we discuss the connection between connected dominating sets and one-bend collinear drawings that leads to \cref{one_bend_collinear_thm}.



% \subsection{Related Work}
%
% \pat{Explain how Kleitman-West result gives $3n/4$.  Explain how Matheson-Tarjan gives $2n/3$. Mention results on connected dominating sets in maximal outerplanar graphs.  Talk about connected dominating sets in unit disk graphs.}


\section{The General Strategy}
\label{strategy}

Throughout this paper, we use standard graph-theoretic terminology as used, for example, by \citet{diestel:graph}.
For a graph $G$, let $|G|=|V(G)|$ denote the number of vertices of $G$.  A \defin{bridge} in a graph $G$ is an edge $e$ of $G$ such that $G-e$ has more connected components than $G$.  For a vertex $v\in G$, $N_G(v):=\{w\in V(G):vw\in E(G)\}$ is the \defin{open neighbourhood} of $v$ in $G$,  $N_G[v]:=N_G(v)\cup\{v\}$ is the \defin{closed neighbourhood} of $v$ in $G$.  For a vertex subset $S\subseteq V(G)$, $N_G[S]:=\bigcup_{v\in S} N_{G}[v]$ is the \defin{closed neighbourhood} of $S$ in $G$ and $N_G(S):=N_G[S]\setminus S$ is the \defin{open neighbourhood} of $S$ in $G$.  A set $X\subseteq V(G)$ \defin{dominates} a set $B\subseteq V(G)$ if $B\subseteq N_G[X]$.  Thus, $X$ is a dominating set of $G$ if and only if $X$ dominates $V(G)$.

A \defin{plane graph} is a graph equipped with a non-crossing embedding in $\mathbb{R}^2$.  A plane graph is \defin{outerplane} if all its vertices appear on the outer face.  A \defin{triangle} is a cycle of length $3$. A \defin{near-triangulation} is a plane graph whose outer face is bounded by a cycle and whose inner faces are all bounded by triangles.  A \defin{generalized near-triangulation} is a plane graph whose inner faces are bounded by triangles. Note that a generalized near triangulation may have multiple components, cut vertices, and bridges.

In several places we will make use of the following observation, which is really a statement about the triangulation contained in $xyz$.
\begin{obs}\label{useful_little_guy}
  Let $H$ be a generalized near-triangulation and let $xyz$ be a cycle in $H$.  Then,
  \begin{compactenum}
    \item If the interior of $xyz$ contains at least one vertex of $H$, then each of $x$, $y$, and $z$ has at least one neighbour in the interior of $xyz$.

    \item If the interior of $xyz$ contains at least two vertices of $H$, then at least two of $x$, $y$, and $z$ have at least two neighbours in the interior of $xyz$.
  \end{compactenum}
\end{obs}





For a plane graph $H$, we use the notation $B(H)$ to denote the vertex set of the outer face of $H$ and define $I(H):=V(H)\setminus B(H)$.  The vertices in $B(H)$ are \defin{boundary vertices} of $H$ and the vertices in $I(G)$ are \defin{inner vertices} of $H$. For any vertex $v$ of $H$, the \defin{inner neighbourhood} of $v$ in $H$ is defined as $N_H^+(v):=N_H(v)\cap I(H)$, the vertices in $N^+_H(v)$ are \defin{inner neighbours} of $v$ in $H$, and $\deg^+_H(v)=|N^+_H(v)|$ is the \defin{inner degree} of $v$ in $H$.

Let $G$ be a triangulation.  Our procedure for constructing a connected dominating set $X$ begins with an incremental phase that eats away at $G$ ``from the outside.'' The process of constructing $X$ is captured by the following definition:   A vertex subset $X\subseteq V(G)$ is \defin{outer-domatic} if it can be partitioned into non-empty subsets $\Delta_0,\Delta_1,\ldots,\Delta_{r-1}$ such that
\begin{compactenum}[(P1)]
    \item $\Delta_0\subseteq B(G)$; \label{outer_face}
    \item $\Delta_i\subseteq B(G-(\bigcup_{j=0}^{i-1}\Delta_j))$ for each $i\in\{1,\ldots,r-1\}$; and \label{incremental}
    \item $G-(\bigcup_{j=0}^{r-1}\Delta_j)$ is outerplane. \label{outerplanar}
\end{compactenum}

\begin{lem}\label{outer_domatic}
    Let $G$ be a triangulation.  Then any outer-domatic $X\subseteq V(G)$ is a connected dominating set of $G$.
\end{lem}

\begin{proof}
  Suppose $X$ is outer-domatic and let $\Delta_0,\ldots,\Delta_{r-1}$ be the corresponding partition of $X$.  For each $i\in\{1,\ldots,r\}$, let $X_i:=\bigcup_{j=0}^{i-1} \Delta_i$.  First observe that, since $\Delta_0\subseteq B(G)$ is non-empty, $X_i$ contains at least one vertex of $B(G)$, for each $i\in\{1,\ldots,r\}$. We claim that,
  \begin{compactenum}[(P1)]\setcounter{enumi}{3}
    \item for each $i\in\{2,\ldots,r\}$ each vertex in $B(G-X_{i-1})$ is adjacent to some vertex in $X_{i-1}$. \label{adjacent}
  \end{compactenum}
  Indeed, for any $i\in\{2,\ldots,r\}$ each vertex $v\in B(G-X_{i-1})$ is either in $B(G)$ or adjacent to a vertex in $X_{i-1}$. Even in the former case, (P1) ensures that $v$ is adjacent to a vertex in $X_1=\Delta_0\subseteq X_{i-1}$, because $G[B(G)]$ is a clique.

  We now prove, by induction on $i$, that $G[X_i]$ is connected, for each $i\in\{1,\ldots,r\}$.
  The fact that $G[B(G)]$ is a clique and \pref{outer_face} implies that $G[X_1]=G[\Delta_0]$ is connected. For each $i\in\{2,\ldots,r\}$, the assumption that $G[X_{i-1}]$ is connected, \pref{incremental}, and \pref{adjacent} then imply that $G[X_i]=G[X_{i-1}\cup\Delta_{i-1}]$ is connected.

  In particular $G[X_r]=G[X]$ is connected.  Finally, \pref{adjacent}, with $i=r$ and \pref{outerplanar} implies that $N_G(X_r)=B(G-X_r)=V(G-X_r)$, so $X_r=X$ is a dominating set of $G$.
\end{proof}

We will present two algorithms that grow a connected dominating set in small batches $\Delta_0,\Delta_1,\ldots,\Delta_{r-2}$ that result in a sequence of sets $X_1,\ldots,X_{r-1}$ where $X_{i}=\bigcup_{j=0}^{i-1}\Delta_j$.  Each of these algorithms is unable to continue once they reach a point where each vertex in $B(G-X_i)$ has inner-degree at most $1$ in $G-X_i$.  We begin by studying the graphs that cause this to happen.

\subsection{Critical Graphs}

A generalized near-triangulation $H$ is \defin{critical} if $\deg^+_H(v)\le 1$ for each $v\in B(H)$. We say that an inner face of $H[B(H)]$ is \defin{marked} if it contains an inner vertex of $H$.

\begin{figure}[htbp]
    \centering
    \includegraphics[page=1]{figs/critical}
    \caption{Some critical graphs.}
    \label{critical_fig}
\end{figure}


\begin{lem}\label{critical_structure}
    Let $H$ be a critical generalized near-triangulation. Then each $f$ face of $H[B(H)]$ contains at most one vertex of $I(H)$ and this vertex is adjacent to every vertex of $f$.
\end{lem}

\begin{proof}
  Let $B:=B(H)$ and $I:=I(H)$.
  By definition, the graph $H[B]$ is outerplanar.  Consider some marked face $f$ of $H[B]$.  This face is marked because it contains at least one vertex in $I$.  Since $H$ is a triangulation, there is an edge $vx$ in $H$ with $v\in B$ on the boundary of $f$ and $x\in I$ in the interior of $f$. Since $H$ is a generalized near-triangulation and $x$ is an inner vertex of $H$, the edge $vx$ is on the boundary of two faces $vxv_1$ and $vxv_{k-1}$ of $H$ with $v_1\neq v_{k-1}$.  Since $\deg^+_H(v)=1$, each of $v_1$ and $v_{k-1}$ are in $B$.  By the same argument, $H$ contains a face $v_1xv_2$ with $v_2\in B$, $v_2\neq v_1$, and repeating this argument shows that $v,v_1,v_2,\ldots,v_{k-1}$ is the cycle in $H[B]$ that bounds $f$.  Therefore, $f$ contains exactly one vertex $x$ of $I$ and $x$ is adjacent to each vertex of $f$.
\end{proof}

\begin{lem}\label{base_case}
    Let $H$ be a critical generalized near-triangulation. Then $|B(H)|\ge 3|I(H)|$ and there exists $\Delta\subseteq B(H)$ of size at most $|I(H)|$ that dominates $I(H)$.
\end{lem}

\begin{proof}
  Let $B:=B(H)$ and $I:=I(H)$.  If $I$ is empty then the result is trivially true, by taking $\Delta:=\emptyset$, so we now assume that $I$ is non-empty.  By \cref{critical_structure}, $H$ is formed from the outerplanar graph $H[B]$ by adding $|I|$ stars, one in the interior of each marked face of $H[B]$.  Since $\deg_H^+(v)=1$ for each $v\in B$, each vertex of $H[B]$ is on the boundary of exactly one marked face.  For each vertex $w\in I$, the marked face $f$ of $H[B]$ that contains $w$ has at least $3$ vertices, which do not belong to any other marked face. Therefore $|B|\ge 3|I|$ and by choosing one vertex from each marked face of $H[B]$ we obtain the desired set $\Delta$.
\end{proof}


\section{A Simple Algorithm}
\label{warm_up}

We start with the simplest possible greedy algorithm, that we call $\textsc{SimpleGreedy}(G)$, to choose $\Delta_0,\ldots,\Delta_{r-1}$.  Suppose we have already chosen $\Delta_0,\ldots,\Delta_{i-1}$ for some $i\ge 0$ and we now want to choose $\Delta_i$.  Let $X_i:=\bigcup_{j=0}^{i-1}\Delta_j$, let $G_i:=G-X_i$, and let $v_i$ be a vertex in $B(G_i)$ that maximizes $\deg^+_{G_i}(v_i)$.  During iteration $i\ge 0$, there are only two cases to consider:
\begin{compactenum}[{[}g1{]}]
    \item If $\deg^+_{G_i}(v_i)\ge 2$ then we set $\Delta_i\gets\{v_i\}$.
    \item If $\deg^+_{G_i}(v_i)\le 1$ for all $v\in B(G_i)$ then $G_i$ is critical and this is the final step, so $r:=i+1$.  By \cref{base_case}, there exists $\Delta_i\subseteq B(G_i)$ of size at most $|I(G_i)|$ that dominates $I(G_i)$. Then $X_r:=X_{r-1}\cup\Delta_{i}$ and we are done.
\end{compactenum}

\begin{thm}\label{simple_greedy}
  When applied to an $n$-vertex triangulation $G$,  $\textsc{SimpleGreedy}(G)$ produces a connected dominating set $X_r$ of size at most $(4n-9)/7$.
\end{thm}

\begin{proof}
By the choice of $\Delta_0,\ldots,\Delta_{r-1}$, $X_r$ is an outer-domatic subset of $V(G)$ so, by \cref{outer_domatic}, $X_r$ is a connected dominating set of $G$.  All that remains is to analyze the size of $X_r$.  For each $i\in\{1,\ldots,r\}$, let $D_i:=N_G[X_i]$ be the subset of $V(G)$ that is dominated by $X_i$, let $I_i:=V(G)\setminus D_i$ be the subset of $V(G)$ not dominated by $X_i$, and let $B_i:=N_G(I_i)$ be the vertices of $G$ that have at least one neighbour in each of $X_i$ and $I_i$.  We use the convention that $D_0:=B(G)$.

First observe that, for $i\in\{0,\ldots,r-2\}$, $|D_{i+1}|\ge |D_i|+\deg_{G_i}^+(v_i)$ since $D_{i+1}\supseteq D_i$ and $D_{i+1}$ contains the $\deg_{G_i}^+(v_i)$ inner neighbours of $v_i$ in $G_i$.  Therefore
\[
    |D_{r-1}| \ge |D_0| + \sum_{i=0}^{r-2} \deg_{G_i}^+(v_i) \ge 3 + \sum_{i=0}^{r-2} 2 =  2r+1 \enspace . \label{double_d}
\]
Since $D_{r-1}$ and $I_{r-1}$ partition $V(G)$,
\begin{equation}
  n = |D_{r-1}| + |I_{r-1}| \ge 2r+1 + |I_{r-1}|  \enspace . \label{c1}
\end{equation}

Since $X_{r-1}$ and $B_{r-1}$ are disjoint and $D_{r-1}\supseteq B_{r-1}\cup X_{r-1}$, we have $|D_{r-1}|\ge |X_{r-1}| + |B_{r-1}|=r-1+|B_{r-1}|$.  Therefore,
\begin{align}
    n & = |D_{r-1}| + |I_{r-1}| \ge r-1 + |B_{r-1}| + |I_{r-1}| \ge r - 1 + 4|I_{r-1}| \enspace , \label{c2}
\end{align}
where the last inequality follows from \cref{base_case}.

The final dominating set $X_r$ has size $|X_r| = |X_{r-1}| + \Delta_{r-1} = r - 1 +|I_{r-1}|$, so the size of $|X_r|$ can be upper-bounded by maximizing $r-1+|I_{r-1}|$ subject to \cref{c1,c2}.  More precisely, by setting $x:=r$ and $y:=|I_{r-1}|$, the maximum size of $X_r$ is upper-bounded by the maximum value of $x-1+y$ subject to the constraints
\begin{align*}
x,y \ge 0 \\
  x - 1 + 4y & \le n \\
  2x + 1 + y & \le n
\end{align*}

% r\le (n+1-I_{r-1})/2$ and $r+4|I_{r-1}|\le n$.
This is an easy linear programming exercise and the maximum value of $X_{r}$ is obtained when $r=(3n-5)/7$ and $|I_{r-1}|=(n+3)/7$, which gives
$|X_r| \le (4n-9)/7$.
\end{proof}

% \subsection{An $O(n)$ Time Implementation}
%
% We note that the implementation of $\textsc{SimpleGreedy}(G)$ is even simpler than the definition given above.  Nothing special needs to be done for the critical graph $G_{r-1}$.  Repeatedly selecting a vertex of maximum innner-degree and removing it will produce a dominating set of size exactly $|I_{r-1}|$.  Thus, $\textsc{SimpleGreedy}(G)$ has a simple linear time implementation.  In this implementation, each vertex $v$ stores a value $d_v$ which is initially set to $\deg_G(v)$.  For the three vertices on the outer face of $G$, $d_v$ is initially set to $\deg_G(v)-2$.  In general, $d_v$ is kept updated so that it is always equal to the inner-degree of $v$ in $G_i$.
%
% Besides the data structure used for representing the triangulation $G$, each vertex $v$ also participates in a doubly-linked list $L_{d_v}$  that stores all the vertices with the same $d_v$ value.  A global doubly-linked list $L$ then stores all the lists $L_d$ such that $L_d$ is non-empty, sorted by increasing order of $d$.   Extracting a vertex of maximum inner-degree can then be done in constant time and the total time spent moving vertices between different lists in $L$ is proportional to the number of edges of $G$. Thus, the entire algorithm can be implemented in $O(n)$ time.

\section{A Better Algorithm: Proof of \cref{main_result2}}
\label{full_result}

Next we devise an algorithm that produces a smaller connected dominating set than what $\textsc{SimpleGreedy}(G)$ can guarantee.  This involves a more careful analysis of the cases in which \textsc{SimpleGreedy} is forced to take a vertex $v_i$ with $\deg^+_{G_i}(v_i)=2$.  We will show that in most cases, any time the algorithm is forced to choose a vertex $v$ that has inner-degree $2$ in $G_i$, this can immediately be followed by choosing a vertex $w$ that has inner-degree at least $3$ in $G_i-v$.  This is explained in \cref{combo}.

When this is no longer possible, the algorithm will be forced to directly handle a graph $G_i$ in which $\deg^+_{G_i}(v)\le 2$ for all $v\in B(G_i)$ and $G_i-(B_i)$ is critical.  In \cref{two_critical_section} we explain how this can be done using a set $X_{r-1}$ whose size depends only on $G_i-B(G_i)$.  The results in \cref{two_critical_section} require that the graph $G_i-B(G_i)$ not have any vertices of degree less than $2$.  The steps required to eliminate degree-$1$ and degree-$0$ vertices from $G_i-B(G_i)$ are explained in \cref{zero_kill_sec,one_kill_sec}.

% In this section, we will show that during any step one of the following cases occurs.

% \pat{Give overview of algorithm here.}

\subsection{Dom-Minimal Dom-Respecting Graphs}

We begin by identifying unnecessary vertices and edges that can appear in the graphs $G_1,\ldots,G_{r-1}$ during the construction of $X$.   We say that a near-triangulation $H$ is \defin{dom-minimal} if
\begin{compactenum}[({DM}1)]
    \item each vertex $v\in B(H)$ has $\deg^+_H(v)\ge 1$;  \label[dm]{bad_vertex}
    \item for each $v\in B(H)$ with $\deg^+_H(v)=1$,  $H[N_H[v]]$ is isomorphic to $K_4$; and \label[dm]{inner_degree_1}
    \item each edge $vw$ on the boundary of the outer face of $H$ is also on the boundary of some inner face $vwx$ of $H$, where $x\in I(H)$. \label[dm]{bad_edge}
    % \saeed{this sentence looks unclear }
\end{compactenum}
We say that a generalized near-triangulation $H$ is \defin{dom-minimal} if each of its biconnected components are dom-minimal.

\begin{obs}\label{bridgeless}
    Any dom-minimal generalized near-triangulation $H$ is bridgeless.
\end{obs}

\begin{proof}
   If $vw$ is a bridge in $H$ then both $v$ and $w$ are in $B(H)$.  Since $vw$ is a bridge in $H$, there is no path $vxw$ in $H$ and hence no inner face $vwx$ in $H$. Thus $H$ does not satisfy \cref{bad_edge}.
\end{proof}

Let $H$ and $H'$ be two generalized near-triangulations.  We say that $H'$ \defin{dom-respects} $H$ if
\begin{compactenum}[({DP}1)]
  \item $B(H')\subseteq B(H)$; \label[dp]{boundary_subset}
  % \item $N^+_{H'}(v)\subseteq N^+_H(v)$ for all $v\in B(H')$; \label[dp]{same_inner_neighbourhood}
  \item $I(H')=I(H)$; and \label[dp]{same_inner_vertices}
  \item $N_{H'}(v)\cap I(H)\subseteq N_H(v)\cap I(H)$ for all $v\in V(H')$. \label[dp]{same_neighbourhood}
\end{compactenum}

\begin{obs}
  Let $H$ and $H'$ be generalized near-triangulations where $H'$ dom-respects $H$ and let $\Delta'$ be a subset of $V(H')$ that dominates $I(H')$ in $H'$.  Then $\Delta'$ dominates $I(H)$ in $H$.
\end{obs}

\begin{proof}
  By \cref{same_inner_vertices}, $I(H)=I(H')$. For each $w\in I(H)=I(H')$, $w\in \Delta'$ or there exists an edge $vw\in E(H')$ with $v\in\Delta'$.  In the latter case, $vw\in E(H)$ by \cref{same_neighbourhood}, so $\Delta'$ dominates $w$.
\end{proof}

\begin{lem}\label{dom_minimal}
  For any generalized near-triangulation $H$, there exists a dom-minimal generalized near-triangulation $H'$ that dom-respects $H$.
\end{lem}

\begin{proof}
  The proof is by induction on $|V(H)|+|E(H)|$.  If $H$ is already dom-minimal, then setting $H'=H$ satisfies the requirements of the lemma, so assume that $H$ is not dom-minimal.  Since \cref{boundary_subset,same_inner_vertices,same_neighbourhood} are transitive relations, the dom-respecting relation is transitive: If $H'$ dom-respects $H^*$ and $H^*$ dom-respects $H$, then $H'$ dom-respects $H$.  Therefore, it is sufficient to find $H^*$ with fewer edges or fewer vertices than $H$ that dom-respects $H$, and the inductive hypothesis provides the desired dom-minimal graph $H'$ that dom-respects $H^*$ and $H$.

  If $H$ contains a vertex $v\in B(H)$ with $\deg^+_H(v)=0$ then $H-v$ is a generalized near-triangulation, $B(H-v)\subset B(H)$, $I(H-v)=I(H)$, and $N_{H-v}(v)\cap I(H)=N_{H}(v)\cap I(H)$ for all $v\in V(H-v)$. Therefore $H-v$ dom-respects $H$ and has fewer vertices than $H$ so we can apply the inductive hypothesis and be done.  We now assume that $\deg^+_H(v)\ge 1$ for all $v\in B(H)$.  Since $H$ is not dom-minimal then $H$ contains a biconnected component $C$ that is not dom-minimal. (See \cref{minimal_fig}.)
  \begin{figure}
    \centering
    \begin{tabular}{ccc}
      \includegraphics[page=1]{figs/minimal} &
      \includegraphics[page=3]{figs/minimal} &
      \includegraphics[page=5]{figs/minimal} \\
      $\Downarrow$ & $\Downarrow$ & $\Downarrow$ \\
      \includegraphics[page=2]{figs/minimal} &
      \includegraphics[page=4]{figs/minimal} &
      \includegraphics[page=6]{figs/minimal}
    \end{tabular}
    \caption{Three cases on the way to making $H$ dom-minimal.}
    \label{minimal_fig}
  \end{figure}
  \begin{compactenum}
    \item \cref{bad_edge}: If there exists an edge $vw$ on the outer face of $C$ that is not incident to any inner face $vwx$ with $x\in I(C)$ then $H-vw$ is a generalized near-triangulation, $B(H-vw)=B(H)$, and $I(H-vw)=I(H)$, and $N_{H-vw}(v)\cap I(H)=N_{H}(v)\cap I(H)$ for all $v\in V(H-vw)$. Therefore, $H-vw$ dom-respects $H$ and has few edges than $H$. (This includes the case where $C$ consists of the single edge $vw$.)

    \item \cref{bad_vertex}: If there exists a vertex $v\in B(C)$ with $\deg^+_C(v)=0$ then $v$ is incident to an edge $vw$ that is on the outer face of $C$ and on the outer face of $H$. Since $\deg^+_C(v)=0$, $vw$ is not incident to any inner face $vwx$ with $x\in I(C)$ and we can proceed as in the previous case.

    \item \cref{inner_degree_1}: If there exists a vertex $v\in B(C)$ with $\deg^+_C(v)=1$ then $H$ contains faces $xvw$ and $vyw$ where $x$ and $y$ are on the outer face of $C$ and $w$ is an inner vertex. If $H[N_H[v]]$ is not isomorphic to $K_4$, then $xy\not\in E(H)$.  In this case, let $H^\star$ be the graph obtained from $H$ by replacing the edges $xv$ and $vy$ with the edge $xy$. Then $H^\star$ is a generalized near-triangulation, $B(H^*)=B(H)$, $I(H^*)=I(H)$, and $N_{H^*}(v)\cap I(H)\subseteq N_{H}(v)\cap I(H)$.  Therefore $H^*$ dom-respects $H$ and has fewer edges than $H$. \qedhere
  \end{compactenum}
\end{proof}

\subsection{Finding a $2$--$3$ Combo}
\label{combo}

Next we show that, in most cases our algorithm for constructing a connected dominating set is not forced to choose a single vertex of inner-degree $2$. Instead, it can choose a pair $v,w$ such that $\deg^+_H(v)=2$ and $\deg^+_{H-v}(w)\ge 3$. Note that the next two lemmas each consider a graph $H$ that is a near triangulation, not a generalized near-triangulation.

\begin{lem}\label{chord_incident}
  let $H$ be a dom-minimal near-triangulation and let $v_0$ be a vertex in $B(H)$ with $|N_H(v_0)\cap B(H)|\ge 3$.  Then $\deg^+_H(v_0)\ge 2$.  In other words, if $v_0$ is incident to a chord of the outerplane graph $H[B(H)]$, then $v_0$ is incident to at least two inner vertices of $H$.
\end{lem}

\begin{figure}[htbp]
  \centering
  \includegraphics{figs/chord_incident}
  \caption{The proof of \cref{chord_incident}}
  \label{chord_incident_fig}
\end{figure}
\begin{proof}
  Refer to \cref{chord_incident_fig}
  Since $H$ is a near-triangulation its outer face is bounded by a cycle $v_0,\ldots,v_{k-1}$.  Let $a:=\min\{i\in\{2,\ldots,k-2\}:v_0v_i\in E(H)\}$ and $b:=\max\{i\in\{2,\ldots,k-2\}:v_0v_i\in E(H)\}$. (Possibly $a=b$, but both $a$ and $b$ are well-defined since $|N^+_H(v_0)|\ge 3$.)   Since $H$ is dom-minimal, the edge $v_0v_1$ is on the boundary of an inner face $v_0v_1x$ of $H$ where $x$ is an inner vertex of $H$, by \cref{bad_edge}.  Since $H$ is dom-minimal, the edge $v_{k-1}v_0$ is on the boundary of an inner face $v_{k-1}v_0y$ of $H$ where $y$ is an inner vertex of $H$, by \cref{bad_edge}.  Then $x$ is in the interior of the face of $H[B(H)]$ bounded by the cycle $v_0,v_1,\ldots,v_a$ and $y$ is in the interior of the face of $H[B(H)]$ bounded by the cycle $v_0,v_b,\ldots,v_{k-1}$.  Therefore, $x\neq y$ and $N^+_H(v_0)\supseteq\{x,y\}$ so $\deg^+_H(v_0)\ge 2$.
\end{proof}

\begin{lem}\label{degree_2_outer_neighbour}
  Let $H$ be a dom-minimal near-triangulation. Then either:
  \begin{compactenum}
    \item $H$ is isomorphic to $K_4$;
    % \item $H$ consists of a cycle on the vertices in $B(H)$ plus one dominant inner vertex $w$; or
    \item each vertex $w\in B(H-B(H))$ has a neighbour $v$ in $B(H)$ with $\deg^+_H(v)\ge 2$.
  \end{compactenum}
\end{lem}

\begin{proof}
  If $I(H)=\emptyset$ then the second condition of the lemma is trivially satisified, so there is nothing to prove. Otherwise, let $w$ be any vertex in $B(H-B(H))$ and let $vw$ be an edge of $H$ with $v\in B(H)$.
  % If $H$ is the empty graph, then there is nothing to prove. Otherwise $H$ is non-empty, so $B(H)$ is non-empty. Let $v$ be a vertex in $B(H)$.
  % By \cref{bad_edge}, $\deg^+_H(v)\ge 1$ so $H$ contains an edge $vw$ with $w\in I(H)$.
  % Consider the face $f:=v_0,\ldots,v_{k-1}$ of the outerplane graph $H[B(H)]$ that contains $w$.


   \begin{figure}[htbp]
     \centering
     \begin{tabular}{ccc}
       \includegraphics[page=1]{figs/outer_2_2} &
       \includegraphics[page=2]{figs/outer_2_2} &
       \includegraphics[page=3]{figs/outer_2_2}
     \end{tabular}
     % \pat{Redo this figure.}
     \caption{The proof of \cref{degree_2_outer_neighbour}.}
     \label{degree_2_outer_neighbour_fig}
   \end{figure}

   Refer to \cref{degree_2_outer_neighbour_fig}.
   By \cref{inner_degree_1}, $\deg^+_H(v)\ge 2$ or $H[N_H[v_0]]$ is isomorphic to $K_4$. In the former case the vertex $w$ satisfies the second condition of the lemma.  In the latter case, let $x$ and $y$ be the two neighbours of $v$ on the outer face of $H$, so $H[\{v,w,x,y\}]$ is isomorphic to $K_4$.    If the edge $xy$ is not on the outer face of $H$ then, by \cref{chord_incident}, $\deg^+_H(x),\deg^+_H(y)\ge 2$ so $w$ satisfies the second condition and we are done.  Otherwise, if $I(H)=\{w\}$ then $V(H)=\{v,w,x,y\}$ and $H$ is isomorphic to $K_4$ and we are done.  Otherwise $I(H)$ contains at least one vertex $w'\neq w$.  Since $\deg^+_H(v)=1$, the cycle $vxwy$ has no vertices of $H$ in its interior (by \cref{useful_little_guy}), so $I(H)$ contains vertices in the interior of $xyw$.  But then \cref{useful_little_guy} implies that $\deg^+_H(x),\deg^+_H(y)\ge 2$.
\end{proof}

Note that the next three lemmas consider the case where $H$ is a generalized near triangulation.  The following lemma is illustrated in \cref{2_3_figure}.

\begin{figure}
  \centering
  \includegraphics[page=1]{figs/2_3_figure}
  \includegraphics[page=2]{figs/2_3_figure}
  \includegraphics[page=3]{figs/2_3_figure}
  \caption{Removing an inner-degree $2$ vertex $v$ is immediately followed by removing an inner-degree $3$ vertex $w$.}
  \label{2_3_figure}
\end{figure}


\begin{lem}\label{really_good}
  Let $H$ be a dom-minimal generalized near-triangulation.  Then either:
  \begin{compactenum}[(1)]
    \item $H-B(H)$ is critical; \label[p]{two_critical}
    \item $B(H)$ contains a vertex $v$ with $\deg^+_H(v)\ge 3$; or \label[p]{degree_three}
    \item $H$ contains distinct vertices $v_0$, $v_j$, and $w$ such that
    \begin{compactenum}[(a)]
      \item $v_0\in B(H)$ and $\deg^+_H(v_0)=2$;
      \item $w\in B(H-v_0)$ and $\deg^+_{H-v_0}(w)\ge 3$; and
      \item $v_j\in B(H)$ and $N^+_H(v_j) \subseteq N_H[w] $.
    \end{compactenum}
    \label[p]{two_three_pair}
  \end{compactenum}
\end{lem}

\begin{proof}
  We will assume that $H$ does not satisfy \cref{two_critical} or \cref{degree_three} and show that $H$ must satisfy \cref{two_three_pair}.  Since $H-B(H)$ is not critical, $B(H-B(H))$ contains a vertex $w$ with $\deg_{H-B(H)}(w)\ge 2$.

  Let $C$ be the biconnected component of $H$ that contains $w$.  Then $C$ is a near-triangulation and we can apply \cref{degree_2_outer_neighbour} to $C$ and $w$. The first alternative in \cref{degree_2_outer_neighbour} is incompatible with the assumption that $\deg^+_{H-B(H)}(w)\ge 2$.  Therefore, we conclude that $N_H(w)\cap B(H)$ contains a vertex $v_0$ with $\deg^+_H(v_0)\ge 2$.  Since $H$ does not satisfy \cref{degree_three}, $\deg^+_H(v_0)< 3$, so $\deg^+_H(v_0)=2$.  Refer to \cref{really_good_fig}
  \begin{figure}
    \centering
    \begin{tabular}{ccc}
      \includegraphics[page=1]{figs/really_good} &
      \includegraphics[page=2]{figs/really_good} &
      \includegraphics[page=3]{figs/really_good}
    \end{tabular}
    \caption{The proof of \cref{really_good}.
    % \hussein{I think, it might be better if we draw another neighbor of w to its right.}\pat{Do you mean in the third picture?}
    }
    \label{really_good_fig}
  \end{figure}

  Let $v_0,\ldots,v_{k-1}$ be the cycle that bounds the inner face $f$ of $H[B(H)]$ that contains $w$ in its interior. In the remainder of this proof, alls subscripts on $v$ are implicitly modulo $k$.  Since $H$ is a near-triangulation, $H$ contains triangles $v_0v_1x$ and $v_{k-1}v_0 y$ with $x$ and $y$ in the interior of or on the boundary of $f$.  Since $f$ is a face of $H[B(H)]$ each of $x$ and $y$ is in the interior of $f$.  At least one of $x$ or $y$ is equal to $w$, say $x$, since otherwise $\deg^+_H(v_0)\ge 3$.  Therefore $v_0v_1 w$ is an inner face of $H$.

  Let $j\ge 1$ be the maximum integer such that $v_{a-1}v_{a}w$ is an inner face of $H$ for all $a\in\{1,\ldots,j\}$.  Note that $j\le k-1$ since, otherwise, the component of $H-B(H)$ that contains $w$ contains only a single vertex, contradicting the fact that $\deg^+_{H-B(H)}(w)\ge 2$.
  % \footnote{In fact, $j\in\{0,1\}$ by \cref{inner_degree_1}, but this is not important for the current argument.}

  Since $H$ is a near-triangulation and $f$ is a face of $H[B(H)]$, $H$ has some face $v_j v_{j+1} z$ with $z$ in the interior of $f$.  By the definition of $j$, $z\neq w$.  Therefore, $N_H^+(v_j)\supseteq \{w,z\}$ and, since $\deg^+_H(v_j)\le 2$, $N_H^+(v_j))= \{w,z\}$.
  % Therefore $N^+_H(v_j)\subseteq N_H[w]$.
  Since $f$ is a face of $H[B(H)]$, the only neighbours of $v_j$ in $B(H)$ are $v_{j-1}$ and $v_{j+1}$. Since $H$ is a near-triangulation and $\deg^+_H(v_j)\le 2$, this implies that $w v_j z$ is a face of $H$.  In particular $wz$ is an edge of $H$.

  All that remains is to show that $\deg^+_{H-v_0}(w)\ge 3$.  First, observe that $z$ is in $B(H-B(H))$, so $z$ does not contribute to $\deg^+_{H-B(H)}(w)$. We claim that $z$ is in $I(H-v_0)$, so $z$ does contribute to $\deg^+_{H-v_0}(w)$.   Indeed, the only other possibility is that $z$ is adjacent to $v_0$.  In this case, consider the cycle $C:=v_0,\ldots,v_j,z$.  This cycle has $w$ in its interior. The vertices of $N^+_{H-B(H)}(w)$ must be in the interior of $C$. For each $a\in\{1,\ldots,j\}$, $v_{a-1}v_a w$ is a face of $H$ and $v_jwz$ is a face of $H$, so the cycle $D:=v_0,\ldots,v_j,z,w$ does not contain any vertices of $N^+_{H-B(H)}(w)$ in its interior.  Therefore, the vertices in $N^+_{H-B(H)}(w)$ must be in the interior of the cycle $\overline{D}:=v_0,w,z$.  By \cref{useful_little_guy}, $v_0$ is adjacent to some vertex in $I(H)\setminus\{w,z\}$. But this is not possible since it would imply that $\deg^+_H(v_0)\ge 3$.  Therefore $v_0$ is not adjacent to $z$, so $z$ is in the interior of $H-v_0$ and $N^+_{H-v_0}(w)\supseteq N^+_{H-B(H)}(w)\uplus\{z\}$, so $\deg^+_H(w)\ge 3$.
\end{proof}


The following is a restatement of \cref{really_good} in language that is more useful in the description of an algorithm for constructing a connected dominating set.

\begin{cor}\label{really_good_cor}
  Let $H$ be a dom-minimal generalized near-triangulation.  Then either:
  \begin{compactenum}[(1)]
    \item $H-B(H)$ is critical;
    \item there is a vertex $v\in B(H)$ and a dom-respecting subgraph $H'$ of  $H-v$ with $|H'|\le |H|-1$ and $|B(H')|\ge |B(H)|+2$; or
    \item there is an edge $vw\in E(H)$ with $v\in B(H)$, $w\in B(H-B(H))$, and a dom-respecting subgraph $H'$ of $H-\{v,w\}$ with $|H'|= |H|-3$ and $|B(H')|=|B(H)|+2$.
  \end{compactenum}
\end{cor}

\begin{proof}
  In the second case, the graph $H':=H-v$ has $|H'|=|H|-1$ and $|B(H')|\ge |B(H)|+2$. In the third case, the graph $H':=H-\{v_0,w,v_j\}$ has $|H'|=|H|-3$ and $|B(H')|=|B(H)|+2$.
\end{proof}


\subsection{Eliminating Inner Leaves}
\label{one_kill_sec}

Next we show that, even when all vertices in $B(H)$ have inner-degree at most $2$ and $H-B(H)$ is critical, we can still efficiently dominate degree-$1$ vertices in $H-B(H)$.


\begin{lem}\label{leaf_killer}
  Let $H$ be a dom-minimal generalized near-triangulation such that $\deg^+_H(v)\le 2$ for all $v\in B(H)$, $H-B(H)$ is critical, and $H-B(H)$ contains a vertex $w$ with $\deg_{H-B(H)}(w)=1$.  Then there exists $v\in B(H)$ and a dom-respecting subgraph $H'$ of $H-v$ such that $|H'|\le |H|-3$ and $|B(H')|\le |B(H)|-1$.
\end{lem}

\begin{proof}
  Refer to \cref{killing_a_leaf}.
  Let $x$ be the unique neighbour of $w$ in $H-B(H)$. Since $x$ and $w$ are vertices of $H-B(H)$, $x,w\in I(H)$.  Since $w$ is an inner vertex in a near-triangulation, it is incident to $t\ge 3$ faces $v_iv_{i+1}w$ for $i\in\{0,\ldots,v_{t}\}$, with $v_0=v_t=x$.  Since $\deg_{H-B(H)}(w)=1$, $v_1,\ldots,v_{t-1}\in B(H)$.  Therefore, $H$ contains no edge $v_i v_{i+r}$ for any $i\in\{0,\ldots,t-r\}$ and any $r\ge 2$.  Therefore, for each $i\in\{1,\ldots,t-1\}$, the only two inner faces of $H$ that include $v_i$ are $v_{i-1}v_iw$ and $v_iv_{i+1}w$.  Therefore $N^+_H(v_i)=\{w\}$ for each $i\in\{2,\ldots,t-2\}$ and $N^+_H(v_1)=N^+_H(v_{t-1})=\{x,w\}$.\footnote{In fact, \cref{inner_degree_1} implies that $t=3$, but this is not important for this proof.}

  \begin{figure}[htbp]
    \centering
    \begin{tabular}{ccc}
      \includegraphics[page=1]{figs/killing_a_leaf} &
      \includegraphics[page=2]{figs/killing_a_leaf} &
      \includegraphics[page=3]{figs/killing_a_leaf} \\
      $H$ & $H-v$ & $H'$
    \end{tabular}
    \caption{The proof of \cref{leaf_killer}.}
    \label{killing_a_leaf}
  \end{figure}

  Let $v:=v_1$. Apply \cref{dom_minimal} to $H-v$ to get a dom-minimal graph $H'$ that dom-respects $H-v$.  Then $w,x\in B(H-v)$.  Since $H'$ is dom-minimal, $v_1,\ldots,v_{t-1}\not\in V(H')$, by \cref{bad_vertex}.  Therefore $N_H(w)\cap V(H')=\{v_t\}=\{x\}$. Since $H'$ is dom-minimal, $w\not\in V(H')$, by \cref{bad_vertex}. Therefore $V(H')\subseteq V(H)\setminus\{v_1,\ldots,v_{t-1},w\}$, so $|V(H')|\le |H|-t\le |H|-3$.  Finally, $B(H-v)\subseteq B(H)\setminus \{v\}\cup\{w,x\}$. By \cref{boundary_subset}, $B(H')\subseteq B(H-v)\setminus\{v_1,\ldots,v_{t-1},w\}\cup\{x\}$, so $|B(H')|\le |B(H)|+2-t\le |B(H)|-1$.
\end{proof}

\subsection{Eliminating Inner Isolated Vertices}
\label{zero_kill_sec}

We now show that, even when all vertices in $B(H)$ have inner-degree at most $2$, $H-B(H)$ is critical, and $H-B(H)$ has no degree-$1$ vertices, we can still efficiently dominate degree-$0$ vertices in $H-B(H)$.


\begin{lem}\label{degree_zero_killer}
  Let $H$ be a dom-minimal generalized near-triangulation such that $\deg^+_H(v)\le 2$ for all $v\in B(H)$, $H-B(H)$ is critical, $H-B(H)$ contains a vertex $w$ with $\deg_{H-B(H)}(w)=0$ but does not contain any vertex $w'$ with $\deg_{H-B(H)}(w')=1$.  Then at least one of the following is true:
  \begin{compactenum}[(a)]
    \item there exists $v\in B(H)$ and a graph $H'$ that dom-respects $H-v$ such that $|H'|\le |H|-3$ and $|B(H')|\le |B(H)|-1$; \label[p]{kill_pair}
    \item there exists $v\in B(H)$ and a graph $H'$ that dom-respects $H-v$ such that $|H'|\le |H|-4$ and $|B(H')|\le |B(H)|-3$; or \label[p]{kill_k4}
    \item there exists $u,v\in B(H)$ and a graph $H'$ that dom-respects $H-\{u,v\}$ such that $|H'|\le |H|-6$ and $|B(H')|\le |B(H)|-1$. \label[p]{kill_double}
  \end{compactenum}
\end{lem}

\pat{There is an issue with this lemma/proof. I'm working on it now.}
\begin{proof}
  By definition $N_H(w)\subseteq B(H)$ and, since $w$ is an inner vertex in a generalized near triangulation $|N_H(w)|\ge 3$.

  There are two cases to consider (see \cref{isolated_fig}):
  \begin{figure}[htpb]
    \centering
    \begin{tabular}{cc}
      \includegraphics[page=1]{figs/isolated} &
      \includegraphics[page=3]{figs/isolated} \\
      $\Downarrow$ & $\Downarrow$ \\
      \includegraphics[page=2]{figs/isolated} &
      \includegraphics[page=4]{figs/isolated}
    \end{tabular}
    \caption{Eliminating isolated vertices in $H-B(H)$.}
    \label{isolated_fig}
  \end{figure}
  \begin{compactenum}[(i)]
    \item $\deg^+_H(v)=2$ for each $v\in N_H(w)$ and some vertex $v\in N_H(w)$ is a cut vertex of $H$. In this case, $N^+_H(v):=\{w,w'\}$ and the vertices $w$ and $w'$ are in a different biconnected components $C$ and $C'$, respectively, of $H$.  We will show that \cref{kill_pair} applies. By \cref{inner_degree_1}, $C[N_C[v]]$ is isomorphic to $K_4$. Since $\deg_{H-B(H)}(w)=0$, the faces of $C[N_C[v]]$ have no vertices of $H$ in their interiors.   By \cref{inner_degree_1}, $C'[N_{C'}[v]]$ is also isomorphic to $K_4$.  Let $x$ and $y$ be the two neighbours of $v$ on the outer face of $C'$. Since $\deg^+_H(v)=2$, $\deg_{C'}^+(v)=1$, so the cycle $vxw'y$ has no vertices of $H$ in its interior. We claim that the triangle $yw'x$ also contains no vertices of $H$ in its interior. Indeed, if $yw'x$ contains more than one inner vertex then, by \cref{useful_little_guy}, $\deg^+_H(x)>2$ or $\deg^+_H(y)>2$, which is ruled out by the statement of the lemma.  If $yw'x$ contains exactly one inner vertex then $\deg_{H-B(H)}(w')=1$, which is also ruled out by the statement of the lemma.

    Let $H'$ be a dom-minimal graph that dom-respects $H-v$. Then $w$ and $w'$ are both in $B(H-v)$, so $\deg^+_{H-v}(w)=\deg^+_{H-v}(w')=0$. Therefore, neither $w$ nor $w'$ are in $H'$. Therefore $|H'|\le |H-\{v,w,w'\}|=|H|-3$.  Also $B(H-v)=B(H)\setminus\{v\}\cup\{w,w'\}$ but none of $v$, $w$, or $w'$ are in $H'$.  Therefore, by \cref{boundary_subset}, $B(H')\subseteq B(H-v)\setminus\{v,w,w'\}$ so $|B(H')|\le |B(H)|-1$.

    \item $\deg^+_H(v)=2$ for each $v\in N_H(w)$ and no vertex in $N_H(w)$ is a cut vertex.  We will show that \cref{kill_pair} or \cref{kill_double} applies.  Let $F$ be the face of $H[B(H)]$ that contains $w$ in its interior (so $V(F)=N_H(w)$).  If some edge $uv\in E(F)$ has $N^+_H(u)=N^+_H(v)$ then we take $H'$ be a dom-minimal graph that dom-respects $H-v$.  Since $\deg^+_{H-v}(u)=0$ and $\deg^+_{H-v}(w)=0$, neither $u$ nor $w$ is a vertex of $H'$. Therefore $|H'|\le |H-\{u,v,w\}|= |H|-3$.  By \cref{boundary_subset}, $B(H')\subseteq N_H(v)\cup B(H-\{u,v,w\})$, so $|B(H')|\le |B(H)|+2 - 3=|B(H)|-1$.  Therefore \cref{kill_pair} applies.

    Otherwise, consider any edge $zv\in E(F)$ and let $H'_0$ be a dom-minimal graph that dom-respects $H-v$.  If $z$ is not a vertex of $H'_0$ then $|H'_0|\le |H-\{z,v,w\}|\le |H|-3$ and $|B(H'_0)|\le |B(H)|+2-3=|B(H)|-1$, so \cref{kill_pair} applies (with $H':=H'_0$).  Otherwise $z$ is vertex of $H'_0$ and $\deg^+_{H'_0}(z)=1$.  By \cref{inner_degree_1}, $H'_0[N_{H'_0}[z]]$ is isomorphic to $K_4$.  Let $z,x,y,w'$ be the vertices of $N_{H'_0}[z]$ labelled so that $z$, $x$, and $y$ are in $B(H)$ and $w'\in I(H)$.  Since $\deg^+_{H'_0}(z)=1$, the cycle $z,x,w',y$ has no vertices of $H$ in its interior.

    \pat{This may not be needed:}We claim that the cycle $x,w',y$ also has no vertices of $H$ in its interior.  Indeed, if this cycle contains exactly one vertex of $H$ in its interior then, by \cref{useful_little_guy}, $\deg_{H-B(H)}(w')=1$, which is ruled out by the statement of the lemma.  If this cycle contains two or more vertices of $H$ in its interior then, by \cref{useful_little_guy}, $\deg^+_{H'_0}(x)\ge 3$ or $\deg^+_{H'_0}(y)\ge 3$, which is also ruled out by the statement of the lemma.

    If $\deg^+_{H'_0}(x)= 2$ then we take $u:=x$ and consider a dom-minimal graph $H'$ that dom-respects $H-\{u,v\}$.  By \cref{bad_vertex} none of $V(H')\subseteq V(H)\setminus\{v,w,u,w'\}$.


    \item $\deg^+_H(z)=1$ for some $z\in N_H(w)$.  We will show that one of \cref{kill_k4} or \cref{kill_pair} applies.  By \cref{inner_degree_1}, $H[N_H[z]]$ is isomorphic to $K_4$.  Let $x$ and $y$ be the two vertices, other than $z$ on the outer face of $H[N_H[z]]$.  If $x$ has $\deg^+_H(x)=2$ then we consider a dom-minimal graph $H'$ that dom-respects $H-x$.  Since $\deg^+_{H-x}(z)=0$ and $\deg^+_{H-x}(w)=0$, neither $z$ nor $w$ are in $H'$, by \cref{bad_vertex}.  Therefore, $|H'|\le |H-\{x,z,w\}|=|H|-3$.  By \cref{boundary_subset}, $B(H')\subseteq N^+_H(x)\cup B(H-\{x,z,w\})$, so $|B(H')|\le |B(H)|+2-3=|B(H)|-1$.  Therefore \cref{kill_pair} applies (with $v=x$).  The same argument applies if $\deg^+_H(y)=2$.  Otherwise, $N^+_H(z)=N^+_H(x)=N^+_H(y)=\{w\}$ and the component of $H$ that contains $w$ is isomorphic to $K_4$.  In this case, we take $v=z$ and let $H'$ be a dom-minimal graph that dom-respects $H-z$.  Then $V(H')\subseteq V(H-\{x,y,z,w\})$ and $B(H')\subseteq B(H)\setminus \{x,y,z\}$, so \cref{kill_k4} applies.


    In this case let $v$ be any vertex in $N_H(w)$ and let $H'$ be a dom-minimal graph that dom-respects $H-v$.  Then $w\in B(H-v)$, so $\deg^+_{H-v}(v')=0$ for all $v'\in N_H(w)$. Since $H'$ is dom-minimal, $N_H(w)\cap V(H')=\emptyset$, by \cref{bad_vertex}.  Similarly $w\not\in V(H')$, also by \cref{bad_vertex}. Therefore $|H'|\le |H-(\{w\}\cup N_H(w))| \le |H|-4$. Then $B(H-v)=B(H)\setminus\{v\}\cup\{w\}$.
    By \cref{boundary_subset}, $B(H')\subseteq B(H-v)\setminus N_H[w]$, so $|B(H')|\le |B(H)|-\deg_H(w) \le |B(H)|-3$. \qedhere
  \end{compactenum}
\end{proof}



\subsection{$2$-Critical Graphs}
\label{two_critical_section}

We now explain what the algorithm does when it finally reaches a state where  none of \cref{really_good_cor}, \cref{leaf_killer} or \cref{degree_zero_killer} can be used to make an incremental step.  The inapplicability of \cref{leaf_killer,degree_zero_killer,really_good_cor} leads to the following definition:
A generalized near-triangulation $H$ is \defin{$2$-critical} if
\begin{compactenum}[({2-C}1)]
  \item $\deg^+_H(v)\le 2$ for each $v\in B(H)$; \label[tc]{inner_degree_2}
  \item $H-B(H)$ is critical; \label[tc]{inner_critical} and
  \item $\deg_{H-B(H)}(w)\ge 2$ for all $w\in V(H-B(H))$. \label[tc]{inner_vertex_degre_2}
\end{compactenum}
(See \cref{two_critical_figure}.) We will work our way up to a proof of the following lemma, which allows our algorithm to handle $2$-critical graphs directly, in one step:

\begin{figure}
  \centering
  \includegraphics[page=1]{figs/two_critical}
  \caption{A $2$-critical generalized near-triangulation.}
  \label{two_critical_figure}
\end{figure}


\begin{lem}\label{two_critical_handler}
  Let $H$ be a $2$-critical generalized near-triangulation.  Then there exists $X\subseteq V(H)$ of size at most $(2|B(H-B(H))| + I(H-B(H)))/3$ that dominates $I(H)$ and such that each component of $H[X]$ contains at least one vertex in $B(H)$.
\end{lem}


\begin{lem}
  Let $H$ be a dom-minimal $2$-critical generalized near-triangulation.  Then $\deg^+_H(v)=2$ for all $v\in B(H)$.
\end{lem}

\begin{proof}
  Consider some $v\in B(H)$. By \cref{bad_vertex}, $\deg^+_H(v)\ge 1$. Assume for the sake of contradiction that $\deg_H(v)=1$. By \cref{inner_degree_1}, $H[N_H[v]]$ is isomorphic to $K_4$. Let $x$ and $y$ be the neighbours of $v$ on the outer face of $H$ and let $w$ be the inner neighbour of $v$. Since $\deg^+_H(v)=1$ then the cycle $vxwy$ has no vertices of $H$ in its interior.  Since $H$ is $2$-critical, $\deg_{H-B(H)}(w)\ge 2$, which implies that $w$ has at least two neighbours in the interior of the cycle $ywx$.  By \cref{useful_little_guy}, at least one of $x$ or $y$, say $x$, has at least two neighbours in the interior of $ywx$.  \Cref{useful_little_guy} implies that $\deg^+_H(x)\ge 3$, which contradicts the fact that $H$ is $2$-critical.
\end{proof}


\begin{lem}\label{two_critical_boundary_size}
  Let $H$ be a dom-minimal $2$-critical generalized near-triangulation.  Then $|B(H)|\ge |B(H-B(H))|$.
\end{lem}

\begin{proof}
  Let $\mathcal{C}$ be the set of components of $H-B(H)$.
  Let $C$ be a component in $\mathcal{C}$ and let $w_0,\ldots,w_k$ be the clockwise walk around the outer face of $C$, so that $w_0=w_k$.  Then, for each $i\in\{1,\ldots,k\}$, $H$ contains an inner face $w_{i-1}w_iv_i$ that is to the left of the edge $w_{i-1}w_i$ when traversed from $w_{i-1}$ to $w_i$ and $v_i\in B(H)$.  Since $H$ is $2$-critical and does not contain parallel edges, $v_i\neq v_j$ for any $i\neq j$. Let $N_2(C):=\{v_1,\ldots,v_k\}$.  Therefore $|N_2(C)|= k\ge |B(C)|$. Since $H$ is $2$-critical and $\deg^+_C(v)\ge 2$ for all $v\in N_2(C)$, $N_2(C)\cap N_2(C')=\emptyset$ for any distinct components $C,C'\in\mathcal{C}$.  Therefore $|B(H)|\ge \sum_{C\in\mathcal{C}} |N_2(C)|\ge \sum_{C\in\mathcal{C}} |B(C)|=|B(H-B(H))|$.
\end{proof}

For each integer $r\ge 3$, the \defin{$r$-wheel} $W_r$ is the near-triangulation whose outer face is bounded by a cycle $v_0,\ldots,v_{r-1}$ that contains a single vertex $x$ in its interior and that is adjacent to each of $v_0,\ldots,v_{r-1}$.  For even values of $r$, $W_r$ is called an \defin{even wheel}. Note that the following lemma, illustrated in \cref{biconnected_critical_colouring} is about critical graphs, not $2$-critical graphs.

\begin{figure}
  \centering
  \includegraphics[page=7,trim={0 55 0 10},clip]{figs/two_critical}
  \caption{\cref{biconnected_critical}: Partitioning the vertices of a biconnected critical graph into three dominating sets.}
  \label{biconnected_critical_colouring}
\end{figure}

\begin{lem}\label{biconnected_critical}
  Let $H$ be a biconnected critical generalized near-triangulation with at least $3$ vertices and not isomorphic to $W_k$ for any even integer $k$.  Then there exists a partition $\{X_0,X_1,X_2\}$ of $V(H)$ such that
  \begin{compactenum}[(i)]
    \item For each edge $vw$ of $H[B(H)]$, $v\in X_i$ and $w\in X_j$ for some $i\neq j$;\label[p]{proper}
    \item for each $i\in\{0,1,2\}$, $X_i$ dominates $H$. \label[p]{dominates_h}
  \end{compactenum}
\end{lem}

\begin{proof}
  If $H$ is isomorphic to $W_k$ for some odd integer $k\ge 3$, then we take $X_0:=\{v_0, x\}$, $X_1:=\{v_{2i-1}:i\in\{1,\ldots,\lfloor k/2\rfloor\}$, and $X_2:=\{v_{2i}:i\in\{1,\ldots,\lfloor k/2\rfloor\}$.  It is straightforward to verify that these sets satisfy \cref{proper,dominates_h}.  (The fact that $k$ is odd ensures that $v_0$ has a neighbour $v_1\in X_1$ and $v_{k-1}\in X_2$, which ensures \cref{dominates_h}---this is not true for even $k$.)   We now assume that $H$ is not isomorphic to $W_k$ for any integer $k$.  By \cref{critical_structure}, this implies that $H$ is outerplanar or that $H[B(H)]$ has at least two inner faces.

  We now proceed by induction on $|I(H)|$.  If $|I(H)|=0$ then $H$ is an edge-maximal outerplanar graph, and therefore has a proper $3$-colouring.  We take $X_0$, $X_1$, and $X_2$ to be the three colour classes in this colouring.  This choice clearly satisfies \cref{proper}. Since each vertex of $H$ is included in at least one triangle, each vertex of $H$ is dominated by each of $X_0$, $X_1$, and $X_2$, so this choice satisfies \cref{dominates_h}.

  \begin{figure}
    \centering
    \begin{tabular}{ccc}
      \includegraphics[page=1]{figs/biconnected} &
      \includegraphics[page=2]{figs/biconnected} &
      \includegraphics[page=3]{figs/biconnected} \\
      $H$ & $H'$ & $H$
    \end{tabular}
    \caption{The proof of \cref{biconnected_critical}}
    \label{contraction_proof}
  \end{figure}
  If $|I(H)|\ge 1$ then $H$ is not outerplanar.  See \cref{contraction_proof}.  Since $H$ is not isomorphic to $W_k$ for any integer $k$, $H[B(H)]$ contains at least two inner faces.  Let $x$ be an inner vertex of $H$ and let $f$ be the marked face of $H[B(H)]$ that contains $x$.
  % If $f$ has only three vertices, then we apply induction on $H':=H-x$ to obtain sets $X_0'$, $X_1'$, and $X_2'$.  We then set $X_0:=X_0'\cup\{x\}$, $X_1:=X_1'$ and $X_2:=X_2'$.  These sets satisfy \cref{proper} because every edge of $H[B(H)]$ is also an edge of $H'[B(H')]$.  These sets satisfy \cref{colourful} because $f$ is a clique of size $3$ in $H'$ so, by \cref{proper}, each $X_i$ contains at least one vertex of $f$. Now, \cref{colourful} also implies that each $X_i$ dominates $x$ and (by induction) dominates $V(H)\setminus\{x\}$, so each $X_i$ is a dominating set.
  Since $H[B(H)]$ has at least two inner faces and $H$ is biconnected, $f$ contains an edge $vw$ that is on the boundary of two faces of $H$.  Let $H'$ be the graph obtained by contracting the edge $vx$ into $v$. Then $H'[B(H')]=H[B(H)]$ and $H'$ is a biconnected critical generalized near-triangulation so we apply induction to obtain sets $X_0'$, $X_1'$ and $X_2'$.  Without loss of generality, we can assume that $v$ is in $X_1'$.  Then we set $X_0:=X_0'$, $X_1:=X_1'\cup\{x\}$ and $X_2:=X_2'$. Since $H'[B(H')]=H[B(H)]$ this clearly satisfies \cref{proper}.  Since $H'$ contains the edge $vw$ for each $w\in V(f)\setminus\{v\}$, \cref{proper} implies that the vertices of the path $f-v$ are alternately contained in $X_2$ and $X_0$.
  % In particular, since $f$ has at least three vertices these sets satisfy \cref{colourful}.

  All that remains is to show that $X_0$, $X_1$, and $X_2$ satisfy \cref{dominates_h}.  The inductive hypothesis already implies that each of these sets dominates $V(H)\setminus V(f)$.  Since $x$ is adjacent to every vertex of $f$, it is adjacent to at least one vertex of $X_0$ and at least one vertex of $X_2$.  Therefore, each of $X_0$, $X_1$, and $X_2$ dominates $x$.  For each vertex $w\in V(f)\setminus\{v\}$, $w$ is adjacent to $x\in X_1$, $w\in X_{i}$ for some $i\in\{0,2\}$ and $w$ is adjacent to a neighbour $w'\in X_{2-i}$ in $f$, so each of these sets dominates $w$.  Finally, since the vertex $v$ is incident to a chord of $H[B(H)]$, it is incident to a second face $f'\neq f$ of $H[B(H)]$.  Since $f$ is marked and $H$ is critical, $f'$ is not marked.  Therefore $f'$ is a triangle with one vertex in each of $X_0$, $X_1$, and $X_2$. Therefore each of these sets dominates $v$.
\end{proof}

The following lemma, illustrated in \cref{even_wheel}
 explains how we deal with even wheels not covered by \cref{biconnected_critical}:

\begin{lem}\label{wheelie}
  Let $H:=W_k$ for some even integer $k\ge 4$ and let $v$ be any vertex in $B(H)$.  Then there exists a partition $\{X_0,X_1,X_2\}$ of $V(H)$ such that
  \begin{compactenum}[(i)]
    \item For each edge $vw$ of $H[B(H)]$, $v\in X_i$ and $w\in X_j$ for some $i\neq j$;\label[p]{proper2}
    % \item For each inner face $f$ of $H[B(H)]$ and each $i\in\{0,1,2\}$, $V(f)\cap X_i\neq\emptyset$; and \label[p]{colourful}
    % that contains $x_f\in I(H)$ in its interior, exactly one vertex $v_f$ of $f$ is in the set $X_i$ that contains $x$ and $v_f$ is incident to a chord of $H[B(H)]$. \pat{Define chord.}
    \item $X_0$ dominates $V(H)\setminus\{v\}$ and $X_1$ and $X_2$ each dominate $H$. \label[p]{weak_dominates_h}
  \end{compactenum}
\end{lem}

\begin{figure}
  \centering
  \includegraphics{figs/even_wheel}
  \caption{\cref{wheelie}: Partitioning the vertices of an even wheel into sets $X_0$, $X_1$, and $X_2$.}
  \label{even_wheel}
\end{figure}
\begin{proof}
  Label the vertices of $W_k$ as $v_0,\ldots,v_{k-1}$ so that $v=v_0$.  Then the sets $X_1:=\{v_0, x\}$, $X_2:=\{v_{2i-1}:i\in\{1,\ldots,k/2\}\}$, and $X_0:=\{v_{2i}:i\in\{1,\ldots, k/2-1\}\}$ satisfy the requirements of the lemma.
\end{proof}


The following lemma, illustrated in \cref{critical_colouring}, drops the requirement that the critical graph be biconnected and applies even if some of the biconnected components of $H$ are even wheels.

\begin{figure}
  \centering
  \includegraphics[page=2]{figs/two_critical}
  \caption{\cref{critical}: Partitioning the vertices of a connected critical graph into dominating sets $X_0$, $X_1$, and $X_2$.}
  \label{critical_colouring}
\end{figure}

\begin{lem}\label{critical}
  Let $H$ be a connected critical generalized near-triangulation with at least $3$ vertices, no vertices of degree $1$ and not isomorphic to $W_k$ for any even integer $k$.  Then there exists a partition $\{X_0,X_1,X_2\}$ of $V(H)$ such that
  \begin{compactenum}[(i)]
    \item for each edge $vw$ of $H[B(H)]$, $v\in X_i$ and $w\in X_j$ for some $i\neq j$;\label[p]{proper_2}
    \item for each $i\in\{0,1,2\}$, $X_i$ dominates $H$,  \label[p]{dominates_h_minus_l}
  \end{compactenum}
\end{lem}

\begin{proof}
  The proof is by induction on $|H|$.  First, suppose that $|H|=3$. Since $H$ is connected and has no vertices of degree $1$, $H$ is a triangle $v_0v_1v_2$. We take $X_i:=\{v_i\}$ for each $i\in\{0,1,2\}$.  Clearly these sets satisfy the requirements of the lemma.

  If $H$ is biconnected then, since $H$ is not an even wheel, we can immediately apply \cref{biconnected_critical} and we are done.  Otherwise, $H$ contains a cut vertex $v$ that separates $H$ into components $C_1,\ldots,C_k$ and such that $H':=H[V(C_1)\cup\{v\}]$ is biconnected. Refer to \cref{critical_3_colouring}. Since $L=\emptyset$, $H'$ has at least three vertices. If $H'$ is isomorphic to $W_k$ for some even integer $k$ then we apply \cref{wheelie} to $H'$ and $v$ to obtain sets $X_0'$, $X_1'$, and $X_2'$. Otherwise, we apply \cref{biconnected_critical} to $H'$ to obtain sets $X_0'$, $X_1'$, and $X_2'$.  In either case we may assume, without loss of generality that $v\in X_1'$, that $X_1'$ and $X_2'$ each dominate $H'$ and that $X_0'$ dominates $V(H')\setminus\{v\}$.

  \begin{figure}[htbp]
    \centering
    \begin{tabular}{cc}
      \includegraphics[page=1,trim={30 0 30 0},clip]{figs/critical_3_colouring} &
      \includegraphics[page=2]{figs/critical_3_colouring}
    \end{tabular}
    \caption{Two cases in the proof of \cref{critical}.}
    \label{critical_3_colouring}
  \end{figure}

  Let $H'':=H-V(C_1)$. First, suppose that $\deg_{H''}(v)>1$.
  If $H''$ is isomorphic to $W_k$ for some even integer $k$ then we apply \cref{wheelie} to $H''$ and $v$ to obtain sets $X_0''$, $X_1''$, $X_2''$. Otherwise, we apply the inductive hypothesis to $H''$ to obtain sets $X_0''$, $X_1''$, $X_2''$ that each dominate $H''$. In either case we may assume, without loss of generality (by renaming) that $v\in X_1''$, that $X_1''$ and $X_2''$ each dominate $H'$ and that $X_0''$ dominates $V(H'')\setminus\{v\}$.  Then the sets $X_0:=X_0'\cup X_2''$, $X_1:=X_1'\cup X_1''$ and $X_2:=X_2'\cup X_0''$ satisfy the requirements of the lemma.  (The only concern is whether each set dominates $v$, but this is guaranteed by the fact that $v\in X_1$, and that $X_2'\subseteq X_2$ and $X_2''\subseteq X_0$ each dominate $v$.)

  Finally, if $\deg_{H''}(v)=1$ then we consider the maximal path $v,v_1,v_2,\ldots,v_{r-1},v_r$ such that $\deg_{H''}(v_i)=2$ for each $i\in\{1,\ldots,r-1\}$.  Let $H''':=H''-\{v,v_1,\ldots,v_{r-1}\}$ and we treat $H'''$ exactly as we treated $H''$ in the previous paragraph to obtain sets $X_0'''$, $X_1'''$ and $X_2'''$.  Without loss of generality, we assume that $v_r\in X_{(r-1)\bmod 3}$, that $X_{(r-1)\bmod 3}$ and $X_{(r\bmod 3)}$ each dominate $H'''$ and that $X_{(r-2)\bmod 3}$ dominates  $V(H''')\setminus\{v_r\}$. Let $X_0'':=X_0'''\cup\{v_i:i\equiv 2\pmod 3\}$, $X_1'':=X_1'''\cup\{v\}\cup\{v_i:i\equiv 0\pmod 3\}$, and  $X_2'':=X_2'''\cup\{v_i:i\equiv 1\pmod 3\}$.  Then $v\in X_1''$, $X_1''$ and $X_2''$ each dominate $H''$, and $X_0''$ dominates $V(H'')\setminus\{v\}$.  We can now define the sets $X_0$, $X_1$, and $X_2$ exactly as we did in the previous paragraph.
\end{proof}


At last, the following lemma, illustrated in \cref{two_critical_colouring}, shows how we combine everything to find three sets whose total size is at most $2|B(H-B(H))| + |I(H-B(H))|$.
% \saeed{this is correct. but do you mean $H-B(H)$ instead of $H$?}\pat{Yes, good catch}

\begin{figure}
  \centering
  \includegraphics[page=4]{figs/two_critical}
  \caption{\cref{three_sets_coverage}: Finding three sets $X_0$, $X_1$, and $X_2$ that dominate $I(H)$ in a $2$-critical graph $H$.}
  \label{two_critical_colouring}
\end{figure}


\begin{lem}\label{three_sets_coverage}
  Let $H$ be a $2$-critical generalized near-triangulation.  Then there exists $X_0,X_1,X_2\subseteq V(H)$ such that
  \begin{compactenum}[(i)]
    \item $|X_0|+|X_1|+|X_2| \le 2|B(H-B(H))|+|I(H-B(H))|$; \label[p]{total_size}
    \item for each $i\in\{0,1,2\}$, $X_i$ dominates $I(H)$ in $H$; and \label[p]{dominates_i}
    \item for each $i\in\{0,1,2\}$, each component of $H[X_i]$ contains at least one vertex in $B(H)$. \label[p]{connectivity}
  \end{compactenum}
\end{lem}

\begin{proof}
  % Refer to \cref{covers}.
  Let $\mathcal{C}$ be the set of components of $H-B(H)$
  and let $\mathcal{C}_{\boxtimes}$ be the set of components in $\mathcal{C}$  that are even wheels.

  For each component $C$ in $\mathcal{C}_{\boxtimes}$ we choose the vertex $x$ that dominates $C$, some vertex $w$ in $B(C)$ and some vertex $v\in B(H)$ adjacent to $w$.  We add $\{v,w,x\}$ to each of $X_0$, $X_1$, and $X_2$. The vertex $x$ ensures that each $X_i$ dominates $C$ and the vertices $v$ and $w$ ensure that the component of $H[X_i]$ that contains $x$ contains at least one vertex in $B(H)$.  Doing this for every component in $\mathcal{C}_{\boxtimes}$ contributes a total of $9|\mathcal{C}_{\boxplus}|$ vertices to $X_0$, $X_1$, and $X_2$. On the other hand, $|B(C)|\ge 4$ and $|I(C)|\ge 1$ for each $C\in \mathcal{C}_{\boxplus}$, so $\sum_{C\in \mathcal{C}_{\boxplus}} (2|B(C)| + I(C))\ge (2\cdot 4+1)|\mathcal{C}_{\boxplus}| = 9|\mathcal{C}_{\boxplus}|$.

  For each component $C$ in $\mathcal{C}\setminus\mathcal{C}_{\boxplus}$, we apply \cref{critical} to obtain sets $X_0'$, $X_1'$, $X_2'$. For each $i\in\{0,1,2\}$ and each $w\in X_i'\cap B(H-B(H))$ we choose a vertex $v\in B(H)$ adjacent to $w$ and add both $v$ and $w$ to $X_i$. \Cref{critical} ensures that each $X_i$ dominates $C$ and the vertex $v$ ensures that the component of $H[X_i]$ that contains $w$ contains at least one vertex of $B(H)$.
  Doing this for each component $\mathcal{C}\setminus\mathcal{C}_\boxplus$ contributes a total of at most  $\sum_{C\in\mathcal{C}\setminus\mathcal{C}_{\boxplus}}(|C|+|B(C)|)=\sum_{C\in\mathcal{C}\setminus \mathcal{C}_{\boxplus})}(2|B(C)|+|I(C)|)$ to $X_0$, $X_1$, and $X_2$.

  The resulting sets $X_1$, $X_2$, and $X_3$ each dominate $\bigcup_{C\in\mathcal{C}} V(C)=I(H)$ and have total size at most $\sum_{C\in\mathcal{C}} (2|B(C)|+|I(C)|) = 2|B(H-B(H))| + |I(H-B(H))|$.
\end{proof}

\begin{proof}[Proof of \cref{two_critical_handler}]
  Take $X$ to be the smallest of the three sets $X_0$, $X_1$, and $X_2$ guaranteed by \cref{three_sets_coverage}.
\end{proof}


\subsection{The Algorithm}
% \hussein{bg5 is not compatible with lemma 10}\pat{Fixed}


All of this has been leading up to a variant  $\textsc{SimpleGreedy}(G)$ that we call $\textsc{BetterGreedy}(G)$.  Suppose we have already chosen $\Delta_0,\ldots,\Delta_{i-1}$ for some $i\ge 0$ and we now want to choose $\Delta_i$.  Let $X_i:=\bigcup_{j=0}^{i-1}\Delta_j$, let $G_i$ be a dom-minimal graph that dom-respects $G-X_i$, and let $v_i$ be a vertex in $B(G_i)$ that maximizes $\deg^+_{G_i}(v_i)$.  During iteration $i\ge 0$, there are now more cases to consider:
\begin{compactenum}[{[}bg1{]}]
    \item If $\deg^+_{G_i}(v_i)\ge 3$ then we set $\Delta_i\gets\{v_i\}$.
    \label[bg]{bg_high_degree}
    \item Otherwise, if $G_i-B(H_i)$ contains a vertex of degree $1$ we set $\Delta_i:=\{v_i\}$ where $v_i$ is the vertex $v$ guaranteed by \cref{leaf_killer}.
    \item Otherwise, if $G_i-B(H_i)$ contains a vertex of degree $0$ we set $\Delta_i:=\{v_i\}$ where $v_i$ is the vertex $v$ guaranteed by \cref{degree_zero_killer}.
    \item Otherwise, if there exists distinct $u,v\in B(G_i)$ and $w\in B(G_i-B(G_i))$ such that $\deg^+_{G_i}(v)=2$, $\deg^+_{G_i-v}(w)\ge 3$, and $N^+_{G_i}(u)\subseteq N_{G_i}(w)$ then set $\Delta_i:=\{v,w\}$.
    \label[bg]{bg_two_three}
    \item Otherwise, $G_i$ is $2$-critical and $i+1=r$.  By \cref{two_critical_handler}, there exists $\Delta_{r-1}\subseteq V(G_i)$ of size at most $2|B(G_i-B(G_i))|/3 + |I(G_i-B(G_i))|/3$ that dominates $I(G_i)$.
    \label[bg]{bg_two_critical}
\end{compactenum}

\begin{thm}\label{better_greedy}
  When applied to an $n$-vertex triangulation $G$,  $\textsc{BetterGreedy}(G)$ produces a connected dominating set $X_r$ of size at most $(10n-18)/21$.
\end{thm}

\begin{proof}
  By \cref{leaf_killer,degree_zero_killer,really_good} during each of the first $r-1$ steps, one of the following occurs:
  \begin{compactitem}
    \item[$x_t$:] For some $t\ge 3$, we can add a single vertex $v_i$ that increases the size of the dominated set $D_{i+1}:=N[X_{i+1}]$ by $t$ and increases the size of the boundary set $B_{i+1}:=N_G(I(G-D_{i+1}))$ by at most $t-1$.
    \item[$a$:] We can add a vertex $v_i$ that increases the size of the dominated set $D_{i+1}$ by $2$ and \emph{decreases} the size of the boundary set $B_{i+1}$ by at least $1$.
    \item[$b$:] We can add a vertex $v_i$ that increases the size of the dominated set $D_{i+1}$ by $1$ and \emph{decreases} the size of the boundary set $B_{i+1}$ by at least $3$.
    \item[$c$:] We can add a pair of vertices $\{v_i,w_i\}$ that increase the size of the dominated set $D_{i+1}$ by at least $5$ and increases the size of the boundary set $B_{i+1}$ by at most $2$.
    \item[$\bullet$:] We can directly complete the connected dominating set $X_r=X_{i+1}$ by adding a set $\Delta_{r-1}=\Delta_i$ of at most $(2|B(G_{i}-B(G_i))|+|I(G_i-B(G_i))|)/3$ additional vertices where, as before $G_i:=G[B_i\cup (V(G)\setminus D_i)]$ and $r=i+1$.
  \end{compactitem}

  Refer to \cref{dbrs}. Let $a$, $b$, $c$, and $\langle x_t\rangle_{t\ge 3}$ denote the number of times each of these cases occurs in the first $r-1$ steps, and let $D:=D_{r-1}$, $B:=B_{r-1}$, and $X:=X_{r-1}$.  Then,
  \begin{align}
    |D| & \ge 3 + \sum_{t\ge 3}tx_t + 2a + b + 5c \label{dd_size} \\
    |B| & \le 3 + \sum_{t\ge 3}(t-1)x_t - a - 3b + 2c \enspace \label{bb_size} \\
    |X| & \le \sum_{t\ge 3}x_t + a + b + 2c \label{x_size} \enspace .
  \end{align}

  \begin{figure}
      \centering
      \includegraphics{figs/dbrs}
          \caption{The sets $X$, $D$, $B$, $R$, and $S$}
      \label{dbrs}
  \end{figure}

  Let $R:=B(G_{r-1}-B(G_{r-1}))$ and $S:=I(G_{r-1}-B(G_{r-1}))$.  Since $\{D,R,S\}$ is a partition of $V(G)$,
  \begin{equation}
    |D|+|R|+|S|= n \enspace . \label{ttotal_size}
  \end{equation}
  By \cref{two_critical_boundary_size}, $|B_{r-1}|\ge |B(G_i-B(G_i))|$, i.e., $|B|\ge |R|$.  Putting everything together we get the constraints:
  \begin{align}
        3 + \sum_{t\ge 3}tx_t + 2a + b + 5c + |R| + |S| \le n&
          & \text{(by \cref{dd_size} and \cref{ttotal_size})}  \label{a} \\
        3 + \sum_{t\ge 3}(t-1)x_t - a - 3b + 2c \ge |R|
          & & \text{(by \cref{bb_size} and since $|B|\ge |R|$)} \label{b} \\
  \end{align}
  with all values non-negative.  The size of the final connected dominating set $X_r$ is then at most
  \begin{equation}
    |X_r| = |X| + |\Delta_{r-1}| \le \sum_{t\ge 3}x_t + a + b + 2c + 2|R|/3 + |S|/3  \enspace . \label{oobjective}
  \end{equation}

  \begin{clm}\label{simplify}
    If $(a,b,c,|R|,|S|,x_3,x_4,\ldots)$ are non-negative and satisfy \cref{a,b}, then setting $x_3\gets x_3+\sum_{t\ge 4}(t-1)x_t/3$ and $x_t\gets 0$ for all $t\ge 4$ also satisfy \cref{a,b} and do not decrease \cref{oobjective}.
  \end{clm}

  \begin{clmproof}
    Suppose $x_t>0$ for some integer $t\ge 4$, otherwise there is nothing to prove.  Let $i:=\min\{t\ge 4: x_t>0\}$ and set $x_3\gets x_3+(t-1)x_t/3$ and $x_t\gets 0$. This change causes the left-hand-side of \cref{a} to decrease by $x_t$. This change does not affect the left-hand-side of \cref{b}.  This change increases the value of \cref{oobjective} by $(t-1)x_t/3-x_t\ge 0$.
  \end{clmproof}

  By \cref{simplify}, maximizing \cref{oobjective} subject to the constraints given by \cref{a,b} is a linear program in six variables $(x_3,a,b,c,|R|,|S|)$ which can be done easily.  The maximum is achieved when $x_3=a=b=c=|S|=0$, $c=(n-6)/7$ and $|R|=(2n+9)/7$, at which point \cref{oobjective} evaluates to $(10n-18)/21$.
\end{proof}

%%%%%%% The graph on surfaces section %%%%%

\section{Connected Dominating Set for Surface Triangulations}
\label{bounded_genus}
In this section, we establish \cref{genus_result}, the extension of \cref{main_result2} to surface triangulations of genus $g = o(n)$.
Briefly, a \defin{surface} or \defin{2-manifold} $\mathcal{S}$ is a compact connected Hausdorff topological space such that every point in $\mathcal{S}$ is locally homeomorphic to the plane i.e. it has a neighbourhood homeomorphic to $\mathbb{R}^2$. Every such surface can be created from the sphere $\mathbb{S}^2$, by adding handles and cross-caps. The \defin{Euler genus} of a surface with $h$ handles and $c$ cross-caps is $2h+c$.
% define surfaces related info
%Briefly, a \defin{surface or 2-manifold} $\mathcal{S}$ is a compact connected Hausdorff topological space such that every point in $\mathcal{S}$ is locally homeomorphic to the plane i.e. it has a neighbourhood homeomorphic to $\mathbb{R}^2$. The genus $g$ of a surface $\mathcal{S}$ is the number of handles or cross-caps added to the sphere $\mathbb{S}^2 = \{x \in \mathbb{R}^3 : ||x||= 1\}$ to form a surface homeomorphic to $\mathcal{S}$. In the former case, we call the surface $\mathcal{S}_g$, and the later case we call it $\mathcal{N}_g$. A surface is \defin{orientable} if it is homeomorphic to $\mathcal{S}_g$ for some non-negative $g$, we say it is \defin{non-orientable} if it is homeomorphic to a surface $\mathcal{N}_g$ for some positive $g$. The Euler-genus $\Bar{g}$ of a surface $\mathcal{S}$ is equal to $2g$ if $\mathcal{S}$ is orientable and $g$ if $\mathcal{S}$ is non-orientable. For a detailed account on graph and surfaces, we refer the reader to \cite{mohar2001graphs, lee2010introduction}.\pat{Consider using Euler genus:  The Euler genus of a surface with $h$ handles and $c$ crosscaps is $2h + c$.}

% \begin{patnote}
%    Briefly, a \defin{surface or 2-manifold} $\mathcal{S}$ is a compact connected Hausdorff topological space such that every point in $\mathcal{S}$ is locally homeomorphic to the plane i.e. it has a neighbourhood homeomorphic to $\mathbb{R}^2$. Every such surface can created from the sphere $\mathbb{S}^2$, by adding handles and cross-caps. The \defin{Euler genus} of a surface with $h$ handles and $c$ cross-caps is $2h+c$.
    % The genus $g$ of a surface $\mathcal{S}$ is the number of handles or cross-caps added to the sphere $\mathbb{S}^2 = \{x \in \mathbb{R}^3 : ||x||= 1\}$ to form a surface homeomorphic to $\mathcal{S}$. In the former case, we call the surface $\mathcal{S}_g$, and the later case we call it $\mathcal{N}_g$. A surface is \defin{orientable} if it is homeomorphic to $\mathcal{S}_g$ for some non-negative $g$, we say it is \defin{non-orientable} if it is homeomorphic to a surface $\mathcal{N}_g$ for some positive $g$. The Euler-genus $\Bar{g}$ of a surface $\mathcal{S}$ is equal to $2g$ if $\mathcal{S}$ is orientable and $g$ if $\mathcal{S}$ is non-orientable. For a detailed account on graph and surfaces, we refer the reader to \cite{mohar2001graphs, lee2010introduction}.\pat{Consider using Euler genus:  The Euler genus of a surface with $h$ handles and $c$ crosscaps is $2h + c$.} \hussein{add the definition for sigma too}
% \end{patnote}
% \hussein{I think, we should omit the discussion about oriantability as now the new theorem works on any surface in contrast to the previous one!!}

%%% define embedding related information
We follow the definitions given in \citet[Appendix~B]{diestel:graph}. An \defin{arc}, a \defin{circle}, and a \defin{disc} in a surface $\mathcal{S}$, is a subsets of $\mathcal{S}$ that is homeomorphic to $[0, 1]$, to a unit circle $\mathbb{S}^1 = \{x \in \mathbb{R}^1: ||x|| = 1\}$, or to a unit disc $\mathbb{B}^2 = \{x \in \mathbb{R}^2 : ||x|| < 1\}$, respectively.
% A closed disk is a disk that contains its boundary.
The set of all arcs in $\mathcal{S}$ is denoted by $A_\mathcal{S}$.
% \hussein{each arc in S is a subset of S; there is a kind of redandency.} \pat{Sure, but a point in $\mathcal{S}$ is a different kind of object than an arc in $\mathcal{S}$. Vertices map to point of $S$ and edges map to arcs.}
An \defin{embedding} of a graph $G$ in a surface $\mathcal{S}$ is a map \defin{$\sigma: V(G)\cup E(G) \longrightarrow \mathcal{S}\cup A_\mathcal{S}$} that sends vertices of $G$ to distinct points in $\mathcal{S}$ and sends an edge $xy$ to
% an $\sigma(x)-\sigma(y)$ arc in $\mathcal{S}$\pat{Suggested rewrite:
an arc $\sigma(xy)$ in $\mathcal{S}$ with endpoints $\sigma(x)$ and $\sigma(y)$ in such a way that interior of $\sigma(xy)$ is disjoint from $\{\sigma(v):v\in V(G)\}$ and from the interior of $\sigma(vw)$ for every $vw\in E(G)\setminus\{xy\}$.  For a subgraph $Z$ of $G$, we call $\sigma(Z):=\{\sigma(v):v\in V(X)\}\cup \bigcup_{vw\in E(X)}\sigma(vw)$ the \defin{embedded subgraph} $Z$.
% where no inner points of this arc intersects another arc or an image of a vertex. We let $\sigma(G):=\{\sigma(v):v\in V(G)\}\cup \bigcup_{e\in E(G)}\sigma(e)$.
A graph $G$ equipped with an embedding $\sigma$ on a surface $\mathcal{S}$ is called an \defin{$\mathcal{S}$-embedded} graph. A \defin{face} of $G$ in $\mathcal{S}$ is a component of $\mathcal{S} \setminus \sigma(G)$.
% In this context, we call $\sigma(vw)$ the \defin{embedded edge} $vw$.


The \defin{surface-map} $\Sigma$ of an $\mathcal{S}$-embedded graph $G$ is a tuple $(V, E, F)$ where $V$ is the set of vertices, $E$ is the set of edges, and $F$ is the set of faces in the embedding $\sigma$ of $G$. We call a surface-map a \defin{surface triangulation} if every face in $F$ is a disc in $\mathcal{S}$ whose boundary is an embedded $3$-cycle of $G$.  The \defin{Euler genus} of a surface triangulation $\Sigma$ is the Euler genus of the surface $\mathcal{S}$.
% \hussein{I think this is better because if our embedding of $G$ has every face is a disk, then it works, I guess.} \pat{Good, let's just leave it defined for surface triangulations then.}
%\hussein{ok, we can double check here \url{https://jeffe.cs.illinois.edu/teaching/comptop/2023/notes/19-surface-maps.pdf}} \pat{I don't think there's anything to check. Euler's formula tells us exactly the number of edges in a surface triangulation, as a function of $n$ and $g$.  If there were a smaller genus surface on which we could embed $G$ then we would get a contradiction.}
% \hussein{I think we should talk about the surface with minimum genus, because $\Sigma$ can be planar but embedded on higher surface; I think saying $\Sigma$ is maximal triangulation could work, but not sure}. \pat{I think it's the case that the graph of a surface-triangulation has the same genus as the surface (because every triangle is a disk in the surface).  Let me check.  Actually, it doesn't matter because the theorem refers to a genus-$g$ surface triangulation.} \hussein{here $\Sigma$ is ambiguous it can be a planar graph for example...} \pat{But if you embed a planar triangulation on a higher genus surface, then one of its faces will not be a disc, so you don't get a surface triangulation.}
% \hussein{yes I agree, so when it is maximal triangulation, I guess we are fine but our letter $\Sigma$ does not refer to a triangulation but to a surface-map, the first sentence in the paragraph...!!} \pat{That's ok. The surface map is defined with respect to an underlying surface $\mathcal{S}$.  It's actually instructions for making $\mathcal{S}$ by gluing together faces, so it's ok to call the genus of $\Sigma$ the genus of $\mathcal{S}$.} \hussein{that is fine, I am thinking of it as data-structure for the embedded graph rather than a simplical complex...}

% \pat{This isn't needed if we use Euler genus: The \defin{orientable genus} and the \defin{non-orientable genus} of a graph $G$ is the minimum $g$ and the minimum $g'$ such that $G$ has an embedding into surface $\mathcal{S}_g$, or $\mathcal{N}_{g'}$ respectively.  The genus of the surface-map is the genus of the underlying graph. Similarly, we define the Euler-genus of a graph and a surface-map.}
% \pat{Using Euler genus would simplify this to a single sentence.}


% A \defin{cut graph} of a map $\Sigma$ on a surace $S$ is any subgraph $X$ of $\Sigma$ such that $S \setminus X$ an open topological disk. ($\setminus$ means to remove the points from from $S$, not deleting the edges from the map).

As in \cite{EricksonNotes}, \defin{slicing} a surface map $\Sigma$ along a subgraph $Z \subseteq G$ with at least one edge produces a new map $\Sigma \bbslash Z$ which contains $\deg_{Z}(v)$ copies of every vertex $v$ of $Z$, two copies of every edge of $Z$, and at least one new face in addition to the faces of $\Sigma$.  (See \cref{slicing}.)
% \pat{I find this confusing, because a slicing produces a new graph and a new embedding of this graph, but a surface-map just refers to a set of vertices, edges, and faces.  Also, $\Sigma\bbslash Z$ contains every vertex of $G$ with $\deg_X(v)=0$.}
The faces of $\Sigma \bbslash Z$ that are not faces of $\Sigma$ are called \defin{holes} that are missing from the surface. A \defin{planarizing subgraph} of $\Sigma$ is any subgraph $Z \subseteq G$ such that the surface-map obtained after slicing along $Z$ has genus $0$ with one or more boundary cycles \cite{EricksonNotes}.  A key property of the slicing operation is the following:  If vertices $v'$ and $w'$ are on the boundary of the same hole in $\Sigma\bbslash Z$, then the corresponding vertices $v$ and $w$ of $\Sigma$ are in the same component of $Z$. We use of the following theorem of \citet{10.5555/644108.644208}.


\begin{figure}
    \begin{center}
        \begin{tabular}{cc}
            \includegraphics[width=.45\textwidth,page=2]{figs/slicing-orig} &
            \includegraphics[width=.45\textwidth,page=4]{figs/slicing-orig} \\
            $\Sigma$ & $\Sigma\bbslash Z$
        \end{tabular}
    \end{center}
    \caption{Slicing a surface map $\Sigma$ along a subgraph $Z$ (in red). Holes in $\Sigma\bbslash Z$ are shown in black and hole boundaries are red.}
    \label{slicing}
\end{figure}

%is embedded on a closed disk with holes, consequently in the plane.
% \pat{I don't think this is quite true.  If $Z$ contains  contractible cycles, then the surface underlying $\Sigma\bbslash Z$ is disconnected, but a closed disc with holes is connected.  The proof of \cref{Eppstein} does not make any effort to avoid contractible cycles in $Z$.  I think $X$ is a planarizing subgraph of $G$ if each component of $\mathcal{S}\setminus \sigma(X)$ is homeomorphic to a disc, possibly with holes.} \hussein{I fix it, there is something called cut graph that leaves a closed disk, but planarriser is not a cut graph but cut graph is a planariser.}
% \hussein{Should I cite the lecture notes for this paragraph?} \pat{Yes}
% A cycle is \defin{non-contractible} if it is not homotopic to a point (or nullhomotopic) i.e it cannot continuously deformed to a point in $\Sigma$.
%\pat{The preceding paragraph talks about the new surface $\Sigma'$ that we get by cutting along $C$, but doesn't explain the new graph $G'$ that we get (which now has two copies of each vertex and edge of $C$).  Maybe you want to separate the description of cutting a surface $\Sigma$ along a closed curve $C$ from the description of cutting a $\Sigma$-embedded graph $G$ along a curve $C$ that contains some vertices and edges of $G$ (but doesn't cross any edge of $G$).  This might make it easier to explain what happens when you cut $G$ along more than one cycle.}
% Then, we define the topology on the modified set as follow. Let $N$ be any strip neighbourhood of $C$ in $S$, and put $X' := \{x' | x\in C\}$ and $X'' := \{x'' | x \in C\}$. If $N$ is a cylinder, then $N \setminus C$ has two components $N'$ and $N''$, and choose the neighbourhoods of the new points $x'$ and $x''$ in $S'$ so that $X'$ and $X''$ becomes a boundary circle of $N'$ and $N''$ in $S'$, respectively, and $N' \cup X'$ and $N'' \cup X''$ become disjoint cylinders in $S'$. Finally, we turn $S'$ into a surface by capping its holes by taking two disjoint disk and identifying their boundary circles with $X'$, and $X''$ respectively.
% The \pat{orientable?} \defin{genus} of a graph $G$ is the minimum positive integer $g$ such that $G$ can be embedded on a surface of genus $g$.
%%% TODO:
% 1. define the embedding correctly. ok
% 2. define the genus of a graph . ok
% 3. define a non-seperating cycle. ok
% 4. define a slicing ok.
% 5. maybe I will add a figure.
\begin{thm}[\citet{10.5555/644108.644208}]\label{Eppstein}
Every surface triangulation $\Sigma$ with $n$ vertices and Euler-genus $g < n$ has a planarizing subgraph $Z$ with $O(\sqrt{gn})$ vertices and edges, which can be computed in $O(n)$ time.
\end{thm}

We can now prove the main result of this section, which readily establishes \cref{genus_result}.

\begin{lem}\label{generic_surface_thm}
    Let $f:\mathbb{N} \longrightarrow \mathbb{N}$ be a non-decreasing function such that every $n$-vertex (planar) triangulation has a connected dominating set of size at most $f(n)$. Then every $n$-vertex Euler genus-$g$ surface triangulation $\Sigma$ has a connected dominating set of size at most $f(n + O(\sqrt{gn})) + O(\sqrt{gn})$.
\end{lem}

\begin{proof}
Let $\Sigma:=(V,E,F)$ and let $G$ be the graph with vertex set $V$ and edge set $E$.
By applying Theorem \ref{Eppstein}, we obtain a planarizing graph
$Z$ of $\Sigma$. Treat $\Sigma \bbslash Z$ as a plane graph and add a set $E'$ of edges to obtain a planar triangulation $G'$ that has $n+O(\sqrt{gn})$ vertices. By assumption, $G'$ has a connected dominating set $X'$ whose size is at most $f(n + O(\sqrt{gn}))$.

The connected dominating set $X'$ contains vertices of $G$ and vertices of $\Sigma\bbslash Z$ that do not appear in $G$.  (These latter vertices are copies of vertices of $Z$.) To obtain a connected dominating set $X$ for $G$, we set $X:=(X'\cap V(G))\cup V(Z)$.   We now show that $X$ satisfies the requirements of the lemma.
\begin{itemize}
    \item \emph{$X$ is a dominating set:} Since $X'$ is a dominating set of $G'$, each vertex $w\in V(G)\setminus X$ is adjacent, in $G'$, to some vertex $v'\in X'$.  If $v'\in V(G)$ then $v'\in X$.  If $v'\in V(G')\setminus V(G)$, then $v'$ is a copy of some vertex $v$ of $Z$, in which case $v\in X$. In either case $w$ has neighbour, in $G$, that is contained in $X$.  Thus, $X$ is a dominating set of $G$.

    \item \emph{$G[X]$ is connected:} Let $v$ and $w$ be any two vertices in $X$. Then each of $v$ and $w$ has at at least one corresponding vertex $v'$ and $w'$, respectively, in $G'$.   Since $X'$ is a connected dominating set of $G'$, there exists a path $P':=v',z_0,\ldots,z_r,w'$ in $G'$ such that $z_0,\ldots,z_r$ is path in $G'[X']$.  If $P'$ does not contain any edge of $E'$ then $P'$ has a corresponding path in $G[X]$, so $v$ and $w$ are in the same component of $G[X]$.  If some edge $x'y'$ of $P'$ is in $E'$ then $x'$ and $y'$ are on the boundary of the same hole in $\Sigma\bbslash Z$.  Then $x'$ and $y'$ are copies of two vertices $x$ and $y$ of $G$ that are contained in the same component of $Z$.  In this case, we can replace the edge $x'y'$ with a path, in $Z$, from $x$ to $y$.  Doing this for each edge of $P'$ that is in $E'$ shows that there is a walk in $G[X]$ from $v$ to $w$, for each pair $v,w\in X$. Therefore $G[X]$ is connected.

    \item \emph{$X$ has size $f(n + O(\sqrt{gn}))+O(\sqrt{gn})$:} The size of $X$ is at most $|X'| + |V(Z)|\le f(n + O(\sqrt{gn})) + O(\sqrt{gn})$.
\end{itemize}
Therefore, $X$ is a connected dominating set of $G$ of size at most $f(n+O(\sqrt{gn}))+ O(\sqrt{gn})$.
\end{proof}

% \begin{thm} \label{surf_result}
% For every $n\ge 3$, every surface triangulation $\Sigma$ with $n$ vertices and genus $g = o(n)$ has a connected dominating set $X$ of size at most $10n/21 + O(\sqrt{gn})$.
% \end{thm}
\begin{proof}[Proof of \cref{genus_result}]
    By \cref{main_result2}, we can apply \cref{generic_surface_thm} with $f(n)= 10n/21$. We obtain a connected dominating set of size at most $10 (n + O(\sqrt{gn}))/21 + O(\sqrt{gn}) = 10n/21 + O(\sqrt{gn})$.
    % The equivalent statement about spanning trees follows from \cref{main_theorem_cor}.
    The linear-time algorithm follows from the linear-time algorithms for \cref{main_result2} and \cref{Eppstein}.
\end{proof}

% \begin{proof}
% Let $\Sigma:=(V,E,F)$ and let $G$ be the graph with vertex set $V$ and edge set $E$.
% By applying Theorem \ref{Eppstein}, we obtain a planarizing graph
% % \pat{minimal planarizing graph is not defined or discussed. Maybe just say, above, that a minimal planarizing graph has no degree-$1$ vertices.}\hussein{I mentioned this in the last paragraph, I will see how I change it ...}
% $Z$ of $\Sigma$.  Treat $\Sigma \bbslash Z$ as a plane graph and add a set $E'$ of edges to obtain a planar triangulation $G'$ that has $n:=n+O(\sqrt{gn})$ vertices.
% % obtained from performing the slicing along $Z$, we obtain a planar triangulation.
% % Denote by $E'$ the edges used to triangulate the holes of $\Sigma \bbslash Z$.
% Apply \cref{main_result2} to $G'$, to obtain a connected dominating set $X'$ of $G'$ whose size is at most $10n'/21+O(1)=10n/21+O(\sqrt{gn})$.

% The connected dominating set $X'$ contains vertices of $G$ and vertices of $\Sigma\bbslash Z$ that do not appear in $G$.  (These latter vertices are copies of vertices of $Z$.) To obtain a connected dominating set $X$ for $G$, we set $X:=(X'\cap V(G))\cup V(Z)$.   We claim that $X$ satisfies the requirements of the lemma.
% \begin{itemize}
%     \item \emph{$X$ is a dominating set:} Since $X'$ is a dominating set of $G'$, each vertex $w\in V(G)\setminus X$ is adjacent, in $G'$, to some vertex $v'\in X'$.  If $v'\in V(G)$ then $v'\in X$.  If $v'\in V(G')\setminus V(G)$, then $v'$ is a copy of some vertex $v$ of $Z$, in which case $v\in X$. In either case $w$ has neighbour, in $G$, that is contained in $X$.  Thus, $X$ is a dominating set of $G$.

%     \item \emph{$G[X]$ is connected:} Let $v$ and $w$ be any two vertices in $X$. Then each of $v$ and $w$ has at at least one corresponding vertex $v'$ and $w'$, respectively, in $G'$.   Since $X'$ is a connected dominating set of $G'$, there exists a path $P':=v',z_0,\ldots,z_r,w'$ in $G'$ such that $z_0,\ldots,z_r$ is path in $G'[X']$.  If $P'$ does not contain any edge of $E'$ then $P'$ has a corresponding path in $G[X]$, so $v$ and $w$ are in the same component of $G[X]$.  If some edge $x'y'$ of $P'$ is in $E'$ then $x'$ and $y'$ are on the same hole in $\Sigma\bbslash Z$.  Then $x'$ and $y'$ are copies of two vertices $x$ and $y$ of $G$ that are contained in the same component of $Z$.  In this case, we can replace the edge $x'y'$ with a path, in $Z$, from $x$ to $y$.  Doing this for each edge of $P'$ that is in $E'$ shows that there is a walk in $G[X]$ from $v$ to $w$, for each pair $v,w\in X$. Therefore $G[X]$ is connected.

%     \item \emph{$X$ has size $10n/21+O(\sqrt{gn})$:} The size of $X$ is at most $|X'| + |V(Z)|\le 10n'/21 + O(\sqrt{gn}) \le 10n/21 + O(\sqrt{gn})$.
% \end{itemize}
% Therefore, $X$ is a connected dominating set of $G$ of size at most $10n/21+ O(\sqrt{gn})$.
% \end{proof}
% % add to $X'$ the vertices of $Z$ after deleting the edges in $E'$ and undoing the slicing along $Z$.



% By adding the vertices in $Z$, we ensure that every vertex that was dominated through an edge in $E'$ is in the dominating set. Thus, $X$ is a dominating set of $G$.  It remains to prove that $G[X]$ is connected. We have the following cases.  If $z \in Z \cap X'$ then nothing to proof. If $z \in Z$ is dominated by a vertex $x' \in X'$ then $z$ still adjacent to $x'$ in $X'$. Otherwise, $z \in Z$ was dominated by $z' \in Z$, then $z$ and $z'$ are connected by a path in $Z$ after adding $Z$ to $X'$; $Z$ will never be an isolated vertex since $Z$ is minimal. If $Z$ is minimal, then it doesn't have any vertices of degree less than $1$, since removing a degree-$1$ or less vertex $Z$ corresponds to gluing together two adjacent edges on a hole in $\Sigma\bbslash Z$ or gluing the space at that point. It remains to bound the size of $X$.

%  $$|X| = |X'| + |V(Z)|,$$
%  $$|X| \leq \frac{10 n'}{21} + c\sqrt{gn},$$
%  $$|X| \leq \frac{10 (n + c\sqrt{gn})}{21} + c\sqrt{gn},$$
%  $$|X| \leq \frac{10n}{21} + (\frac{10}{21} + 1)c\sqrt{gn}.$$

%  Thus,
% $$|X| = \frac{10n}{21} + O(\sqrt{gn})$$
% as desired.

% \begin{patnote}
%   I think this result can be made stronger (with a simpler proof) using the following theorem (see Page 3 here \url{https://jeffe.cs.illinois.edu/teaching/comptop/2023/notes/22-planarization.pdf}):

%   \begin{thm}
%       Every surface triangulation $\Sigma$ with $n$ vertices and Euler genus $g<n$ has a planarizing
% subgraph $Z$ with $O(\sqrt{gn})$ vertices and edges, which can be computed in $O(n)$ time.
%   \end{thm}



%   Here, \defin{planarizing subgraph} means that if we cut the surface $\Sigma$ along each edge of $Z$, then the resulting surface $\Sigma\bbslash Z$ is homeomorphic to a disk with holes.  Alternatively, if we start with all the individual triangular faces and glue two faces $xyz$ and $yxw$ together along their shared edge $xy$ if and only if $xy\not\in E(Z)$ then the surface we get is homeomorphic to a disk with holes.  Notice that, if $Z$ is minimal, then it doesn't have any vertices of degree $1$, since removing a degree-$1$ vertex $Z$ corresponds to gluing together two adjacent edges on a hole in $\Sigma\bbslash Z$.

%   If we call $G$ the triangulation that is embedded in $\Sigma$, then we get a graph $G'$ where each edge of $Z$ appears twice and each vertex $v\in V(X)$ appears $\deg_X(v)$ times.  So the graph $G'$ has $n':=n+O(\sqrt{gn})$ vertices. Now add edges to triangulate $G'$ and apply \cref{main_result2} to get a connected dominating set $X'$ of $G'$ of size $10n'/21=10n/21 + O(\sqrt{gn})$.  Then set  $X:=X'\cup V(Z)$ to get a connected dominating set of $G$ that has size $10n/21 + O(\sqrt{gn})$.

%   Advantages: We get a result for $g$ up to $o(n)$ and it works for Euler genus, orientable or non-orientable.
% \end{patnote}

% \hussein{1. Define Cut Graph. 2. Epstein Theorem. 3. Rephrase theorem 6}


% \begin{patnote}
% We can go over the proof of this result tomorrow.  Erickson's notes are basically his explanation of Section 5 in this paper: \url{https://dl.acm.org/doi/pdf/10.5555/644108.644208}.  The separator $Z$ consists of
% \begin{compactitem}
%     \item BFS layers $i+ak$ for some $i\in\{0,\ldots,k-1\}$ and all $a\in\mathbb{N}$, where $k\approx\sqrt{n/g}$;
%     \item plus $O(g)$ paths $P_1,\ldots,P_{r}$ where each path $P_j$ has both endpoints in layer $i+ak$ and is contained in layers $i+ak,\ldots,i+(a+1)k-1$, for some integer $a$.
% \end{compactitem}
% \end{patnote}


% In order to prove Theorem \ref{surf_result}, we will use the following lemmas.
% \begin{lem}[\cite{doi:10.1137/0401020}\label{noncont}]
%   Every n-vertex triangulation of an orientable surface $\Sigma$ of genus $g > 0$ has a non-separating cycle of length at most $O(\sqrt{n})$.
% \end{lem}
% Notice that Lemma \ref{noncont} implies Lemma \ref{to_g_zero}.
% \begin{lem}{\cite{doi:10.1137/0401020}\cite{GILBERT1984391}}\label{to_g_zero}
%     Any embedded triangulation $G$ with $n$ vertices and genus $g$ can be transformed into a genus-$0$ graph by cutting along $g$ cycles of total length $O(g\sqrt{n})$.
% \end{lem}


% \begin{proof}[Proof of Theorem \ref{surf_result}]
%     We will proceed by induction on the genus $g$. The base case follows immediately from Theorem \ref{main_result2}. Assume the results holds for triangulation of surfaces of genus $< g$. By Lemma \ref{noncont}, we have a non-contractible non-seperating cycle $C$ of length at most $O(\sqrt{n})$. The graph $G \setminus \setminus C$ has two copies of $C$, denote these cycles by $C^{\flat}$ and $C^{\sharp}$. For each of these cycles create a corresponding closed disk $D^{\flat}$  and $D^{\sharp}$. Then, identify the boundary of $D^{\flat}$ to the cycle $C^{\flat}$, similarly for $D^{\sharp}$. Next, add edges in faces bounded by $D^{\flat}$ and $D^\sharp$ so our graph $G\setminus\setminus C$ triangulates the surface obtained by slicing through $C$. Notice that the new surface has genus $< g$. Thus, we are home by induction, and we have connected dominating set of size at most $\frac{10 (n + |C|)}{21} + (g - 1) \sqrt{n + |C|}$. Now, consider the graph obtained by removing the edges inside $D^\flat$ and $D^{\sharp}$ and  gluing the cycles $C^\flat$ and $C^\sharp$. Observe, that some vertices of $C^{\flat}$ and $C^{\sharp}$ may not be dominated anymore. Thus, we add them to the new dominating set. The total size of the new dominating set is $\frac{10 (n + |C|)}{21} + (g - 1) \sqrt{n + |C|} + \sqrt{n} - |C|$.
% \end{proof}

% \pat{There is an issue here in that it's not clear what it means two cut $G$ along two (or more) $C_1$ and $C_2$ if $C_1$ and $C_2$ intersect.  If we cut along $C_1$ first to get a graph $G_1$, then $C_2$ is not a cycle in the embedded graph $G_1$ since it contains consecutive vertices $x'$ and $x''$ with no edge between them.}


% \begin{proof}[Proof of Theorem \ref{surf_result}]
% Using Lemma \ref{to_g_zero}, we transform the graph $G$ to a graph $G'$ of genus zero, by cutting through a set of non-separating cycles $\mathcal{C}$ of size $g$. Then, we triangulate every face of $G'$.\pat{I think you should say here that the only non-triangular faces of $G'$ come from the caps that were glued on when cutting $G'$.} Thus, $G'$ became planar triangulation and Theorem \ref{main_result2} applies. Let $X'$ be the obtained dominating set of size at most $\frac{10|G'|}{21}$. To obtain a dominating set $X$ for $G$, we add to $X$ the vertices of the cycles in $\mathcal{C}$. It is immediate to verify that $X$ is connected, every vertex in a cycle $C \in \mathcal{C}$ is either in $X'$ or adjacent to a vertex in $X'$. It remains to bound the size of $X$.
% $$|X| = |X'| + |\bigcup_{C \in \mathcal{C}} V(C)|,$$
% $$|X| \leq \frac{10 |G'|}{21} + cg\sqrt{|G|},$$
% $$|X| \leq \frac{10 (n + cg\sqrt{n})}{21} + cg\sqrt{n},$$
% $$|X| \leq \frac{10n}{21} + (\frac{10}{21} + 1)cg\sqrt{n}.$$

% Thus,
% $$|X| = \frac{10n}{21} + O(cg\sqrt{n})$$
% as desired.

% \end{proof}


% A \defin{non-separating cycle} $C$ of a graph $G$ that is embedded on a surface $\Sigma$, is a cycle of $G$ whose removal leaves $G$ connected, in addition its embedding $\sigma(C)$ is a non-separating \emph{circle} in $\Sigma$.


% A \defin{non-separating circle} $C$ in $\Sigma$ such that $\Sigma \setminus C$ has one \defin{component}, an arc connected topological space. \defin{Cutting} $\Sigma$ along a \emph{non-separating} circle $C$, in a surface $\Sigma \neq \mathbb{S}^2$, is to create a new space $\Sigma'$ from $\Sigma$ by replacing every point $x \in C$ with two points $x',x''$. Then, consider any strip neighbourhood $N$ of $C$ in $\Sigma$, and let $X' := \{x' : x\in C\}$ and $X'' := \{x'' : x \in C\}$. Since $\Sigma$ is orientable, $N$ is a cylinder and $N\setminus C$ has two components $N'$ and $N''$ and we define their boundary circles to be $X'$ and $X''$, respectively. To turn $\Sigma'$ into a surface, we cap its holes by taking two disks disjoint from $\Sigma$ and identifying their boundary circles with $X'$, and $X''$ respectively.



%%%%%%%% The application section %%%%

\section{An Application in Graph Drawing: Proof of \cref{one_bend_collinear_thm}}
\label{one_bend}
This section demonstrates an application of connected dominating sets to graph drawing. We establish that each planar graph with a small connected dominating set has a one-bend drawing with a large collinear set. We start by introducing a topological equivalent of one-bend collinear sets as in \cite{DBLP:journals/jocg/LozzoDFMR18}.


% We are interested in finding bounds on the size of $h$-bend collinear set. For $0$-bend collinear set, constructions of non-Hamiltonian cubic triconnected planar graphs \cite{DBLP:journals/jct/GrunbaumW73,DBLP:conf/wg/RavskyV11,DBLP:journals/jocg/LozzoDFMR18} imply that planar triangulation has a 0-bend collinear set of size at most $O(n^{\sigma})$ where $\sigma < 0.986$. Moreover, every planar triangulation has a 0-bend collinear set of size at least $\Omega(\sqrt{n})$ \cite{DBLP:journals/dcg/BoseDHLMW09} \cite{DBLP:journals/jgaa/Dujmovic17}. In a recent work, Dujmovic and Morin \cite{DBLP:conf/compgeom/DujmovicM19} show that every planar triangulation with maximum degree $\Delta$ has $0$-bend collinear set of size at least $\Omega(\frac{n^{0.8}}{\Delta^4})$.

% For the one-bend drawing of planar triangulation with $n$ vertices,  Everett et al. \cite{DBLP:conf/gd/EverettLLW07} show the existence of a set $\mathcal{U}$ of $n$ distinct points in the plane such that every $n$-vertex planar graph admits an embedding on vertex set $\mathcal{U}$ with at most one bend along each edge. Furthermore, Giacomo et al. \cite{DBLP:journals/comgeo/GiacomoDLW05} show that for a linear ordering $L$ of vertices of a planar triangulation $G$ and strictly convex curve $\lambda$, there is a one-bend planar drawing of $G$ such that vertices of $G$ appear on $\lambda$ with the same order in $L$.

% With further relaxation on the drawing of the edges, de Fraysseix et al. \cite{DBLP:journals/combinatorica/FraysseixPP90} show that any set of $n$ points in the plane is a universal set for the two-bend drawing of planar graphs. Furthermore, They show that the planar embedding problem of any $n$-vertex graph onto an $n$ collinear points in $\mathbb{R}^2$ with at most one bend along each edge is NP-complete.

\subsection{Characterisation of 1-Bend Collinear Sets}
A \defin{curve} $C$ is a continuous mapping from $[0, 1]$ to $\mathbb{R}^2$. We usually call $C(0)$ and $C(1)$ the \defin{endpoints} of $C$. If $C(0)=C(1)$ then  the curve is \defin{closed}. Otherwise, it is \defin{open}. A curve $C$ is called \defin{simple} if $C$ is $C(x) \neq C(y)$ for all $0\le x<y\le 1$ with the exception of $x=0$, $y=1$. $C$ is a \defin{Jordan Curve} if it is simple and closed.

Let $G$ be plane graph, a Jordan curve $C$ is a \defin{$k$-proper good curve} if it contains a point in the interior of some face of $G$ (\defin{good}), and the intersection between $C$ and each edge $e$ of $G$ is empty, or at most $k$ points, or the entire edge $e$ (\defin{$k$-proper}).

% Da Lozzo et al.
\citet{DBLP:journals/jocg/LozzoDFMR18} characterize collinear sets in the straight line drawing of a planar graph using 1-proper good curves.

\begin{thm}[\cite{DBLP:journals/jocg/LozzoDFMR18} ] \label{straightline-topological}
Let $G$ be a plane graph. A set $S \subseteq V(G)$ is a collinear set if and only if there exists a $1$-proper good curve that contains $S$.
\end{thm}

The following lemma, illustrated in \cref{subdivisions}, gives a similar condition for one-bend collinear sets.

\begin{figure} [htbp]
  \centering
  \includegraphics[page=6]{figs/proper_good} \\
  $\Downarrow$ \\
  \includegraphics[page=7]{figs/proper_good}
  \caption{Subdividing $G$ so that a $2$-proper good curve $C$ becomes a $1$-proper good curve for the subdivided graph $G^+$.}
  \label{subdivisions}
\end{figure}

\begin{obs} \label{1-bend-topological}
    Let $G$ be a plane graph. A set $S \subseteq V(G)$ is a one-bend collinear set if $G$ has a $2$-proper good curve $C$ that contains $S$.
\end{obs}

\begin{proof}
Let $C$ be a 2-proper good curve that contains $S$. For each edge $e \in E(G)$ such that $|C \cap e| = 2$, we introduce a new subdivision vertex $u_e$ between the two intersection points of $C$ and $e$. By adding these new vertices, we obtain a plane drawing of a subdivision of $G$, denoted as $G^+$. Since every edge of $G^+$ is intersected by $C$ at most once, $C$ is a $1$-proper good curve for $G^+$. Thus, by \cref{straightline-topological}, $S$ is a collinear set for $G^+$. Note that a straight line drawing of $G^+$ is a one-bend drawing for $G$. Therefore, $S$ is a one-bend collinear set for $G$.
\end{proof}


\subsection{From a Spanning Tree to a One-bend Collinear Set}

We prove that the leaves of a spanning tree of a planar graph induce a one-bend collinear set. Precisely, we prove the following theorem.

\begin{lem} \label{spanning_tree_to_collinear_set}
  Let $G$ be a planar graph and $T$ be a spanning tree of $G$. Then, the leaves of $T$ form a one-bend collinear set for $G$.
\end{lem}

\begin{proof}
    Let $\Gamma$ be a straight-line drawing of $G$.
    By \cref{1-bend-topological}, it is enough to introduce a 2-proper good curve $\ell$ on $\Gamma$ containing all the leaves of $T$. To navigate the curve $\ell$ on the drawing $\Gamma$, we construct an envelope around $\Gamma$ as follows. For each vertex $v \in V(G)$, we draw a small circle, $C_v$, centered at $v$. We make the radii of the circles small enough such that each vertex $v \in V(G)$, $C_v$ intersects only the edges incident to $v$ and it is disjoint from all the other circles that correspond to the other vertices. Moreover, for each edge $uv \in E(G)$, we draw two parallel segments on both sides of $uv$ with endpoints on the boundary of corresponding circles of $u$ and $v$. These parallel segments are close enough to the corresponding edges such that no two of them intersect. (see \cref{tree_walking}). Note that each edge $uv \in E(G)$ crosses the envelope exactly twice, once at $C_u$ and once at $C_v$.

    \begin{figure}
        \centering
        % \includegraphics[page=2]{figs/proper_good} \\
        % $\Downarrow$ \\
        \includegraphics[page=3]{figs/proper_good}\\
        $\Downarrow$ \\
        \includegraphics[page=4]{figs/proper_good}
        \caption{Constructing a $2$-proper good curve for $G$ that contains all the leaves of the tree $T$.}
        \label{tree_walking}
    \end{figure}


    Assume $T$ is rooted at an arbitrary vertex of degree at least 2. We build the curve $\ell$ on the envelope of $\Gamma$ as follows. Starting from the root, we traverse the tree in \textit{depth first search} order. For each edge $uv \in E(T)$, we add the segment on the right side of the traversal direction of $uv$ into the curve $\ell$.

    For each leaf $u$ of $T$, let $v_u$ be its neighbor in $T$. To include all the leaves of $T$ on the curve $\ell$, we join $u$ to the endpoint of segments around the edge $uv_u \in E(T)$ on $C_u$. To keep the curve $\ell$ closed, for each non-leaf vertex $u \in V(T)$, we append to $\ell$ the circular arcs from $C_u$ between the segments in $\ell$ in the order of the traversal. By the properties of the depth first traversal, $\ell$ is a closed curve. By construction, $\ell$ contains all the leaves of $T$ and all the other vertices of $T$ are inside $\ell$. Moreover, for each edge $uv \in E(G)$:

    \begin{enumerate}
        \item [(P1)] If $uv \in E(T)$ and neither $u$ nor $v$ is a leaf, then $|uv \cap \ell| = 0$,

        \item [(P2)] If $uv \in E(T)$ and either $u$ or $v$ is a leaf of $T$, then $|uv \cap \ell| = 1$, and

        \item [(P3)] If $uv \notin E(T)$, then $|uv \cap \ell| = 2$.
    \end{enumerate}

    Properties P1-P3 guarantee that $\ell$ is a 2-proper curve. Since the tree $T$ is not empty, $\ell$ intersects the circle of some vertex in $T$, so $\ell$ touches a face of $\Gamma$. Therefore, $\ell$ is $2$-proper good curve and by \cref{1-bend-topological}, there exists a one-bend collinear set for $G$ formed by the leaves of $T$.
\end{proof}

\begin{proof}[Proof of \cref{one_bend_collinear_thm}]
Let $G$ be an $n$-vertex planar graph. \cref{main_result2} implies that $G$ has a spanning tree with at least $11n/21$ leaves. Using this tree in \cref{spanning_tree_to_collinear_set} establishes \cref{one_bend_collinear_thm}.
\end{proof}

\section{Discussion}
\label{discussion}

% It is worth pointing out the following corollary of \cref{main_result2}:

% \pat{Finish this....}
% \begin{thm}
%   Let $G$ be an $n$-vertex graph and let $Y\subseteq V(G)$ be such that $G[Y]$ is connected and $G-Y$ is isomorphic to a generalized near-triangulation $G^*$ in which each vertex in $B(G^*)$ has a neighbour in $Y$.  Then $G$ has a connected dominating set of size at most $|Y|+10n/21$.
% \end{thm}

% \begin{proof}
%     Embed $G-Y$ as a generalized near triangulation and let $G'$ be the graph obtained by adding three vertices $x,y,z$ to this embedding and adding edges to obtain a triangulation in which $xyz$ is the outer face.  The graph $G'$ is a triangulation with $n-|Y|+3$ vertices so, by \cref{better_greedy} it has a connected dominating set $X_0$ of size at most $(10n-|Y|+12)/21$.
% \end{proof}

In the introduction we argued that, if $X$ is a connected dominating set in a triangulation $G$, then the induced graph $G[V(G)\setminus X]$ is an outerplane graph.  Although it is not immediately obvious, finding the largest induced outerplane graph in a triangulation $G$ is equivalent to the problem of finding the smallest connected dominating set.

\begin{thm}
  Let $G$ be a triangulation with $n\ge 4$ vertices, let $X$ be a minimum-sized connected dominating set of $G$, and let $Y$ be a maximum-sized subset of $V(G)$ such that all vertices of $G[Y]$ lie on a common face of $G[Y]$. Then $|X|+|Y|=n$.
\end{thm}

\begin{proof}
  Let $X$ be a minimum-size connected dominating set of $G$ and let $Y'=V(G)\setminus X$.  Then $G[Y']$ is outerplane, since every vertex in $Y':=V(G)\setminus X$ is on the boundary of the face of $G[Y]$ that contains all vertices of $X$ in its interior.  Since $Y$ has maximum size $|X|+|Y|\ge |X|+|Y'|= n$.

  Now consider a set $Y$ of maximum size such that all vertices of $G[Y]$ lie on a common face $F_Y$ of $G[Y]$ and, among all such maximum-size sets, choose $Y$ to maximize the number of vertices of $G$ that are contained in the interior $F_Y$.  Without loss of generality, suppose $F_Y$ is the outer face of $G[Y]$, so $G[Y]$ is outerplane.  Let $X':=V(G)\setminus Y$. Since $n\ge 4$, $Y$ does not contain all three vertices on the outer face of $G$, so $X'$ dominates the vertices on the outer face of $G$. For any vertex $w\in Y$ not on the outer face of $G$, some neighbour of $v\in N_G(w)$ is in $X'$ since, otherwise $G[N_G(v)]\subseteq G[Y]$ contains a cycle with $w$ in its interior, contradicting the fact that $G[Y]$ is outerplane. Therefore $X'$ is a dominating set of $G$.

  We now show that all vertices of $X'$ are in the outer face of $G[Y]$, which implies that $G[X']$ is connected.  Suppose, by way of contradiction, that some inner face $F$ of $G[Y]$ contains at least one vertex of $X'$ in its interior.  Since $G$ is connected, there is at least one vertex $v\in V(F)$ such that $N_G(v)$ contains at least one vertex in the interior of $F$.  Let $Z:=\{w\in N_G(v):\text{$w$ is in the interior of $F$}\}$. Let $Y':=Y\cup Z \setminus\{v\}$.  Then $|Y'|\ge |Y|$, $G[Y']$ is outerplane, and the outer face of $G[Y']$ contains more vertices of $G$ than the outer face of $G[Y]$. This contradicts the choice of $Y$.

  Therefore $X'$ is a connected dominating set of $G$.  Since $X$ is of minimum size, $|X|+|Y|\le |X'|+|Y|=n$.  Therefore $n\le |X|+|Y|\le n$, so $|X|+|Y|=n$, as required.
\end{proof}


We conclude with two open questions:

\begin{compactenum}
  \item Is it true that every $n$-vertex triangulation has a connected dominating set of size at most $n/3+O(1)$?  A positive answer to this question seems to require additional new ideas.  In particular, it would seem to require a more global approach than the greedy approaches presented here. (This question is also posed by \citet[Question~4.2]{bradshaw.masarik.ea:robust}.)

  \item What is the maximum value $\alpha$ such that every $n$-vertex planar graph contains a one-bend collinear set of size $\alpha n-O(1)$?  \cref{one_bend_collinear_thm} shows $\alpha \ge 11/21$ and disjoint copies of the Goldner-Harary graph show that $\alpha \le 10/11$.

\end{compactenum}



\bibliographystyle{plainurlnat}
\bibliography{main}

% \appendix

% \section{Sage Code}

% \lstinputlisting[basicstyle=\ttfamily\scriptsize,language=Python]{cds.py}


\end{document}
