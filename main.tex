\documentclass[12pt]{article}
\usepackage[OT1]{fontenc}
\usepackage[utf8]{inputenc}
\usepackage{kpfonts}
\usepackage{graphicx}
\usepackage[dvipsnames]{xcolor}
\usepackage{amsthm}
\usepackage[margin=1.25in]{geometry}
\usepackage[sf,bf,small,raggedright,compact]{titlesec}
\usepackage{hyperref}
\definecolor{linkblue}{named}{MidnightBlue}
\hypersetup{colorlinks=true, linkcolor=linkblue,  anchorcolor=linkblue,
        citecolor=linkblue, filecolor=linkblue, menucolor=linkblue,
        urlcolor=linkblue}
\usepackage[capitalize]{cleveref}
\usepackage{paralist}
\usepackage[longnamesfirst,numbers,sort&compress]{natbib}



\newcommand{\pref}[1]{(P\ref{#1})}
\setlength{\parskip}{1ex}

\newtheorem{thm}{Theorem}
\newtheorem{obs}{Observation}
\newtheorem{clm}{Claim}
\newenvironment{clmproof}{\noindent\emph{Proof of Claim:}}{\hfill$\blacksquare$\par}
\newtheorem{lem}{Lemma}

\DeclareMathOperator{\dist}{d}
\DeclareMathOperator{\pack}{pack}
\DeclareMathOperator{\hit}{hit}

\newcommand{\defin}[1]{\emph{\textcolor{Maroon}{#1}}}

\newcommand{\pat}[1]{[\textcolor{red}{PM: #1}]}

\title{Connected Dominating Sets in Triangulations}
\author{Prosenjit Bose \and Vida Dujmović \and Hussein Houdrouge \and Pat Morin \and Saeed Odak \and Anyone Else?}
\date{October 2023}



\begin{document}

\maketitle

% \begin{abstract}
%   We show that every $n$ vertex triangulation $G$ has a spanning tree with at least $n/2\pm{?}$ leaves.  This improves the previous best bound of $2n/5\pm {?}$, due to Kleitman and West (1991). \pat{Is this true? I only find $n/3$.}
% \end{abstract}


\section{Introduction}

A set $X$ of vertices in a graph $G$ is a \defin{dominating set} if each vertex of $G$ is in $X$ or adjacent to a vertex in $X$.  A dominating set $X$ of $G$ is \defin{connected} if the induced graph $G[X]$ is connected.

% Observe that, for any connected dominating set $X$, $G$ has a spanning-tree in which all vertices in $V(G)\setminus X$ are leaves.  Similarly, by removing the leaves of any spanning tree of $G$ we obtain a tree whose vertex set is a connected dominating set of $G$.
%  Therefore, an $n$-vertex graph $G$ has a connected dominating set of size $q$ if and only if $G$ has a spanning tree with at least $n-q$ leaves.

The main result in this paper is the following:

\begin{thm}\label{main_result2}
  For any $n\ge 4$, any $n$-vertex triangulation $G$ has a connected dominating set $X$ of size at most $(7n-14)/13$ and there exists an algorithm that finds $X$ in linear time.
\end{thm}

\subsection{Related Work}

Dominating sets in graphs is an enormous field of research, with several books  devoted to the topic \cite{haynes.hedetniemi.ea:domination,haynes.hedetniemi.ea:topics}.

\pat{Expand.  Explain how Kleitman-West result gives $3n/4$.  Explain how Matheson-Tarjan gives $2n/3$.  Talk about connected dominating sets in unit disk graphs.}


% \begin{thm}\label{main_result}
%   For any $n\ge 4$, any $n$-vertex triangulation $G$ has a connected dominating set of size at most $4n/7 + O(1)$.
% \end{thm}


\section{The Proofs}

For a graph $G$, let $|G|=|V(G)|$ denote the number of vertices of $G$.  A \defin{bridge} in a graph $G$ is an edge $e$ of $G$ such that $G-e$ has more connected components than $G$.  For a vertex $v\in G$, $N_G(v):=\{w\in V(G):vw\in E(G)\}$ is the \defin{open neighbourhood} of $v$ in $G$,  $N_G[v]:=N_G(v)\cup\{v\}$ is the \defin{closed neighbourhood} of $v$ in $G$.  For a vertex subset $S\subseteq V(G)$, $N_G[S]:=\bigcup_{v\in S} N_{G}[v]$ is the \defin{closed neighbourhood} of $S$ in $G$ and $N_G(S):=N_G[S]\setminus S$ is the \defin{open neighbourhood} of $S$ in $G$.  A set $X\subseteq V(G)$ \defin{dominates} a set $B\subseteq V(G)$ if $B\subseteq N_G[X]$.  Thus, $X$ is a dominating set of $G$ if and only if $X$ dominates $V(G)$.

A \defin{plane graph} is a graph equipped with a non-crossing embedding in $\mathbb{R}^2$.  A plane graph is \defin{outerplane} if all its vertices appear on the outer face.  A \defin{triangle} is a cycle of length $3$. A \defin{near-triangulation} is a plane graph whose outer face is bounded by a cycle and whose inner faces are all bounded by triangles.  A \defin{generalized near-triangulation} is a plane graph whose inner faces are bounded by triangles.


For a plane graph $H$, we use the notation $B(H)$ to denote the vertex set of the outer face of $H$ and define $I(H):=V(H)\setminus B(H)$.  The vertices in $B(H)$ are \defin{boundary vertices} of $H$ and the vertices in $I(G)$ are \defin{inner vertices} of $H$. For any vertex $v$ of $H$, the \defin{inner neighbourhood} of $v$ in $H$ is defined as $N_H^+(v):=N_H(v)\cap I(H)$, the vertices in $N^+_H(v)$ are \defin{inner neighbours} of $v$ in $H$, and $\deg^+_H(v)=|N^+_H(v)|$ is the \defin{inner degree} of $v$ in $H$.

Let $G$ be a triangulation.  Our procedure for constructing a connected dominating set $X$ begins with an incremental phase that eats away at the triangulation $G$ ``from the outside.'' The process of constructing $X$ is captured by the following definition:   A vertex subset $X\subseteq V(G)$ is \defin{outer-domatic} if it can be partitioned into non-empty subsets $\Delta_0,\Delta_1,\ldots,\Delta_{r-1}$ such that
\begin{compactenum}[(P1)]
    \item $\Delta_0\subseteq B(G)$; \label{outer_face}
    \item $\Delta_i\subseteq B(G-(\bigcup_{j=0}^{i-1}\Delta_j))$ for each $i\in\{1,\ldots,r-1\}$; and \label{incremental}
    \item $G-(\bigcup_{j=0}^{r-1}\Delta_j)$ is outerplanar. \label{outerplanar}
\end{compactenum}

\begin{lem}\label{outer_domatic}
    Let $G$ be a triangulation.  Then any outer-domatic $X\subseteq V(G)$ is a connected dominating set of $G$.
\end{lem}

\begin{proof}
  Suppose $X$ is outer-domatic and let $\Delta_0,\ldots,\Delta_{r-1}$ be the corresponding partition of $X$.  For each $i\in\{1,\ldots,r\}$, let $X_i:=\bigcup_{j=0}^{i-1} \Delta_i$.  First observe that, since $\Delta_0\subseteq B(G)$ is non-empty, $X_i$ contains at least one vertex of $B(G)$, for each $i\in\{1,\ldots,r\}$. We claim that,
  \begin{compactenum}[(P1)]\setcounter{enumi}{3}
    \item for each $i\in\{2,\ldots,r\}$ each vertex in $B(G-X_{i-1})$ is adjacent to some vertex in $X_{i-1}$. \label{adjacent}
  \end{compactenum}
  Indeed, for any $i\in\{2,\ldots,r\}$ and any vertex $v\in B(G-X_{i-1})$ is either in $B(G)$ or adjacent to a vertex in $X_{i-1}$. Even in the former case, (P1) ensures that $v$ is adjacent to a vertex in $X_1=\Delta_0\subseteq X_{i-1}$, because $G[B(G)]$ is a clique.

  We now prove, by induction on $i$, that $G[X_i]$ is connected, for each $i\in\{1,\ldots,r\}$.
  The fact that $G[B(G)]$ is a clique and \pref{outer_face} implies that $G[X_1]=G[\Delta_0]$ is connected. For each $i\in\{2,\ldots,r\}$, the assumption that $G[X_{i-1}]$ is connected, \pref{incremental}, and \pref{adjacent} then imply that $G[X_i]=G[X_{i-1}\cup\Delta_{i-1}]$ is connected.

  In particular $G[X_r]=G[X]$ is connected.  Finally, \pref{adjacent}, with $i=r$ and \pref{outerplanar} implies that $N_G(X_r)=B(G-X_r)=V(G-X_r)$, so $X_r=X$ is a dominating set of $G$.
\end{proof}

We will present two algorithms that grow a connected dominating in small batches $\Delta_0,\Delta_1,\ldots,\Delta_{r-2}$ that result in a sequence of sets $X_1,\ldots,X_{r-1}$ where $X_{i}=\bigcup_{j=0}^{i-1}\Delta_j$.  However, each of these algorithms is unable to continue once they reach a point where each vertex in $B(G-X_i)$ has inner-degree at most $1$ in $G-X_i$.  We begin by studying the graphs that cause this to happen.

\subsection{Critical Graphs}

A generalized near-triangulation $H$ is \defin{critical} if $\deg^+_H(v)\le 1$ for each $v\in B(H)$.


\begin{lem}\label{base_case}
    Let $H$ be a critical generalized near-triangulation. Then $|B(H)|\ge 3|I(H)|$ and there exists $\Delta\subseteq B(H)$ of size at most $|I(H)|$ that dominates $I(H)$.
\end{lem}

\begin{proof}
  Let $B:=B(H)$ and $I:=I(H)$.  If $I$ is empty then the result is trivially true, by taking $X:=\emptyset$, so we now assume that $I$ is non-empty.  By definition, the graph $H[B]$ is outerplanar.  Say that an inner face of $H[B]$ is \defin{marked} if it contains an inner vertex of $H$.  Consider some marked face $f$ of $H[B]$.  This face is marked because it contains at least one vertex in $I$.  Since $H$ is a triangulation, there is an edge $vx$ in $H$ with $v\in B$ on the boundary of $f$ and $x\in I$ in the interior of $f$. Since $G$ is a triangulation and $x$ is an inner vertex of $H$, the edge $vx$ is on the boundary of two faces $vxv_1$ and $vxv_{k-1}$ of $H$ with $v_1\neq v_{k-1}$.  Since $\deg^+_H(v)=1$, each of $v_1$ and $v_{k-1}$ are in $B$.  By the same argument, $H$ contains a face $v_1xv_2$ with $v_2\in B$, $v_2\neq v$, and repeating this argument shows that $v,v_1,v_2,\ldots,v_{k-1}$ is the cycle in $H[B]$ that bounds $f$.  Therefore, $f$ contains exactly one vertex $x_f:=x$ of $I$ and $x_f$ is adjacent to each vertex of $f$.  Thus, $H$ is formed from an outerplanar graph $H[B]$ by adding $|I|$ stars, one in the interior of each marked face of $H[B]$.  Furthermore, since $\deg_H^+(v)=1$ for each $v\in B$, each vertex of $H[B]$ is on the boundary of exactly one marked face.

    \begin{figure}
        \begin{center}
            \includegraphics[page=1]{figs/critical} \\
            % \includegraphics[page=2]{figs/critical} \\
        \end{center}
        \caption{Some critical graphs \pat{Add some examples where $H[B(H)]$ has some non-triangular faces}}
        \label{critical_fig}
    \end{figure}

  Consider the graph $H'$ obtained by adding edges to the inner faces of $H[B]$ so that each inner face is a triangle. For each marked face $f$ of $H[B]$, select one triangular face $t_f$ of $H'$ that is contained in $f$ and \defin{mark} $t_f$. Thus there is a bijection from the set of marked faces of $H[B]$ to the set of marked faces in $H'$ and both of these sets have size $|I|$.  Since each vertex in $B$ is on the boundary of at most one marked face of $H[B]$, each vertex in $B$ is on the boundary of at most one marked triangular face of $H'$.  Therefore, $|B| \ge 3|I|$ so $|H|=|B|+|I|\ge 4|I|$. By choosing one vertex from each marked face of $H$ we obtain the desired set $\Delta$.
\end{proof}

\section{A Simple Algorithm}

We start with the simplest possible greedy algorithm, that we call $\textsc{SimpleGreedy}(G)$, to choose $\Delta_0,\ldots,\Delta_{r-1}$.  Suppose we have already chosen $\Delta_0,\ldots,\Delta_{i-1}$ for some $i\ge 0$ and we now want to choose $\Delta_i$.  Let $X_i:=\bigcup_{j=0}^{i-1}\Delta_j$, let $G_i:=G-X_i$, and let $v_i$ be a vertex in $B(G_i)$ that maximizes $\deg^+_{G_i}(v_i)$.  During iteration $i\ge 0$, there are only two cases to consider:
\begin{compactenum}
    \item If $\deg^+_{G_i}(v_i)\ge 2$ then we set $\Delta_i\gets\{v_i\}$.
    \item If $\deg^+_{G_i}(v_i)\le 1$ for all $v\in G_i$ then $G_i$ is critical and this is the final step, so $r:=i+1$.  By \cref{base_case}, there exists $\Delta_i\subseteq B_i$ of size at most $|I_i|$ that dominates $I_i$. Then $X_r:=X_{r-1}\cup\Delta_{i}$ and we are done.
\end{compactenum}

\begin{thm}\label{simple_greedy}
  When applied to an $n$-vertex triangulation $G$,  $\textsc{SimpleGreedy}(G)$ produces a connected dominating set $X_r$ of size at most $(4n-9)/7$.
\end{thm}

\begin{proof}
By the choice of $\Delta_0,\ldots,\Delta_{r-1}$, $X_r$ is an outer-domatic subset of $V(G)$ so, by \cref{outer_domatic}, $X_r$ is a connected dominating set of $G$.  All that remains is to analyze the size of $X_r$.  For each $i\in\{1,\ldots,r\}$, let $D_i:=N_G[X_i]$ be the subset of $V(G)$ that is dominated by $X_i$, let $I_i:=V(G)\setminus D_i$ be the subset of $V(G)$ not dominated by $X_i$, and let $B_i:=N_G(I_i)$ be the vertices of $G$ that have at least one neighbour in each of $X_i$ and $I_i$.  We use the convention that $D_0:=B(G)$.

First observe that, for $i\in\{0,\ldots,r-2\}$, $|D_{i+1}|\ge |D_i|+\deg_{G_i}^+(v_i)$ since $D_{i+1}\supseteq D_i$ and $D_{i+1}$ contains the $\deg_{G_i}^+(v_i)$ inner neighbours of $v_i$ in $G_i$.  Therefore
\[
    |D_{r-1}| \ge |D_0| + \sum_{i=0}^{r-2} \deg_{G_i}^+(v_i) \ge 3 + \sum_{i=0}^{r-2} 2 =  2r+1 \enspace . \label{double_d}
\]
Since $D_{r-1}$ and $I_{r-1}$ partition $V(G)$,
\begin{equation}
  n = |D_{r-1}| + |I_{r-1}| \ge 2r+1 + |I_{r-1}|  \enspace . \label{c1}
\end{equation}
% so
% \begin{equation}
%     r\le \frac{n-|I_{r-1}|-1}{2}  \enspace .
% \end{equation}

Since $X_{r-1}$ and $B_{r-1}$ are disjoint and $D_{r-1}\supseteq B_{r-1}\cup X_{r-1}$, we have $|D_{r-1}|\ge |X_{r-1}| + |B_{r-1}|=r-1+|B_{r-1}|$.  Therefore,
\begin{align}
    n & = |D_{r-1}| + |I_{r-1}| \ge r-1 + |B_{r-1}| + |I_{r-1}| = r-1 + |B_{r-1}| + |I_{r-1}| \notag
    \\
    & \ge r - 1 + 4|I_{r-1}| \enspace , \label{c2}
\end{align}
where the last inequality follows from \cref{base_case}.

The final dominating set $X_r$ has size $|X_r| = |X_{r-1}| + \Delta_{r-1} = r - 1 +|I_{r-1}|$, so the size of $|X_r|$ can be upper-bounded by maximizing $r-1+|I_{r-1}|$ subject to \cref{c1,c2}.  More precisely, by setting $x:=r$ and $y:=|I_{r-1}|$, the maximum size of $X_r$ is upper-bounded by the maximum value of $x-1+y$ subject to the constraints
\begin{align*}
  x - 1 + 4y & \le n \\
  2x + 1 + y & \le n
\end{align*}

% r\le (n+1-I_{r-1})/2$ and $r+4|I_{r-1}|\le n$.
This is an easy linear programming exercise and the maximum value of $X_{r}$ is obtained when $r=(3n-5)/7$ and $|I_{r-1}|=(n+3)/7$, which gives
$|X_r| \le (4n-9)/7$.
\end{proof}


We note that the implementation of $\textsc{SimpleGreedy}(G)$ is even simpler than the definition given above.  Nothing special needs to be done for the critical graph $G_{r-1}$.  Repeatedly selecting a vertex of maximum innner-degree and removing it will produce a dominating set of size exactly $|I_{r-1}|$.  Thus, $\textsc{SimpleGreedy}(G)$ has a simple linear time implementation.  In this implementation, each vertex $v$ stores a value $d_v$ which is initially set to $\deg_G(v)$.  For the three vertices on the outer face of $G$, $d_v$ is initially set to $\deg_G(v)-2$.  In general, $d_v$ is kept updated so that it is always equal to the inner-degree of $v$ in $G_i$.

Besides the data structure used for representing the triangulation $G$, each vertex $v$ also participates in a doubly-linked list $L_{d_v}$  that stores all the vertices with the same $d_v$ value.  A global doubly-linked list $L$ then stores all the lists $L_d$ such that $L_d$ is non-empty, sorted by increasing order of $d$.   Extracting a vertex of maximum inner-degree can then be done in constant time and the total time spent moving vertices between different lists in $L$ is proportional to the number of edges of $G$. Thus, the entire algorithm can be implemented in $O(n)$ time.

\section{A Less Simple Algorithm}

Next we devise an algorithm that produces a smaller connected dominating set than what is guaranteed by $\textsc{SimpleGreedy}(G)$.  This involves a more careful analysis of the cases in which \textsc{SimpleGreedy} is forced to take a vertex $v_i$ with $\deg^+_{G_i}(v_i)=2$.  We begin by identifying unnecessary vertices and edges that can appear in the graphs $G_1,\ldots,G_{r-1}$ during the construction of $X$.   We say that a near-triangulation $H$ is \defin{dom-minimal} if
\begin{compactenum}[({DM}1)]
    \item each vertex $v\in B(H)$ has $\deg^+_H(v)\ge 1$; and \label{bad_vertex}
    \item each edge $vw$ on the boundary of the outer face of $H$ is on the boundary an inner face $vwx$ of $H$ for some $x\in I(H)$. \label{bad_edge}
\end{compactenum}
We say that a generalized near-triangulation $H$ is \defin{dom-minimal} if each of its biconnected components are dom-minimal.

\begin{obs}
    Any dom-minimal generalized near-triangulation $H$ is bridgeless.
\end{obs}

\begin{proof}
   If $vw$ is a bridge in $H$ then both $v$ and $w$ are in $B(H)$.  Since $vw$ is a bridge in $H$, there is no inner face $vwx$ in $H$. Thus $H$ does not satisfy the second condition for dom-minimality.
\end{proof}

A subgraph $H'$ of a generalized near-triangulation $H$ is \defin{dom-preserving} if
\begin{compactenum}[({DP}1)]
  \item $B(H')\subseteq B(H)$;
  \item $N^+_{H'}(v)=N^+_H(v)$ for all $v\in B(H')$;
  \item $I(H')=I(H)$; and
  \item $N_{H'}(v)=N_H(v)$ for all $v\in I(H)$.
\end{compactenum}

\begin{obs}
  Let $H$ be a generalized near-triangulation, let $H'$ be a dom-preserving subgraph of $H$, and let $\Delta$ be a subset of $V(H)$ that dominates $I(H)$.  Then $\Delta\cap V(H')$ dominates $I(H)$.
\end{obs}

\begin{proof}
  Each vertex $v\in I(H')$ is adjacent to some vertex $w\in \Delta$.  Since $N_{H'}(v)=N_H(v)$, $w\in\Delta\cap V(H')$, so $v$ is dominated by $\Delta\cap V(H')$.  Since this is true for each $v\in I(H)=I(H')$, $\Delta\cap V(H')$ dominates $I(H)$.
\end{proof}

\begin{lem}\label{dom-minimal}
  For any generalized near-triangulation $H$, there exists a dom-preserving subgraph $H'$ of $H$ that is dom-minimal.
\end{lem}

\begin{proof}
  The proof is by induction on $|V(H)|+|E(H)|$.  If $H$ is already dom-minimal, then setting $H'=H$ satisfies the requirements of the lemma, so assume that $H$ is not dom-minimal.  It is straightforward to verify that the dom-preserving subgraph relationship is transitive, so if $H$ has a dom-preserving subgraph $H^*$ and $H^*$ has a dom-preserving subgraph $H'$ then $H'$ is a dom-preserving subgraph of $H$.  Therefore, it is sufficient to show the existence of a dom-preserving subgraph $H^*$ of $H$ with fewer edges or fewer vertices than $H$, and the inductive hypothesis provides the desired dom-minimal dom-preserving subgraph $H'$ of $H$.

  If $H$ contains a vertex $v\in B(H)$ with $\deg^+_H(v)=0$ then $H-v$ is a dom-preserving subgraph of $H$ with fewer vertices than $H$.  We now assume that $\deg^+_H(v)\ge 1$ for all $v\in B(H)$.  Since $H$ is not dom-minimal then $H$ contains a biconnected component $C$ that is not dom-minimal.
  \begin{compactenum}
    \item If there exists an edge $vw$ on the outer face of $C$ that is not incident to any inner face $vwx$ with $x\in I(C)$ then $B(H-vw)=B(H)$ and $I(H-vw)=I(H)$, and $H-vw$ is a is dom-preserving subgraph of $H$ that has fewer edges than $H$.

    \item If there exists a vertex $v\in B(C)$ with $\deg^+_C(v)=0$ then $v$ is incident to an edge $vw$ that is on the outer face of $C$ and on the outer face of $H$. Since $\deg^+_C(v)=0$, $vw$ is not incident to any inner face $vwx$ with $x\in I(C)$ and we can proceed as in the previous case. \qedhere
  \end{compactenum}
\end{proof}

We say that a generalized near-triangulation $H$ is \defin{good} if
\begin{compactenum}[(G1)]
      \item $H$ is critical;
      \item there exists $v\in B(H)$ with $\deg^+_H(v)\ge 3$; or
      \item there exists distinct $v,w\in B(H)$ such that $\deg^+_H(v)=2$ and $N^+_H(w)\subseteq N^+_H(v)$.
\end{compactenum}

\begin{lem}\label{not_good}
  Let $H$ be a dom-minimal generalized near-triangulation.  Then \begin{compactenum}[(i)]
    \item $H$ is good or
    \item $B(H)$ contains a vertex $v$ with $\deg^+_H(v)=2$ such that $G-v$ is good.
  \end{compactenum}
\end{lem}

\begin{proof}
  Since $H$ is dom-minimal, each vertex in $B(H)$ has inner-degree greater than $0$.  If some vertex in $B(H)$ has inner-degree at least $3$ then $H$ is good and there is nothing to prove, so we may assume that each vertex in $B(H)$ has inner-degree $1$ or $2$.  Since $H$ is not critical, at least one vertex in $B(H)$ has inner-degree $2$.

  If $H$ is $2$-connected, let $C:=H$ and let $v_0$ be any vertex of $H$ with inner-degree $2$.
  If $H$ is not $2$-connected, then let $C$ be a biconnected component of $H$ that contains a single cut-vertex $v_0$ of $H$ (so $C$ is a leaf in the block-cut tree of $H$). Then $v_0$ is contained in a second biconnected component $C'\neq C$ of $H$. Since $H$ is dom-minimal, $\deg^+_{C}(v_0)\ge 1$ and $\deg^+_{C'}(v_0)\ge 1$, so $\deg^+_H(v)= 2$.

  Since $C$ is biconnected its outer face is bounded by a cycle $v_0,v_1,v_2,\ldots,v_{k-1}$.  If $\deg^+_C(v_1)=1$ then, since $H$ is dom-minimal, the edge $v_0v_1$ is on the boundary of an inner face $v_0v_1x$ with $x\in I(H)$.  Since $v_0$ is the only cut-vertex of $H$ in $C$, $N^+_H(v_1)=N^+_C(v_1)=\{x\}\subseteq N_H(v_0)$ and $\deg^+_H(v_0)=2$, so $H$ is good and there is nothing to prove.  We may therefore  assume that $\deg^+_C(v_1)=2$ and,  by the same reasoning, that $\deg^+_C(v_i)=2$ for each $i\in\{1,\ldots,k-1\}$.  (Otherwise, $N^+_H(v_i)\subseteq N^+_H(v_{i-1})$ and $\deg^+_H(v_{i-1})=2$ for some $i\in\{2,\ldots,k-1\}$ so $H$ is good.)

  To summarize, so far we know that some vertex $x\in I(H)$ is in $N^+_H(v_0)$, $N^+_H(v_1)$, and $N^+_H(v_{k-1})$.  If $k=3$ then, since $H$ is dom-minimal, $H$ contains a face $v_1v_{k-1}y$ with $y\in I(H)$.  Since $\deg^+_H(v_1)=\deg^+_H(v_{k-1})=2$, $y\neq x$ and $N^+_H(v_1)=N^+_H(v_2)=\{x,y\}$. Again, this implies that $H$ is good.  We now assume that $k\ge 4$.

  The edge $v_1v_2$ is on the boundary of a face $v_1v_2y$.  Since $\deg^+_H(v_1)=2$, $x\neq y$.  If $N^+_H(v_1)=N^+_H(v_2)=\{x,y\}$ then $v_1$ and $v_2$ satisfy (GM3), so $H$ is good.  Otherwise, consider the graph $H':=H-v_0$.  Then $\deg^+_{H'}(v_1)=1$, $\deg^+_{H'}(v_2)=2$, and $N_{H'}(v_1)\subseteq N_{H'}(v_2)$.   Therefore $G'$ is good, so $G$ contains a vertex $v=v_0$ of inner-degree $2$ such that $G-v$ is good.  This completes the proof.
\end{proof}


This gives variant of the $\textsc{SimpleGreedy}(G)$ that we call $\textsc{BetterGreedy}(G)$.  Suppose we have already chosen $\Delta_0,\ldots,\Delta_{i-1}$ for some $i\ge 0$ and we now want to choose $\Delta_i$.  Let $X_i:=\bigcup_{j=0}^{i-1}\Delta_j$, let $G_i$ be a dom-preserving subgraph of $G-X_i$ that is dom-minimal, and let $v_i$ be a vertex in $B(G_i)$ that maximizes $\deg^+_{G_i}(v_i)$.  During iteration $i\ge 0$, there are now four cases to consider:
\begin{compactenum}
    \item If $\deg^+_{G_i}(v_i)\ge 3$ then we set $\Delta_i\gets\{v_i\}$.
    \item If there exists distinct $v_i,w\in B(H)$ such that $\deg^+_H(v_i)=2$ and $N^+_H(w)\subseteq N^+_H(v_i)$ then set $\Delta_i:=\{v_i\}$.
    \item If $\deg^+_{G_i}(v_i)\le 1$ for all $v\in G_i$ then $G_i$ is critical and this is the final step, so $r:=i+1$.  By \cref{base_case}, there exists $\Delta_i\subseteq B_i$ of size at most $|I_i|$ that dominates $I_i$. Then $X_r:=X_{r-1}+\Delta_{i}$ and we are done.
    \item Otherwise, $G_i$ is not good.  By \cref{not_good}, $G$ contains a vertex $v_i$ of inner-degree $2$ such that $G-v_i$ is good.  Set $\Delta_i:=\{v_i\}$.
\end{compactenum}

\begin{thm}\label{better_greedy}
  When applied to an $n$-vertex triangulation $G$,  $\textsc{BetterGreedy}(G)$ produces a connected dominating set $X_r$ of size at most $(7n-14)/13$.
\end{thm}

\begin{proof}
  Let $x_2$ be the number of times $\textsc{BetterGreedy}(G)$ falls into Case~4.
  For each integer $t\ge 3$, let $x_t$ be the number of times $\textsc{BetterGreedy}(G)$ falls into case $1$ and chooses $v_i$ such that $\deg^+_{G_i}(v_i)=t$.  Let $z$ be the number of times $\textsc{BetterGreedy}(G)$ falls into Case~2.  From \cref{not_good}, we know that Case~4 never occurs during two consecutive iterations.  Therefore,
  \begin{equation}
      x_2 \le \sum_{t\ge 3}x_t + z + 1 \enspace .  \label{a1}
  \end{equation}
  The size of the dominated set $D_{r-1}$ is at least $|D_{r-1}| \ge 3 + \sum_{t\ge 2}tx_t + 2z$.
  Since $|D_{r-1}|+|I_{r-1}|=n$,
  \begin{equation}
   3 + \sum_{t\ge 2}tx_t + 2z + I_{r-1} \le |D_{r-1}| + I_{r-1} \le n  \enspace . \label{a2}
  \end{equation}
  The of the boundary $B_{r-1}$ is at most $|B_{r-1}|\le 3+\sum_{t\ge 2}(t-1)x_t$.
  By \cref{base_case},
  \begin{equation}
    3+\sum_{t\ge 2}(t-1)x_t \ge |B_{r-1}| \ge 3|I_{r-1}| \label{a3}
  \end{equation}
  The size of the final set $X_{r}$ is
  \begin{equation}
    |X_r| = \sum_{t\ge 2}x_t + z + |I_{r-1}| \enspace . \label{objective2}
  \end{equation}
  The unbounded number of variables $x_t$, $t\ge 2$ is problematic. We resolve this with the following claim:
  \begin{clm}
    In any assignment of non-negative values to $|I_{r-1}|$, $z$, and $x_2,x_3,\ldots$ that maximizes \cref{objective2} subject to \cref{a1,a2,a3}, $x_t=0$ for each $t\ge 3$.
  \end{clm}
  \begin{clmproof}
     If $x_t = c >0$ for some $t\ge 3$, then we can set $x_t\gets 0$, $z\gets z+tc/2$, and $|I_{r-1}|\gets |I_{r-1}|-(t-1)c/3$.  This increases the value of \cref{objective2} by $tc/2-(t-1)c/3 =(t+2)c/6>0$.  The new values still satisfy \cref{a1} because the right hand side of \cref{a1} increases and the left hand side is unchanged.  The new values still satisfy \cref{a2} because the left hand side of \cref{a2} changes by $tc - 2tc/2 = 0$.  The new values still satisfy \cref{a3} because the left and right hand sides of \cref{a3} each decrease by $(t-1)c$.
  \end{clmproof}
  Therefore, \cref{a1,a2,a3} are three inequalities in three non-negative variables $x_2$, $z$, and $|I_{r-1}|$.  Maximizing \cref{objective2} subject to these constraints is an easy linear programming exercise that gives $x_2:=(3n-6)/13$, $z=(3n-19)/13$ and $I_{r-1}=(n+11)/13$, which gives $|X_r|= (7n+8)/13$.
\end{proof}

\bibliographystyle{plainurl}
\bibliography{main}

\end{document}
