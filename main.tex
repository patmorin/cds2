\documentclass[12pt]{article}
\usepackage[OT1]{fontenc}
\usepackage[utf8]{inputenc}
\usepackage{kpfonts}
\usepackage{graphicx}
\usepackage[dvipsnames]{xcolor}
\usepackage{amsthm}
\usepackage[margin=1.25in]{geometry}
\usepackage[sf,bf,small,raggedright,compact]{titlesec}
\usepackage{hyperref}
\definecolor{linkblue}{named}{MidnightBlue}
\hypersetup{colorlinks=true, linkcolor=linkblue,  anchorcolor=linkblue,
        citecolor=linkblue, filecolor=linkblue, menucolor=linkblue,
        urlcolor=linkblue}
\usepackage[capitalize]{cleveref}
\usepackage{paralist}
\usepackage[longnamesfirst,numbers,sort&compress]{natbib}
% \usepackage[normalem]{ulem}


\usepackage{listings,newtxtt}
\lstset{basicstyle=\ttfamily, keywordstyle=\bfseries}


\newcommand{\pref}[1]{(P\ref{#1})}
\setlength{\parskip}{1ex}

\newtheorem{thm}{Theorem}
\newtheorem{obs}{Observation}
\newtheorem{lem}{Lemma}
\crefname{lem}{Lemma}{Lemmata}
\newtheorem{cor}{Corollary}
\newtheorem{clm}{Claim}
\newenvironment{clmproof}{\noindent\emph{Proof of Claim:}}{\hfill{$\blacksquare$}\par}

\crefname{dm}{}{}
\creflabelformat{dm}{#2(DM#1)#3}
\crefname{dp}{}{}
\creflabelformat{dp}{#2(DP#1)#3}
\crefname{p}{}{}
\creflabelformat{p}{#2(#1)#3}
\crefname{bg}{}{}
\creflabelformat{bg}{#2[bg#1]#3}


\DeclareMathOperator{\corners}{\Gamma}
\DeclareMathOperator{\dist}{d}
\DeclareMathOperator{\pack}{pack}
\DeclareMathOperator{\hit}{hit}

\newcommand{\defin}[1]{\emph{\textcolor{Maroon}{#1}}}


\theoremstyle{definition}
\newtheorem{definition}{Definition}[section]
\newtheorem{observation}{Observation}

\newcommand{\pat}[1]{[\textcolor{red}{PM: #1}]}
\newcommand{\vida}[1]{[{\color{pink}VD: #1}]}
\newcommand{\saeed}[1]{[{\color{blue}SO: #1}]}
\newcommand{\hussein}[1]{[\textcolor{purple}{$H^2$: #1}]}

\title{Connected Dominating Sets in Triangulations}
\author{Prosenjit Bose \and Vida Dujmović \and Hussein Houdrouge \and Pat Morin \and Saeed Odak}
% \date{}



\begin{document}

\maketitle

\begin{abstract}
  We show that every $n$-vertex triangulation has a connected dominating set of size at most $10n/21$.  Equivalently, every $n$ vertex triangulation has a spanning tree with at least $11n/21$ leaves. Prior to the current work, the best known bounds were $2n/3$ and $n/3$, respectively. (In the special case of Hamiltonian triangulations the previous bounds were $n/2$.)  As an application of this, we show that every $n$-vertex planar graph has a one-bend non-crossing drawing in which some set of at least $11n/21$ vertices is drawn on the $x$-axis.
\end{abstract}

% \vida{Note that if all the vertices of G are on a line then no edge in a 1-bend drawing crosses the line and thus G is subhamiltonian. Since not all planar graphs are subhamiltonian, that implies that the tight bound for the drawings above is strictly less than n. I wonder how much smaller than n it has to be. Is it a fraction of n? Do we know anything about the side of the largest induced subhamiltonian graph in a planar triangulation? Whatever we have on that line induces a subhamiltonian planar graph. So basically the result above also says that each planar graph has induced subgraph of size at least 11n/21 that is subhamiltonian planar. Do not know if that is interesting. n/4 follows trivially from 4-colour theorem. n/2 follows from the fact that every other layer of BFS induced an outerplanar graph.

% In either case we should say that n is not possible (or even better if we know that c*n is not possible for some $c<1$)
% }

% \saeed{As far as I am aware, all we know is Goldner–Harary Graph (\url{https://en.wikipedia.org/wiki/Goldner–Harary_graph}) is the smallest triangulation that is not subhamiltonian. you can put 10/11 of the vertices on a line. so the trivial upper bound on the size of the one-bend collinear set is 10/11. It would be interesting if we could glue a lot of copies of this example nicely to get a better upper bound. We conjecture that the size of the connected dominating set is at most $\frac{n}{3}$.  \sout{I guess (might be wrong) that the size of the largest collinear set is also at most $\frac{2n}{3}$.}}

% \vida{what if we take many copies of that graph and add one dominating vertex adjacent to all the vertices on outer faces of GH graph. Did not think about this. It is the first thing that came to my mind}

% \pat{I added a discussion of the relationship with subhamiltonian graphs after the statement of Theorem 2.}

\section{Introduction}

A set $X$ of vertices in a graph $G$ is a \defin{dominating set} if each vertex of $G$ is in $X$ or adjacent to a vertex in $X$.  A dominating set $X$ of $G$ is \defin{connected} if the subgraph $G[X]$ of $G$ induced by $X$ is connected.  There is an enormous body of literature on dominating sets, and several books are devoted to the topic \cite{haynes.hedetniemi.ea:domination,haynes.hedetniemi.ea:topics}.  A typical result in the area is of the form: ``Every $n$-vertex graph in some family $\mathcal{G}$ of graphs has a (connected) dominating set of size at most $f(n)$.'' or ``For infinitely many $n$, there exists an $n$-vertex member of $\mathcal{G}$ with no (connected) dominating set of size less than $g(n)$.''

A \defin{triangulation} is an edge-maximal planar graph.  \citet{matheson.tarjan:dominating} proved that every $n$-vertex triangulation has a dominating set of size at most $n/3=0.33\overline{3}n$ and that there exists $n$-vertex triangulations with no dominating set of size less than $n/4=0.25n$. The gap between these upper and lower bounds stood for over $20$ years until a recent breakthrough by \citet{spacapan:domination} reduced the upper bound to $17n/53\approx 0.32075471698n$.  This was swiftly followed by an improvement to $2n/7\approx 0.2\overline{857142}n$ by \citet{christiansen.rotenberg.ea:triangulations}.

In the current paper, we consider connected dominating sets in triangulations.  An easy consequence of the proof used by \citet{matheson.tarjan:dominating} is that triangulations have connected dominating sets of size at most $2n/3=0.66\overline{6}n$.  A more general result, due to  \citet{kleitman.west:spanning} shows that graphs of minimum-degree $3$ in which each edge is included in a $3$-cycle have connected dominating set of size at most $2(n-5)/3<0.66\overline{6}n$.  \citet{kleitman.west:spanning} also show that graphs of minimum degree $4$ have connected dominating sets of size at most $(3n-8)/5<0.6n$. This result applies to triangulations of minimum degree $4$, which includes all $4$-connected triangulations.  The main result in this paper is the following:

\begin{thm}\label{main_result2}
  For every $n\ge 4$, every $n$-vertex triangulation $G$ has a connected dominating set $X$ of size at most $10n/21= 0.\overline{476190}n$.
\end{thm}

The proof of \cref{main_result2} is constructive and gives an $O(n)$ time algorithm for finding the set $X$.  As far as we know, the best lower bound for this problem is $n/3-2 < 0.33\overline{3} n$, obtained from any triangulation that contains $n/3$ vertex-disjoint pairwise-nested triangles $\Delta_1,\ldots,\Delta_{n/3}$.  In order to dominate $\Delta_1$, any dominating set must contain at least one vertex in $\Delta_1 \cup \Delta_2$. In order to dominate $\Delta_{n/3}$, any dominating set must contain a vertex in $\Delta_{n/3-1}\cup\Delta_{n/3}$.  In order to be connected, any connected dominating set must contain a vertex in each $\Delta_2,\ldots,\Delta_{n/3-1}$.

\saeed{we can force the structure between $\Delta_1$ and $\Delta_2$ s.t. we need 2 vertices from $\Delta_1 \cup \Delta_2$ in the dominating set}


The original motivation for this research was a graph drawing problem.  It is know that every $n$-vertex planar graph $G$ has a non-crossing drawing in the plane with edges of $G$ drawn as line segments and such that some set $Y$ of $\Omega(n^{1/2})$ vertices of $G$ are drawn on the $x$-axis \cite{bose.dujmovic:polynomial,dujmovic:utility}.  The set $Y$ is called a \defin{collinear set}.  For bounded-degree planar graphs, this result can be improved to $|Y|=\Omega(n^{0.8})$ \cite{dujmovic.morin:dual}.  Determining the supremum value of $\alpha$ such that every $n$-vertex planar graph has a collinear set of size $\Omega(n^{\alpha})$ remains a difficult open problem, but it is known that $\alpha < 1$ \cite{ravsky.verbitsky:collinear}.

We consider a relaxation of this problem in which the edges of the graph can be drawn as a polygonal path consisting of at most two line segments.  Such a drawing is called a \defin{$1$-bend} drawing of $G$.  A subset $Y$ of $V(G)$ is a \defin{$1$-bend collinear set} if $G$ has a $1$-bend drawing in which all vertices of $Y$ appear on the $x$-axis.  We show that, for any connected dominating set $X$ in a triangulation $G$, the set $Y:=V(G)\setminus X$ is a $1$-bend collinear set of $G$.  Combined with \cref{main_result2}, this gives:

\begin{thm}\label{one_bend_collinear}
  For every $n\ge 4$, every $n$-vertex planar graph has a $1$-bend collinear set of size at least $11n/21=0.\overline{523809}n$.
\end{thm}

Note that if all the vertices of $G$ are on the $x$-axis then no edge in a $1$-bend drawing of $G$ crosses the $x$-axis, so this one bend drawing is a $2$-page book-embedding of $G$. This implies that $G$ is a spanning subgraph of some Hamiltonian triangulation $G^+$.  The Goldner–Harary graph is an $11$-vertex triangulation that is not Hamiltonian. It follows that the graph $G$ obtained by taking $k$ vertex-disjoint copies of the Goldner-Harary graph has $n:=11k$ vertices and has no $1$-bend collinear set of size greater than $10k=10n/11=0.90\overline{90}n$.





\subsection{Related Work}

\pat{Add: \newline
results on edge-maximal outerplanar graphs --- implies $n/2$ bound for Hamiltonian planar graphs (including $4$-connected.) \newline
connected dominating sets in unit disc graphs \newline
$2$-bend collinear sets \newline
$15$-approximation for minimum connected dominating set in planar graphs \newline
}
\saeed{

For every $n \in \mathbb{N}$, Everett et al. \cite{DBLP:conf/gd/EverettLLW07} show that there exists a universal set $\mathcal{U}_n$ of $n$ distinct points in the plane such that every $n$-vertex planar graph admits a 1-bend drawing with vertices on the points of set $\mathcal{U}_n$. Furthermore, Giacomo et al. \cite{DBLP:journals/comgeo/GiacomoDLW05} demonstrate that for a linear ordering $L$ of vertices of a planar triangulation $G$ and strictly convex curve $\lambda$, there is a $1$-bend plane drawing of $G$ such that vertices of $G$ appear on $\lambda$ with the same order as in $L$.

With further relaxation on the drawing of the edges, de Fraysseix et al. \cite{DBLP:journals/combinatorica/FraysseixPP90} show that any set of $n$ points in the plane is a universal set for the 2-bend drawing of planar graphs. Furthermore, They show that the planar embedding problem of any $n$-vertex graph on an $n$ collinear points in $\mathbb{R}^2$ with at most one bend along each edge is NP-complete.
}

The remainder of this paper is organized as follows:  In \cref{strategy}, we describe the general strategy we use for finding connected dominating sets in triangulations.  In \cref{warm_up} we show that a simple version of this strategy can be used to obtain a connected dominating set of size at most $4n/7= 0.\overline{571428}n$, which is already better than the best known bounds for triangulations.  In \cref{full_result} we show that a more careful construction leads to a proof of \cref{main_result2}.  In \cref{one_bend}, we discuss the connection between connected dominating sets and one-bend collinear drawings that leads to \cref{one_bend_collinear}.

% \subsection{Related Work}
%
% \pat{Explain how Kleitman-West result gives $3n/4$.  Explain how Matheson-Tarjan gives $2n/3$. Mention results on connected dominating sets in maximal outerplanar graphs.  Talk about connected dominating sets in unit disk graphs.}


\section{The General Strategy}
\label{strategy}

For a graph $G$, let $|G|=|V(G)|$ denote the number of vertices of $G$.  A \defin{bridge} in a graph $G$ is an edge $e$ of $G$ such that $G-e$ has more connected components than $G$.  For a vertex $v\in G$, $N_G(v):=\{w\in V(G):vw\in E(G)\}$ is the \defin{open neighbourhood} of $v$ in $G$,  $N_G[v]:=N_G(v)\cup\{v\}$ is the \defin{closed neighbourhood} of $v$ in $G$.  For a vertex subset $S\subseteq V(G)$, $N_G[S]:=\bigcup_{v\in S} N_{G}[v]$ is the \defin{closed neighbourhood} of $S$ in $G$ and $N_G(S):=N_G[S]\setminus S$ is the \defin{open neighbourhood} of $S$ in $G$.  A set $X\subseteq V(G)$ \defin{dominates} a set $B\subseteq V(G)$ if $B\subseteq N_G[X]$.  Thus, $X$ is a dominating set of $G$ if and only if $X$ dominates $V(G)$.

A \defin{plane graph} is a graph equipped with a non-crossing embedding in $\mathbb{R}^2$.  A plane graph is \defin{outerplane} if all its vertices appear on the outer face.  A \defin{triangle} is a cycle of length $3$. A \defin{near-triangulation} is a plane graph whose outer face is bounded by a cycle and whose inner faces are all bounded by triangles.  A \defin{generalized near-triangulation} is a plane graph whose inner faces are bounded by triangles. Note that a generalized near triangulation may have multiple components, cut vertices, and bridges


For a plane graph $H$, we use the notation $B(H)$ to denote the vertex set of the outer face of $H$ and define $I(H):=V(H)\setminus B(H)$.  The vertices in $B(H)$ are \defin{boundary vertices} of $H$ and the vertices in $I(G)$ are \defin{inner vertices} of $H$. For any vertex $v$ of $H$, the \defin{inner neighbourhood} of $v$ in $H$ is defined as $N_H^+(v):=N_H(v)\cap I(H)$, the vertices in $N^+_H(v)$ are \defin{inner neighbours} of $v$ in $H$, and $\deg^+_H(v)=|N^+_H(v)|$ is the \defin{inner degree} of $v$ in $H$.

Let $G$ be a triangulation.  Our procedure for constructing a connected dominating set $X$ begins with an incremental phase that eats away at the triangulation $G$ ``from the outside.'' The process of constructing $X$ is captured by the following definition:   A vertex subset $X\subseteq V(G)$ is \defin{outer-domatic} if it can be partitioned into non-empty subsets $\Delta_0,\Delta_1,\ldots,\Delta_{r-1}$ such that
\begin{compactenum}[(P1)]
    \item $\Delta_0\subseteq B(G)$; \label{outer_face}
    \item $\Delta_i\subseteq B(G-(\bigcup_{j=0}^{i-1}\Delta_j))$ for each $i\in\{1,\ldots,r-1\}$; and \label{incremental}
    \item $G-(\bigcup_{j=0}^{r-1}\Delta_j)$ is outerplanar. \label{outerplanar}
\end{compactenum}

\begin{lem}\label{outer_domatic}
    Let $G$ be a triangulation.  Then any outer-domatic $X\subseteq V(G)$ is a connected dominating set of $G$.
\end{lem}

\begin{proof}
  Suppose $X$ is outer-domatic and let $\Delta_0,\ldots,\Delta_{r-1}$ be the corresponding partition of $X$.  For each $i\in\{1,\ldots,r\}$, let $X_i:=\bigcup_{j=0}^{i-1} \Delta_i$.  First observe that, since $\Delta_0\subseteq B(G)$ is non-empty, $X_i$ contains at least one vertex of $B(G)$, for each $i\in\{1,\ldots,r\}$. We claim that,
  \begin{compactenum}[(P1)]\setcounter{enumi}{3}
    \item for each $i\in\{2,\ldots,r\}$ each vertex in $B(G-X_{i-1})$ is adjacent to some vertex in $X_{i-1}$. \label{adjacent}
  \end{compactenum}
  Indeed, for any $i\in\{2,\ldots,r\}$ and any vertex $v\in B(G-X_{i-1})$ is either in $B(G)$ or adjacent to a vertex in $X_{i-1}$. Even in the former case, (P1) ensures that $v$ is adjacent to a vertex in $X_1=\Delta_0\subseteq X_{i-1}$, because $G[B(G)]$ is a clique.

  We now prove, by induction on $i$, that $G[X_i]$ is connected, for each $i\in\{1,\ldots,r\}$.
  The fact that $G[B(G)]$ is a clique and \pref{outer_face} implies that $G[X_1]=G[\Delta_0]$ is connected. For each $i\in\{2,\ldots,r\}$, the assumption that $G[X_{i-1}]$ is connected, \pref{incremental}, and \pref{adjacent} then imply that $G[X_i]=G[X_{i-1}\cup\Delta_{i-1}]$ is connected.

  In particular $G[X_r]=G[X]$ is connected.  Finally, \pref{adjacent}, with $i=r$ and \pref{outerplanar} implies that $N_G(X_r)=B(G-X_r)=V(G-X_r)$, so $X_r=X$ is a dominating set of $G$.
\end{proof}

We will present two algorithms that grow a connected dominating in small batches $\Delta_0,\Delta_1,\ldots,\Delta_{r-2}$ that result in a sequence of sets $X_1,\ldots,X_{r-1}$ where $X_{i}=\bigcup_{j=0}^{i-1}\Delta_j$.  Each of these algorithms is unable to continue once they reach a point where each vertex in $B(G-X_i)$ has inner-degree at most $1$ in $G-X_i$.  We begin by studying the graphs that cause this to happen.

\subsection{Critical Graphs}

A generalized near-triangulation $H$ is \defin{critical} if $\deg^+_H(v)\le 1$ for each $v\in B(H)$. We that an inner face of $H[B(H)]$ is \defin{marked} if it contains an inner vertex of $H$.

\begin{figure}[htbp]
    \centering
    \includegraphics[page=1]{figs/critical}
    \caption{Some critical graphs.}
    \label{critical_fig}
\end{figure}


\begin{lem}\label{critical_structure}
    Let $H$ be a critical generalized near-triangulation. Then each $f$ face of $H[B(H)]$ contains at most one vertex of $I(H)$ and this vertex is adjacent to every vertex of $f$.
\end{lem}

\begin{proof}
  By definition, the graph $H[B]$ is outerplanar.  Consider some marked face $f$ of $H[B]$.  This face is marked because it contains at least one vertex in $I$.  Since $H$ is a triangulation, there is an edge $vx$ in $H$ with $v\in B$ on the boundary of $f$ and $x\in I$ in the interior of $f$. Since $H$ is a generalized near-triangulation and $x$ is an inner vertex of $H$, the edge $vx$ is on the boundary of two faces $vxv_1$ and $vxv_{k-1}$ of $H$ with $v_1\neq v_{k-1}$.  Since $\deg^+_H(v)=1$, each of $v_1$ and $v_{k-1}$ are in $B$.  By the same argument, $H$ contains a face $v_1xv_2$ with $v_2\in B$, $v_2\neq v$, and repeating this argument shows that $v,v_1,v_2,\ldots,v_{k-1}$ is the cycle in $H[B]$ that bounds $f$.  Therefore, $f$ contains exactly one vertex $x$ of $I$ and $x$ is adjacent to each vertex of $f$.
\end{proof}

\begin{lem}\label{base_case}
    Let $H$ be a critical generalized near-triangulation. Then $|B(H)|\ge 3|I(H)|$ and there exists $\Delta\subseteq B(H)$ of size at most $|I(H)|$ that dominates $I(H)$.
\end{lem}

\begin{proof}
  Let $B:=B(H)$ and $I:=I(H)$.  If $I$ is empty then the result is trivially true, by taking $X:=\emptyset$, so we now assume that $I$ is non-empty.  By \cref{critical_structure}, $H$ is formed from the outerplanar graph $H[B]$ by adding $|I|$ stars, one in the interior of each marked face of $H[B]$.  Furthermore, since $\deg_H^+(v)=1$ for each $v\in B$, each vertex of $H[B]$ is on the boundary of exactly one marked face.  For each vertex $w\in I$, the marked face that $f$ of $H[B]$ that contains $w$ has at least $3$ vertices, which do not belong to any other marked face. Therefore $|B|\ge 3|I|$ and by choosing one vertex from each marked face of $H[B]$ we obtain the desired set $\Delta$.
\end{proof}

% \begin{lem}\label{two_critical_helper}
%   Let $H$ be a biconnected critical generalized near-triangulation with at least two vertices.  Then $H[B(H)]$ has a proper $3$-colouring such that
%   \begin{compactenum}[(i)]
%     \item For each face $f$ of $H[B(H)]$ each colour appeas at least once;
%     \item For each marked face $f$ of $H[B(H)]$, there is a vertex $v_f$ whose colour is different from every other vertex in $f$.
%   \end{compactenum}
% \end{lem}
%
% \begin{proof}
%  The proof is by induction on the number $p$ of inner faces of $H[B(H)]$ with four or more vertices.  If $p=0$ then $H=H[B(H)]$ is an edge-maximal outerplanar graph and $H$ has a proper $3$-colouring that easily satisifies the conditions of the lemma.
%
%  Otherwise, let $f$ be a face of $H[B(H)]$ with four or more vertices. Then, by \cref{critical_structure}, the interior of $f$ contains a single vertex $x\in I(H)$ that is adjacent to every vertex in $f$.  Let $H'$
%
%  Otherwise, let $v$ be a vertex in $I(H)$ and let
%
%
%
%  If $|H|=3$ then, since $H$ is connected and bridgeless, $H$ is a cycle $vrw$. Then the conditions of the lemma are easily satisfied with $X_0:=\{v\}$, $X_1:=\{r\}$, and $X_2:=\{w\}$.  We now assume that $|H|\ge 3$.
%
%   % If $r$ has degree $1$, then let $r'$ be the unique neighbour of $r$ in $H$ and let $H':=H-r$.  Then $r'\in B(H')$ so we apply the inductive hypothesis on the instance $(H',r')$ to obtain sets $X_0',X_1',X_2'$ where $r'\in X_1'$, each of $X_1'$ and $X_2'$ dominate $V(H')$ and $X_0'$ dominates $V(H')\setminus \{r'\}$. Then we take $X_0:=X_{2}'$, $X_1:=X_0'\cup\{r'\}$ and $X_2:=X_1'$.  Then $X_1$ dominates $V(H)$ since $X_0'$ dominates $V(H)\setminus\{r,r'\}$ and $N_H[r]=\{r,r'\}$. Since $r'\in X_1'$, $X_2$ also dominates $V(H)$. The final set $X_0=X_2'$ dominates $V(H')=V(H)\setminus\{r\}$, as required.
%
%   If $r$ is a cut vertex of $H$, then $H-r$ has $k\ge 2$ components $C_1,\ldots,C_k$. For each $i\in\{1,\ldots,k\}$, let $H_i:=H[V(C_i)\cup\{r\}]$ and observe that $r\in B(H_i)$.  For each $i\in\{1,\ldots,k\}$,  we apply the inductive hypothesis on the instance $(H_i,r)$ to obtaine three sets $X_{i,0},X_{i,1},X_{i,2}$. Then, for each $i\in\{1,\ldots,k\}$,  $r\in X_{i,1}$, each of $X_{i,1}$ and $X_{i,2}$ dominate $V(H_i)$, and $X_{i,0}$ dominates $V(H_i)\setminus\{r\}$.  Then taking $X_0:=\bigcup_{i=1}^k X_{i,0}$,$X_1:=\bigcup_{i=1}^k X_{i,1}$, and $X_2:=\bigcup_{i=1}^k X_{i,2}$ gives the desired subsets of vertices.
%
%   If $\deg^+H(r)=1$ then let $f$ be the face of $H[B]$ that contains the single vertex $x\in N_H(r)\cap I$. Let $H'$ be the graph obtrained from $H$ by contracting the edge $rx$, creating a new vertex $r'$.  Then we apply the inductive hypothesis on the instance $(H',r')$ to obtain sets $X_0'$, $X_1'$, and $X_2'$ where $r'\in X_1'$, $X_1'$ and $X_2'$ dominate $V(H')$ and $X_0'$ dominates $V(H')\setminus\{r'\}$.
%
%
%   .  Without loss of generality, $r'\in X_1'$, each of $X_1'$ ad $X_2'$ dominate $V(H')$ and $X_0'$ dominates $V(H)\setminus\{r'
%   \}$.
%
%
%   Otherwise, suppose $r$ is not
%
%   Otherwise, $r$ is a vertex on the boundary of some inner face $f$ of $H[B]$.  If $f$ is a triangle $vrw$, then let $
%
%   If $f$ has four or more vertices then, by \cref{critical_structure}, $r$ is adjacent to an inner vertex $x$ contains in $f$.
%
%
%
%
%    each of which is a dominating set of $H$ and
% \end{proof}





\section{A Simple Algorithm}
\label{warm_up}

We start with the simplest possible greedy algorithm, that we call $\textsc{SimpleGreedy}(G)$, to choose $\Delta_0,\ldots,\Delta_{r-1}$.  Suppose we have already chosen $\Delta_0,\ldots,\Delta_{i-1}$ for some $i\ge 0$ and we now want to choose $\Delta_i$.  Let $X_i:=\bigcup_{j=0}^{i-1}\Delta_j$, let $G_i:=G-X_i$, and let $v_i$ be a vertex in $B(G_i)$ that maximizes $\deg^+_{G_i}(v_i)$.  During iteration $i\ge 0$, there are only two cases to consider:
\begin{compactenum}[{[}g1{]}]
    \item If $\deg^+_{G_i}(v_i)\ge 2$ then we set $\Delta_i\gets\{v_i\}$.
    \item If $\deg^+_{G_i}(v_i)\le 1$ for all $v\in G_i$ then $G_i$ is critical and this is the final step, so $r:=i+1$.  By \cref{base_case}, there exists $\Delta_i\subseteq B_i$ of size at most $|I_i|$ that dominates $I_i$. Then $X_r:=X_{r-1}\cup\Delta_{i}$ and we are done.
\end{compactenum}

\begin{thm}\label{simple_greedy}
  When applied to an $n$-vertex triangulation $G$,  $\textsc{SimpleGreedy}(G)$ produces a connected dominating set $X_r$ of size at most $(4n-9)/7$.
\end{thm}

\begin{proof}
By the choice of $\Delta_0,\ldots,\Delta_{r-1}$, $X_r$ is an outer-domatic subset of $V(G)$ so, by \cref{outer_domatic}, $X_r$ is a connected dominating set of $G$.  All that remains is to analyze the size of $X_r$.  For each $i\in\{1,\ldots,r\}$, let $D_i:=N_G[X_i]$ be the subset of $V(G)$ that is dominated by $X_i$, let $I_i:=V(G)\setminus D_i$ be the subset of $V(G)$ not dominated by $X_i$, and let $B_i:=N_G(I_i)$ be the vertices of $G$ that have at least one neighbour in each of $X_i$ and $I_i$.  We use the convention that $D_0:=B(G)$.

First observe that, for $i\in\{0,\ldots,r-2\}$, $|D_{i+1}|\ge |D_i|+\deg_{G_i}^+(v_i)$ since $D_{i+1}\supseteq D_i$ and $D_{i+1}$ contains the $\deg_{G_i}^+(v_i)$ inner neighbours of $v_i$ in $G_i$.  Therefore
\[
    |D_{r-1}| \ge |D_0| + \sum_{i=0}^{r-2} \deg_{G_i}^+(v_i) \ge 3 + \sum_{i=0}^{r-2} 2 =  2r+1 \enspace . \label{double_d}
\]
Since $D_{r-1}$ and $I_{r-1}$ partition $V(G)$,
\begin{equation}
  n = |D_{r-1}| + |I_{r-1}| \ge 2r+1 + |I_{r-1}|  \enspace . \label{c1}
\end{equation}

Since $X_{r-1}$ and $B_{r-1}$ are disjoint and $D_{r-1}\supseteq B_{r-1}\cup X_{r-1}$, we have $|D_{r-1}|\ge |X_{r-1}| + |B_{r-1}|=r-1+|B_{r-1}|$.  Therefore,
\begin{align}
    n & = |D_{r-1}| + |I_{r-1}| \ge r-1 + |B_{r-1}| + |I_{r-1}| = r-1 + |B_{r-1}| + |I_{r-1}| \notag
    \\
    & \ge r - 1 + 4|I_{r-1}| \enspace , \label{c2}
\end{align}
where the last inequality follows from \cref{base_case}.

The final dominating set $X_r$ has size $|X_r| = |X_{r-1}| + \Delta_{r-1} = r - 1 +|I_{r-1}|$, so the size of $|X_r|$ can be upper-bounded by maximizing $r-1+|I_{r-1}|$ subject to \cref{c1,c2}.  More precisely, by setting $x:=r$ and $y:=|I_{r-1}|$, the maximum size of $X_r$ is upper-bounded by the maximum value of $x-1+y$ subject to the constraints
\begin{align*}
x,y \ge 0 \\
  x - 1 + 4y & \le n \\
  2x + 1 + y & \le n
\end{align*}

% r\le (n+1-I_{r-1})/2$ and $r+4|I_{r-1}|\le n$.
This is an easy linear programming exercise and the maximum value of $X_{r}$ is obtained when $r=(3n-5)/7$ and $|I_{r-1}|=(n+3)/7$, which gives
$|X_r| \le (4n-9)/7$.
\end{proof}

\subsection{An $O(n)$ Time Implementation}

We note that the implementation of $\textsc{SimpleGreedy}(G)$ is even simpler than the definition given above.  Nothing special needs to be done for the critical graph $G_{r-1}$.  Repeatedly selecting a vertex of maximum innner-degree and removing it will produce a dominating set of size exactly $|I_{r-1}|$.  Thus, $\textsc{SimpleGreedy}(G)$ has a simple linear time implementation.  In this implementation, each vertex $v$ stores a value $d_v$ which is initially set to $\deg_G(v)$.  For the three vertices on the outer face of $G$, $d_v$ is initially set to $\deg_G(v)-2$.  In general, $d_v$ is kept updated so that it is always equal to the inner-degree of $v$ in $G_i$.

Besides the data structure used for representing the triangulation $G$, each vertex $v$ also participates in a doubly-linked list $L_{d_v}$  that stores all the vertices with the same $d_v$ value.  A global doubly-linked list $L$ then stores all the lists $L_d$ such that $L_d$ is non-empty, sorted by increasing order of $d$.   Extracting a vertex of maximum inner-degree can then be done in constant time and the total time spent moving vertices between different lists in $L$ is proportional to the number of edges of $G$. Thus, the entire algorithm can be implemented in $O(n)$ time.

\section{A Better Algorithm}
\label{full_result}

Next we devise an algorithm that produces a smaller connected dominating set than what $\textsc{SimpleGreedy}(G)$ can guarantee.  This involves a more careful analysis of the cases in which \textsc{SimpleGreedy} is forced to take a vertex $v_i$ with $\deg^+_{G_i}(v_i)=2$.  In this section, we will show that during any step one of the following cases occurs.

% The commented out text below turns out to be completely bullshit
% \subsection{A Heuristic Analysis}
% Before diving into the details of how to make all this work, we present a heuristic analysis that explains why it gives a set $X$ of size at most $10n/21$.  Roughly speaking, Case~1 above implies that each vertex left in $G_{r-1}$ contributes $1/3$ to the size of the final dominating set $X_r$.  This implies that Case~2 above is more efficient than taking a vertex of inner-degree $3$.  Indeed, Case~1 adds two vertices $v_i$ and $y_i$ and increases the size of the boundary by $2$.  If the algorithm were to stop in the next iteration, these two new boundary vertices would contribute an additional $2/3$ to the size of $X_r$. Thus, Case~1 has a cost of $2+2/3=8/3$.  On the other hand, Case~1 decreases the number of undominated vertices by $5$, which is a measure of its progress.  On the other hand, choosing a vertex of inner-degree $3$ adds one vertex to $X$ and increases the size of the boundary by $2$, resulting in a cost of $1+2/3=5/3$ but only decreases the number of undominated vertices by $3$.  If we consider the ``progress over cost ratio'', the former case has a ratio of $5/(8/3)=15/8=1.875$ while the latter case has a ratio of $3/(5/3)=1.8$. We will formalize this informal argument by setting up a linear program to bound the maximum size of the final dominating set $X_r$. \pat{Maybe we can turn this informal argument into a potential function argument, instead?}

% In the end, optimizing this linear program shows that they hypothetical worst-case for the resulting algorithm is that it chooses inner-degree $3$ vertices for the first $r-1$ consecutive rounds, resulting in a graph $G_{r-1}$ with $|B(G_{r-1})|\approx 2r$ and $|I(G_{r-1})|\approx n-3r$, so $|G_{r-1}|\approx 2r+n-3r=n-r$.  The algorithm then finishes in Case~1, above, by taking a set $\Delta_{r-1}$ of size $|G_{r-1}|\approx (n-r)/3$.  This results in a dominating set $X_r$ of size roughly $r+(n-r)/3=n/3+(2/3)r$. This is a strictly increasing function of $r$, but the value of $r$ is constrained by the fact that $G_{r-1}$ has no vertices of inner-degree $3$, which implies that $|B(G_{r-1})|\ge |I(G_{r-1})|$.  This translates to the constraint $2r \le n-3r$ or $r \le n/5$.  Thus, the final dominating set has size at most $n/3 + (2/3)n/5=10n/21$.  We now make this heuristic analysis more formal.

\subsection{Dom-Respecting Graphs}

We begin by identifying unnecessary vertices and edges that can appear in the graphs $G_1,\ldots,G_{r-1}$ during the construction of $X$.   We say that a near-triangulation $H$ is \defin{dom-minimal} if
\begin{compactenum}[({DM}1)]
    \item each vertex $v\in B(H)$ has $\deg^+_H(v)\ge 1$;  \label[dm]{bad_vertex}
    \item for each $v\in B(H)$ with $\deg^+_H(v)=1$,  $H[N_H[v]]$ is isomorphic to $K_4$; and \label[dm]{inner_degree_1}
    \item each edge $vw$ on the boundary of the outer face of $H$ is on the boundary an inner face $vwx$ of $H$ for some $x\in I(H)$. \label[dm]{bad_edge}
\end{compactenum}
We say that a generalized near-triangulation $H$ is \defin{dom-minimal} if each of its biconnected components are dom-minimal.

\begin{obs}\label{bridgeless}
    Any dom-minimal generalized near-triangulation $H$ is bridgeless.
\end{obs}

\begin{proof}
   If $vw$ is a bridge in $H$ then both $v$ and $w$ are in $B(H)$.  Since $vw$ is a bridge in $H$, there is no path $vxw$ in $H$ and hence no inner face $vwx$ in $H$. Thus $H$ does not satisfy \cref{bad_edge}.
\end{proof}

% For a graph $H$ and an edge $vw$, $H_{v\leftarrow w}$ is the graph with vertex set $V(H)\setminus w$ and edge set $E(H)\cup\{vx : wx\in E(H)\}\setminus\{wx\in E(H)\}$. We say that $H_{v\leftarrow w}$ is the graph obtained from $H$ by performing the edge contraction $v\leftarrow w$.  The edge contraction $v\leftarrow w$ is a \defin{subcontraction} if $N_H(w)\subseteq N_H[v]$.  A graph $H'$ is a \defin{subminor} of a graph $H$ if $H'$ can be obtained from $H$ by a sequence of edge and vertex deletions and edge subcontractions. A subminor $H'$ of a generalized near-triangulation $H$ is \defin{dom-preserving} if
Let $H$ and $H'$ be two generalized near-triangulations.  We say that $H'$ \defin{dom-respects} $H$ if
\begin{compactenum}[({DP}1)]
  \item $B(H')\subseteq B(H)$; \label[dp]{boundary_subset}
  % \item $N^+_{H'}(v)\subseteq N^+_H(v)$ for all $v\in B(H')$; \label[dp]{same_inner_neighbourhood}
  \item $I(H')=I(H)$; and \label[dp]{same_inner_vertices}
  \item $N_{H'}(v)\cap I(H)\subseteq N_H(v)\cap I(H)$ for all $v\in V(H')$. \label[dp]{same_neighbourhood}
\end{compactenum}

\begin{obs}
  Let $H$ and $H'$ be generalized near-triangulations where $H'$ dom-respect $H$ and let $\Delta'$ be a subset of $V(H')$ that dominates $I(H')$ in $H'$.  Then $\Delta'$ dominates $I(H)$ in $H$.
\end{obs}

\begin{proof}
  By \cref{same_inner_vertices}, $I(H)=I(H')$. For each $w\in I(H)=I(H')$, $w\in \Delta'$ or there exists an edge $vw\in E(H')$ with $v\in\Delta'$.  In the latter case, $vw\in E(H)$ by \cref{same_neighbourhood}, so $\Delta'$ dominates $w$.
\end{proof}

\begin{lem}\label{dom_minimal}
  For any generalized near-triangulation $H$, there exists dom-minimal generalized near-triangulation $H'$ that dom-respects $H$.
\end{lem}

\begin{proof}
  The proof is by induction on $|V(H)|+|E(H)|$.  If $H$ is already dom-minimal, then setting $H'=H$ satisfies the requirements of the lemma, so assume that $H$ is not dom-minimal.  Since \cref{boundary_subset,same_inner_vertices,same_neighbourhood} are transitive relations, the dom-respecting relation is transitive: If $H'$ dom-respects $H^*$ and $H^*$ dom-respects $H$, then $H'$ dom-respects $H$.  Therefore, it is sufficient to find $H^*$ with fewer edges or fewer vertices than $H$ that dom-respects $H$, and the inductive hypothesis provides the desired dom-minimal graph $H'$ that dom-respects $H^*$ and $H$.

  If $H$ contains a vertex $v\in B(H)$ with $\deg^+_H(v)=0$ then $H-v$ is a generalized near-triangulation, $B(H-v)=B(H)$, $I(H-v)=I(H)$, and $N_{H-v}(v)\cap I(H)=N_{H}(v)\cap I(H)$ for all $v\in V(H-v)$. Therefore $H-v$ dom-respects $H$ and has fewer vertices than $H$.  We now assume that $\deg^+_H(v)\ge 1$ for all $v\in B(H)$.  Since $H$ is not dom-minimal then $H$ contains a biconnected component $C$ that is not dom-minimal. (See \cref{minimal_fig}.)
  \begin{figure}
    \centering
    \begin{tabular}{ccc}
      \includegraphics[page=1]{figs/minimal} &
      \includegraphics[page=3]{figs/minimal} &
      \includegraphics[page=5]{figs/minimal} \\
      $\Downarrow$ & $\Downarrow$ & $\Downarrow$ \\
      \includegraphics[page=2]{figs/minimal} &
      \includegraphics[page=4]{figs/minimal} &
      \includegraphics[page=6]{figs/minimal}
    \end{tabular}
    \caption{Three cases on the way to making $H$ dom-minimal.}
    \label{minimal_fig}
  \end{figure}
  \begin{compactenum}
    \item \cref{bad_edge}: If there exists an edge $vw$ on the outer face of $C$ that is not incident to any inner face $vwx$ with $x\in I(C)$ then $H-vw$ is a generalized near-triangulation, $B(H-vw)=B(H)$, and $I(H-vw)=I(H)$, and $N_{H-vw}(v)\cap I(H)=N_{H}(v)\cap I(H)$ for all $v\in V(H-vw)$. Therefore, $H-vw$ dom-respects $H$ and has few edges than $H$. (This includes the case where $C$ consists of a single edge.)

    \item \cref{bad_vertex}: If there exists a vertex $v\in B(C)$ with $\deg^+_C(v)=0$ then $v$ is incident to an edge $vw$ that is on the outer face of $C$ and on the outer face of $H$. Since $\deg^+_C(v)=0$, $vw$ is not incident to any inner face $vwx$ with $x\in I(C)$ and we can proceed as in the previous case.

    \item \cref{inner_degree_1}: If there exists a vertex $v\in B(C)$ with $\deg^+_C(v)=1$ then $H$ contains faces $xvw$ and $vyw$ where $x$ and $y$ are on the outer face of $C$ and $w$ is an inner vertex. If $H[N_H[v]]$ is not isomorphic to $K_4$, then $xy\not\in E(H)$.  In this case, let $H^\star$ be the graph obtained from $H$ by replacing the edges $xv$ and $vy$ with the edge $xy$. Then $H^\star$ is a generalized near-triangulation, $B(H^*)=B(H)$, $I(H^*)=I(H)$, and $N_{H^*}(v)\cap I(H)\subseteq N_{H}(v)\cap I(H)$.  Therefore $H^*$ dom-respects $H$ and has fewer edges than $H$. \qedhere
  \end{compactenum}
\end{proof}

\subsection{Finding a $2$--$3$ Combo}

Next we show that, in most cases our algorithm for constructing a connected dominating set is never forced to choose a single-vertex of inner-degree $2$. Note that the next two lemmas each consider a graph $H$ that is a near triangulation, not a generalized near-triangulation.

\begin{lem}\label{chord_incident}
  let $H$ be a dom-minimal near-triangulation and let $v_0$ be a vertex in $B(H)$ with $|N_H(v_0)\cap B(H)|\ge 3$.  Then $\deg^+_H(v_0)\ge 2$.  In other words, if $v_0$ is incident to a chord of the outerplane graph $H[B(H)]$, then $v_0$ is incident to at least two inner vertices of $H$.
\end{lem}

\begin{figure}[htbp]
  \centering
  \includegraphics{figs/chord_incident}
  \caption{The proof of \cref{chord_incident}}
  \label{chord_incident_fig}
\end{figure}
\begin{proof}
  Refer to \cref{chord_incident_fig}
  Since $H$ is a near-triangulation its outer face is bounded by a cycle $v_0,\ldots,v_{k-1}$.  Let $a:=\min\{i\in\{2,\ldots,k-2\}:v_0v_i\in E(H)\}$ and $b:=\max\{i\in\{2,\ldots,k-2\}:v_0v_i\in E(H)\}$. (Possibly $a=b$, but both $a$ and $b$ are well-defined since $|N^+_H(v_0)|\ge 3$.)   Since $H$ is dom-minimal, the edge $v_0v_1$ is on the boundary of an inner face $v_0v_1x$ of $H$ where $x$ is an inner vertex of $H$, by \cref{bad_edge}.  Since $H$ is dom-minimal, the edge $v_{k-1}v_0$ is on the boundary of an inner face $v_{k-1}v_0y$ of $H$ where $y$ is an inner vertex of $H$, by \cref{bad_edge}.  Then $x$ is in the interior of the face of $H[B(H)]$ bounded by the cycle $v_0,v_1,\ldots,v_a$ and $y$ is in the interior of the face of $H[B(H)]$ bounded by the cycle $v_0,v_b,\ldots,v_{k-1}$.  Therefore, $x\neq y$ and $N^+_H(v_0)\supseteq\{x,y\}$ so $\deg^+_H(v_0)\ge 2$.
\end{proof}

\begin{lem}\label{degree_2_outer_neighbour}
  Let $H$ be a dom-minimal near-triangulation. Then either:
  \begin{compactenum}
    \item $H$ is isomorphic to $K_4$;
    % \item $H$ consists of a cycle on the vertices in $B(H)$ plus one dominant inner vertex $w$; or
    \item each vertex $w\in B(H-B(H))$ has a neighbour $v$ in $H$ with $\deg^+_H(v)\ge 2$.
  \end{compactenum}
\end{lem}

\begin{proof}
  If $I(H)=\emptyset$ then the second condition of the lemma is trivially satisified, so there is nothing to prove. Otherwise, let $w$ be any vertex in $B(H-B(H))$ and let $v_0w$ be an edge of $H$ with $v\in B(H)$.
  % If $H$ is the empty graph, then there is nothing to prove. Otherwise $H$ is non-empty, so $B(H)$ is non-empty. Let $v$ be a vertex in $B(H)$.
  % By \cref{bad_edge}, $\deg^+_H(v)\ge 1$ so $H$ contains an edge $vw$ with $w\in I(H)$.
  % Consider the face $f:=v_0,\ldots,v_{k-1}$ of the outerplane graph $H[B(H)]$ that contains $w$.


   \begin{figure}[htbp]
     \centering
     \begin{tabular}{ccc}
       \includegraphics[page=1]{figs/outer_2_2} &
       \includegraphics[page=2]{figs/outer_2_2} &
       \includegraphics[page=3]{figs/outer_2_2}
     \end{tabular}
     % \pat{Redo this figure.}
     \caption{The proof of \cref{degree_2_outer_neighbour}.}
     \label{degree_2_outer_neighbour_fig}
   \end{figure}

   Refer to \cref{degree_2_outer_neighbour_fig}.
   By \cref{inner_degree_1}, $\deg^+_H(v)\ge 2$ or $H[N_H[v_0]]$ is isomorphic to $K_4$. In the former case the vertex $w$ satisfies the second condition of the lemma.  In the latter case, let $x$ and $y$ be the two neighbours of $v$ on the outer face of $H$, so $H[\{v,w,x,y\}]$ is isomorphic to $K_4$.    If the edge $xy$ is not on the outer face of $H$ then, by \cref{chord_incident}, $\deg^+_H(x),\deg^+_H(y)\ge 2$ so $w$ satisfies the second condition and we are done.  Otherwise, if $I(H)=w$ then $V(H)=\{v,w,xy\}$ and $H$ is isomorphic to $K_4$ and we are done.  Otherwise $I(H)$ at least one vertex $w'\neq w$.  Since $\deg^+_H(v)=1$, the cycle $vxwy$ has no vertices of $H$ in its interior, so $I(H)$ contains vertices in the interior of $xyw$.  By this implies that $\deg^+_H(x),\deg^+_H(y)\ge 2$.
\end{proof}


Note that the next three lemmas consider the case where $H$ is a generalized near triangulation.

\begin{lem}\label{really_good}
  Let $H$ be a dom-minimal generalized near-triangulation.  Then either:
  \begin{compactenum}[(1)]
    \item $H-B(H)$ is critical; \label[p]{two_critical}
    \item $B(H)$ contains a vertex $v$ with $\deg^+_H(v)\ge 3$; or \label[p]{degree_three}
    \item $H$ contains distinct vertices $v_0$, $v_j$, and $w$ such that
    \begin{compactenum}[(a)]
      \item $v_0\in B(H)$ and $\deg^+_H(v_0)=2$;
      \item $w\in B(H-v_0)$ and $\deg^+_{H-v_0}(w)\ge 3$; and
      \item $v_j\in B(H)$ and $N^+_H(v_j) \subseteq N_H[w] $.
    \end{compactenum}
    \label[p]{two_three_pair}
  \end{compactenum}
\end{lem}

\begin{proof}
  We will assume that $H$ does not satisfy \cref{two_critical} or \cref{degree_three} and show that $H$ must satisfy \cref{two_three_pair}.  Since $H-B(H)$ is not critical, $B(H-B(H))$ contains a vertex $w$ with $\deg_{H-B(H)}(w)\ge 2$.

  Let $C$ be the biconnected component of $H$ that contains $w$.  Then $C$ is a near-triangulation and we can apply \cref{degree_2_outer_neighbour} to $C$ and $w$. The first alternative in \cref{degree_2_outer_neighbour} is incompatible with the assumption that $\deg^+_{H-B(H)}(w)\ge 2$.  Therefore, we conclude that $N_H(w)\cap B(H)$ contains a vertex $v_0$ with $\deg^+_H(v_0)\ge 2$.  Since $H$ does not satisfy \cref{degree_three}, $\deg^+_H(v_0)< 3$, so $\deg^+_H(v_0)=2$.  Refer to \cref{really_good_fig}
  \begin{figure}
    \centering
    \begin{tabular}{ccc}
      \includegraphics[page=1]{figs/really_good} &
      \includegraphics[page=2]{figs/really_good} &
      \includegraphics[page=3]{figs/really_good}
    \end{tabular}
    \caption{The proof of \cref{really_good}}
    \label{really_good_fig}
  \end{figure}

  Let $v_0,\ldots,v_{k-1}$ be the cycle that bounds the inner face $f$ of $H[B(H)]$ that contains $w$ in its interior.  Since $H$ is a near-triangulation, $H$ contains triangles $v_0v_1x$ and $v_{k-1}v_0 y$ with $x$ and $y$ in the interior of or on the boundary of $f$.  Since $f$ is a face of $H[B(H)]$ each of $x$ and $y$ is in the interior of $f$.  At least one of $x$ or $y$ is equal to $w$, say $x$, since otherwise $\deg^+_H(v_0)\ge 3$.  Therefore $v_0v_1 w$ is an inner face of $H$.

  Let $j\ge 1$ be the maximum integer such that $v_{a-1}v_{a}w$ is an inner face of $H$ for all $a\in\{1,\ldots,j\}$.  Note that $j<k-1$ since, otherwise, the component of $H-B(H)$ that contains $w$ contains only a single vertex, contradicting the fact that $\deg^+_{H-B(H)}(w)=2$.
  % \footnote{In fact, $j\in\{0,1\}$ by \cref{inner_degree_1}, but this is not important for the current argument.}

  Since $H$ is a near-triangulation and $f$ is a face of $H[B(H)]$, $H$ has some face $v_j v_{j+1} z$ with $z$ in the interior of $f$.  By the definition of $j$, $z\neq w$.  Therefore, $N_H^+(v_j)\supseteq \{w,z\}$ and, since $\deg^+_H(v_j)\le 2$, $N_H^+(v_j))= \{w,z\}$.  Therefore $N^+_H(v_j)\subseteq N_H[w]$.  Since $f$ is a face of $H[B(H)]$, the only neighbours of $v_j$ in $B(H)$ are $v_{j-1}$ and $v_{j+1}$. Since $H$ is a near-triangulation, this implies that $w v_j z$ is a face of $H$.  In particular $wz$ is an edge of $H$.

  All that remains is to show that $\deg^+_{H-v_0}(w)\ge 3$.  First, observe that $z$ is in $B(H-B(H))$, so $z$ does not contribute to $\deg^+_{H-B(H)}(w)$. We claim that $z$ is in $I(H-v_0)$, so $z$ does contribute to $\deg^+_{H-v_0}(w)$.   Indeed, the only other possibility is that $z$ is adjacent to $v_0$.  In this case, consider the cycle $C:=v_0,\ldots,v_j,z$.  This cycle has $w$ in its interior. The vertices of $N^+_{H-B(H)}(w)$ must be in the interior of $C$. For each $a\in\{1,\ldots,j\}$, $v_{a-1}v_a w$ is a face of $H$, so the cycle $D:=v_0,\ldots,v_j,w$ does not contain any vertices of $N^+_{H-B(H)}(w)$ in its interior.  Therefore, the vertices in $N^+_{H-B(H)}(w)$ must be in the interior of the cycle $\overline{D}:=v_0,w,z$.  Since $H$ is a near-triangulation, this implies that at least one of $v_0$ is adjacent to some vertex in $I(H)\setminus\{w,z\}$. But this is not possible since it would imply that $\deg^+_H(v_0)\ge 3$.  Therefore $v_0$ is not adjacent to $z$, so $z$ is in the interior of $H-v_0$ and $N^+_{H-v_0}(w)\supseteq N^+_{H-B(H)}(w)\uplus\{z\}$, so $\deg^+_H(w)\ge 3$.
\end{proof}


The following is a restatement of \cref{really_good} in language that is more useful in the description of an algorithm for constructing a connected dominating set.

\begin{cor}\label{really_good_cor}
  Let $H$ be a dom-minimal generalized near-triangulation.  Then either:
  \begin{compactenum}[(1)]
    \item $H-B(H)$ is critical;
    \item there is a vertex $v\in B(H)$ and a dom-respecting subgraph $H'$ of  $H-v$ with $|H'|\le |H|-1$ and $|B(H')|=|B(H)|+2$; or
    \item there is an edge $vw\in E(H)$ with $v\in B(H)$, $w\in B(H-B(H))$, and a dom-respecting subgraph $H'$ of $H-\{v,w\}$ with $|H'|= |H|-3$ and $|B(H')|=|B(H)|+2$.
  \end{compactenum}
\end{cor}

\begin{proof}
  In the second case, the graph $H':=H-v$ has $|H'|=|H|-1$ and $|B(H')|=|B(H)|+2$. In the third case, the graph $H':=H-\{v_0,w,v_j\}$ has $|H'|=|H|-3$ and $|B(H')|=|B(H)|+2$.
\end{proof}


\subsection{Eliminating an Inner Leaf}

Next we show that, even when all vertices in $B(H)$ have inner-degree at most $2$ and $H-B(H)$ is critical, we can still efficiently dominate degree-$1$ vertices in $H-B(H)$.


\begin{lem}\label{leaf_killer}
  Let $H$ be a dom-minimal generalized near-triangulation such that $\deg^+_H(v)\le 2$ for all $v\in B(H)$, $H-B(H)$ is critical, and $H-B(H)$ contains a vertex $w$ with $\deg_{H-B(H)}(w)=1$.  Then there exists $v\in B(H)$ and a dom-respecting subgraph $H'$ of $H-v$ such that $|H'|\le |H|-3$ and $|B(H')|\le |B(H)|-1$.
\end{lem}

\begin{proof}
  Refer to \cref{killing_a_leaf}.
  Let $x$ be the unique neighbour of $w$ in $H-B(H)$. Since $x$ and $w$ are vertices of $H-B(X)$, $x,w\in I(H)$.  Since $w$ is an inner vertex in a near-triangulation, it is incident to $t\ge 3$ faces $v_iv_{i+1}w$ for $i\in\{0,\ldots,v_{t}\}$, with $v_0=v_t=x$.  Since $\deg_{H-B(H)}(w)=1$, $v_1,\ldots,v_{t-1}\in B(H)$.  Therefore, $H$ contains no edge $v_i v_{i+r}$ for any $i\in\{0,t-r\}$ and any $r\ge 2$.  Therefore, for each $i\in\{1,\ldots,t-1\}$, the only two inner faces of $H$ that include $v_i$ are $v_{i-1}v_iw$ and $v_iv_{i+1}w$.  Therefore $N^+_H(v_i)=\{w\}$ for each $i\in\{2,\ldots,t-2\}$ and $N^+_H(v_1)=N^+_H(v_{t-1})=\{x,w\}$.\footnote{In fact, \cref{inner_degree_1} implies that $t=3$, but this is not important for this proof.}

  \begin{figure}[htbp]
    \centering
    \begin{tabular}{ccc}
      \includegraphics[page=1]{figs/killing_a_leaf} &
      \includegraphics[page=2]{figs/killing_a_leaf} &
      \includegraphics[page=3]{figs/killing_a_leaf} \\
      $H$ & $H-v$ & $H'$
    \end{tabular}
    \caption{The proof of \cref{leaf_killer}}
    \label{killing_a_leaf}
  \end{figure}

  Let $v:=v_1$. Apply \cref{dom_minimal} to $H-v$ to get a dom-minimal graph $H'$ that dom-respects $H-v$.  Then $w,x\in B(H-v)$.  Since $H'$ is dom-minimal, $v_1,\ldots,v_{t-1}\not\in V(H')$, by \cref{bad_vertex}.  Therefore $N_H(w)\cap V(H')=\{v_t\}=\{x\}$. Since $H'$ is dom-minimal, $w\not\in V(H')$, by \cref{bad_vertex}. Therefore $V(H')\subseteq V(H)\setminus\{v_1,\ldots,v_{t-1},w\}$, so $|V(H')|\le |H|-t\le |H|-3$.  Finally, $B(H-v)\subseteq B(H)\setminus \{v\}\cup\{w,x\}$. By \cref{boundary_subset}, $B(H')\subseteq B(H-v)\setminus\{v_1,\ldots,v_{t-1},w\}\cup\{x\}$, so $|B(H')|\le |B(H)|+2-t\le |B(H)|-1$.
\end{proof}

\subsection{Eliminating Inner Isolated Vertices}

\begin{lem}\label{degree_zero_killer}
  Let $H$ be a dom-minimal generalized near-triangulation such that $\deg^+_H(v)\le 2$ for all $v\in B(H)$ and such that $H-B(H)$ contains a vertex $w$ with $\deg_{H-B(H)}(w)=0$ but does not contain any vertex $w'$ with $\deg_{H-B(H)}(w')=1$.  Then at least one of the following is true:
  \begin{compactenum}[(a)]
    \item there exists $v\in B(H)$ and a graph $H'$ that dom-respects $H-v$ such that $|H'|\le |H|-3$ and $|B(H')|\le |B(H)|-1$; or \label[p]{kill_pair}
    \item there exists $v\in B(H)$ and a graph $H'$ that dom-respects $H-v$ such that $|H'|\le |H|-4$ and $|B(H')|\le |B(H)|-3$. \label[p]{kill_k4}
  \end{compactenum}
\end{lem}

\begin{proof}
  By definition $N_H(w)\subseteq B(H)$ and, since $w$ is an inner vertex in a generalized near triangulation $|N_H(w)|\ge 3$. There are two cases to consider (see \cref{isolated_fig}):
  \begin{figure}[htpb]
    \centering
    \begin{tabular}{cc}
      \includegraphics[page=1]{figs/isolated} &
      \includegraphics[page=3]{figs/isolated} \\
      $\Downarrow$ & $\Downarrow$ \\
      \includegraphics[page=2]{figs/isolated} &
      \includegraphics[page=4]{figs/isolated}
    \end{tabular}
    \caption{Eliminating isolated vertices in $H-B(H)$.}
    \label{isolated_fig}
  \end{figure}
  \begin{compactenum}[(a)]
    \item There exists $v\in N_H(w)$ with $\deg^+_H(v)=2$. In this case, $N^+_H(v)=\{w,w'\}$ and the vertices $w$ and $w'$ are in a different biconnected components $C$ and $C'$, respectively, of $H$. By \cref{inner_degree_1}, $C[N_C[v]]$ is isomorphic to $K_4$. Since $\deg_{H-B(H)}(w)=0$, the faces of $C[N_C[v]]$ have no vertices of $H$ in their interiors.   By \cref{inner_degree_1}, $C'[N_{C'}[v]]$ is also isomorphic to $K_4$.  Let $x$ and $y$ be the two neighbours of $v$ on the outer face of $C'$. Since $\deg^+_H(v)=2$, $\deg_{C'}^+(v)=1$, so the cycle $vxw'y$ has no vertices of $H$ in its interior.
    We claim that the triangle $yw'x$ also contains no vertices of $H$ in its interior. Indeed, if $yw'x$ contains more than one inner vertex then $\deg^+_H(x)>2$ or $\deg^+_H(y)>2$, which is ruled out by the statement of the lemma.  If $yw'x$ contains exactly one inner vertex then $\deg_{H-B(H)}(w')=1$, which is ruled out by the statement of the lemma.

    Let $H'$ be a dom-minimal graph that dom-respects $H-v$. Then $w$ and $w'$ are both in $B(H-v)$, so $\deg^+_{H-v}(w)=\deg^+_{H-v}(w')=0$. Therefore, neither $w$ nor $w'$ are in $H'$. Therefore $|H'|\le |H-\{v,w,w'\}|=|H|-3$.  Also $B(H-v)=B(H)\setminus\{v\}\cup\{w,w'\}$ but none of $v$, $w$, or $w'$ are in $H'$.  Therefore, by \cref{boundary_subset}, $B(H')\subseteq B(H-v)\setminus\{v,w,w'\}$ so $|B(H')|\le |B(H)|-1$.

    \item $\deg^+_H(v)=1$ for all $v\in N_H(w)$.  In this case let $v$ be any vertex in $N_H(w)$ and let $H'$ be a dom-minimal graph that dom-respects $H-v$.  Then $w\in B(H-v)$, so $\deg^+_{H-v}(v')=0$ for all $v'\in N_H(w)$. Since $H'$ is dom-minimal, $N_H(w)\cap V(H')=\emptyset$, by \cref{bad_vertex}.  Similarly $w\not\in V(H')$, also by \cref{bad_vertex}. Therefore $|H'|\le |H-(\{w\}\cup N_H(w))| \le |H|-4$. Then $B(H-v)=B(H)\setminus\{v\}\cup\{w\}$.
    By \cref{boundary_subset}, $B(H')\subseteq B(H-v)\setminus N_H[w]$, so $|B(H')|\le |B(H)|-\deg_H(w) \le |B(H)|-3$. \qedhere
  \end{compactenum}
\end{proof}



\subsection{$2$-Critical Graphs}


We say that a generalized near-triangulation $H$ is \defin{$2$-critical} if
\begin{compactenum}[({2-C}1)]
  \item $\deg^+_H(v)\le 2$ for each $v\in B(H)$; \label[tc]{inner_degree_2}
  \item $H-B(H)$ is critical; \label[tc]{inner_critical}
  \item $\deg_{H-B(H)}(w)\ge 2$ for all $w\in V(H-B(H))$. \label[tc]{inner_vertex_degre_2}
\end{compactenum}
(See \cref{two_critical_figure}.) We will work our way up to a proof of the following lemma, which allows our algorithm to handle $2$-critical graphs directly, in one step:

\begin{figure}
  \centering
  \includegraphics[page=1]{figs/two_critical}
  \caption{A $2$-critical generalized near-triangulation.}
  \label{two_critical_figure}
\end{figure}


\begin{lem}\label{two_critical_handler}
  Let $H$ be a $2$-critical generalized near-triangulation.  Then there exists $X\subseteq V(H)$ of size at most $(2|B(H-B(H))| + I(H-B(H)))/3$ that dominates $I(H)$ and such that each component of $H[X]$ contains at least one vertex in $B(H)$.
\end{lem}


\begin{lem}
  Let $H$ be a dom-minimal $2$-critical generalized near-triangulation.  Then $\deg^+H(v)=2$ for all $v\in B(H)$.
\end{lem}

\begin{proof}
  Consider some $v\in B(H)$. By \cref{bad_vertex}, $\deg^+_H(v)\ge 1$. Assume for the sake of contradiction that $\deg_H(v)=1$. By \cref{inner_degree_1}, $H[N_H[v]]$ is isomorphic to $K_4$. Let $x$ and $y$ be the neighbours of $v$ on the outer face of $H$ and let $w$ be the inner neighbour of $v$. Since $\deg^+_H(v)=1$ then the cycle $vxwy$ has no vertices of $H$ in its interior.  Since $H$ is $2$-critical, $\deg_{H-B(H)}(w)\ge 2$, which implies that $w$ has at least two neighbours in the interior of the cycle $ywx$.  But this implies that at least one of $x$ or $y$, say $x$, has at least two neighbours in the interior of $ywx$.  But this implies that $\deg^+_H(x)\ge 3$, which contradicts the fact that $H$ is $2$-critical.
\end{proof}


\begin{lem}
  Let $H$ be a dom-minimal $2$-critical generalized near-triangulation.  Then $|B(H)|\ge |B(H-B(H))|$.
\end{lem}

\begin{proof}
  Let $\mathcal{C}$ be the set of components of $H-B(H)$.
  Let $C$ be a component in $\mathcal{C}$ and let $W:=w_0,\ldots,w_k$ be the walk around the outer face of $C$, so that $w_0=w_k$.  Then, for each $i\in\{1,\ldots,k\}$, $H$ contains an inner face $w_{i-1}w_iv_i$ that is to the left of the edge $w_{i-1}w_i$ when traversed from $w_{i-1}$ to $w_i$.  Since $w_i\in I(H)$, $v_i\in B(H)$.  Since $H$ is $2$-critical and does not contain parallel edges, $v_i\neq v_j$ for any $i\neq j$. Let $N_2(C):=\{v_1,\ldots,v_k\}$.  Therefore $|N_2(C)|= k\ge |B(C)|$. Since $H$ is $2$-critical, $N_2(C)\cap N_2(C')=\emptyset$ for any distinct components $C,C'\in\mathcal{C}$.  Therefore $|B(H)|\ge \sum_{C\in\mathcal{C}} |N_2(C)|\ge \sum_{C\in\mathcal{C}} |B(C)|=|B(H-B(H))|$.
\end{proof}

For each integer $r\ge 3$, the \defin{$r$-wheel} $W_r$ is the near-triangulation whose outer face is bounded by a cycle $v_0,\ldots,v_{r-1}$ that contains a single vertex $x$ in its interior and that is adjacent to each of $v_0,\ldots,v_{r-1}$.  For even values of $r$, $W_r$ is called an \defin{even wheel}. Note that the following lemma, illustrated in \cref{biconnected_critical_colouring} is about critical graphs, not $2$-critical graphs.

\begin{figure}
  \centering
  \includegraphics[page=7,trim={0 55 0 10},clip]{figs/two_critical}
  \caption{\cref{biconnected_critical}: Partitioning the vertices of a biconnected critical graph into three dominating sets.}
  \label{biconnected_critical_colouring}
\end{figure}

\begin{lem}\label{biconnected_critical}
  Let $H$ be a biconnected critical generalized near-triangulation with at least $3$ vertices and not isomorphic to $W_k$ for any even integer $k$.  Then there exists a partition $\{X_0,X_1,X_2\}$ of $V(H)$ such that
  \begin{compactenum}[(i)]
    \item For each edge $vw$ of $H[B(H)]$, $v\in X_i$ and $w\in X_j$ for some $i\neq j$;\label[p]{proper}
    \item for each $i\in\{0,1,2\}$, $X_i$ dominates $H$. \label[p]{dominates_h}
  \end{compactenum}
\end{lem}

\begin{proof}
  If $H$ is isomorphic to $W_k$ for some odd integer $k\ge 3$, then we take $X_0:=\{v_0, x\}$, $X_1:=\{v_{2i-1}:i\in\{1,\ldots,\lfloor k/2\rfloor\}$, and $X_2:=\{v_{2i}:i\in\{1,\ldots,\lfloor k/2\rfloor\}$.  It is straightforward to verify that these sets satisfy \cref{proper,dominates_h}.  (The fact that $k$ is odd ensures that $v_0$ has a neighbour $v_1\in X_1$ and $v_{k-1}\in X_2$, which ensures \cref{dominates_h}---this is not true for even $k$.)   We now assume that $H$ is not isomorphic to $W_k$ for any integer $k$.  By \cref{critical_structure}, this implies that $H$ is outerplanar or that $H[B(H)]$ has at least two inner faces.

  We now proceed by induction on $|I(H)|$.  If $|I(H)|=0$ then $H$ is an edge-maximal outerplanar graph, and therefore has a proper $3$-colouring.  We take $X_0$, $X_1$, and $X_2$ to be the three colour classes in this colouring.  This choice clearly satisfies \cref{proper}. Since each vertex of $H$ is included in at least one triangle, each vertex of $H$ is dominated by each of $X_0$, $X_1$, and $X_2$, so this choice satisfies \cref{dominates_h}.

  \begin{figure}
    \centering
    \begin{tabular}{ccc}
      \includegraphics[page=1]{figs/biconnected} &
      \includegraphics[page=2]{figs/biconnected} &
      \includegraphics[page=3]{figs/biconnected}
    \end{tabular}
    \caption{The proof of \cref{biconnected_critical}}
    \label{contraction_proof}
  \end{figure}
  If $|I(H)|\ge 1$ then $H$ is not outerplanar.  See \cref{contraction_proof}.  Since $H$ is not isomorphic to $W_k$ for any integer $k$, $H[B(H)]$ contains at least two inner faces.  Let $x$ be an inner vertex of $H$ and let $f$ be the marked face of $H[B(H)]$ that contains $x$.
  % If $f$ has only three vertices, then we apply induction on $H':=H-x$ to obtain sets $X_0'$, $X_1'$, and $X_2'$.  We then set $X_0:=X_0'\cup\{x\}$, $X_1:=X_1'$ and $X_2:=X_2'$.  These sets satisfy \cref{proper} because every edge of $H[B(H)]$ is also an edge of $H'[B(H')]$.  These sets satisfy \cref{colourful} because $f$ is a clique of size $3$ in $H'$ so, by \cref{proper}, each $X_i$ contains at least one vertex of $f$. Now, \cref{colourful} also implies that each $X_i$ dominates $x$ and (by induction) dominates $V(H)\setminus\{x\}$, so each $X_i$ is a dominating set.
  Since $H[B(H)]$ has at least two inner faces and $H$ is biconnected, $f$ contains an edge $vw$ that is on the boundary of two faces of $H$.  Let $H'$ be the graph obtained by contracting the edge $vx$ into $v$. Then $H'[B(H')]=H[B(H)]$ and $H'$ is a biconnected critical generalized near-triangulation so we apply induction to obtain sets $X_0'$, $X_1'$ and $X_2'$.  Without loss of generality, we can assume that $v$ is in $X_1'$.  Then we set $X_0:=X_0'$, $X_1:=X_1'\cup\{x\}$ and $X_2:=X_2'$. Since $H'[B(H')]=H[B(H)]$ this clearly satisfies \cref{proper}.  Since $H'$ contains the edge $vw$ for each $w\in V(f)\setminus\{v\}$, \cref{proper} implies that the vertices of the path $f-v$ are alternately contained in $X_2$ and $X_0$.
  % In particular, since $f$ has at least three vertices these sets satisfy \cref{colourful}.

  All that remains is to show that $X_0$, $X_1$, and $X_2$ satisfy \cref{dominates_h}.  The inductive hypothesis already implies that each of these sets dominates $V(H)\setminus V(f)$.  Since $x$ is adjacent to every vertex of $f$, it is adjacent to at least one vertex of $X_0$ at least one vertex of $X_2$.  Therefore, each of $X_0$, $X_1$, and $X_2$ dominates $x$.  For each vertex $w\in V(f)\setminus\{v\}$, $w$ is adjacent to $x\in X_1$, $w\in X_{i}$ for some $i\in\{0,2\}$ and $w$ is adjacent a neighbour $w'\in X_{2-i}$ in $f$, so each of these sets dominates $w$.  Finally, since the vertex $v$ is incident to a chord of $H[B(H)]$, it is incident to a second face $f'\neq f$ of $H[B(H)]$.  Since $f$ is marked and $H$ is critical, $f'$ is not marked.  Therefore $f'$ is a triangle with one vertex in each of $X_0$, $X_1$, and $X_2$. Therefore each of these sets dominates $v$.
\end{proof}

The following lemma, illustrated in \cref{even_wheel}
 explains how we deal with even wheels not covered by \cref{biconnected_critical}:

\begin{lem}\label{wheelie}
  Let $H:=W_k$ for some even integer $k\ge 4$ and let $v$ be any vertex in $B(H)$.  Then there exists a partition $\{X_0,X_1,X_2\}$ of $V(H)$ such that
  \begin{compactenum}[(i)]
    \item For each edge $vw$ of $H[B(H)]$, $v\in X_i$ and $w\in X_j$ for some $i\neq j$;\label[p]{proper2}
    % \item For each inner face $f$ of $H[B(H)]$ and each $i\in\{0,1,2\}$, $V(f)\cap X_i\neq\emptyset$; and \label[p]{colourful}
    % that contains $x_f\in I(H)$ in its interior, exactly one vertex $v_f$ of $f$ is in the set $X_i$ that contains $x$ and $v_f$ is incident to a chord of $H[B(H)]$. \pat{Define chord.}
    \item $X_0$ dominates $V(H)\setminus\{v\}$ and $X_1$ and $X_2$ each dominate $H$. \label[p]{weak_dominates_h}
  \end{compactenum}
\end{lem}

\begin{figure}
  \centering
  \includegraphics{figs/even_wheel}
  \caption{\cref{wheelie}: Partitioning the vertices of an even wheel into sets $X_0$, $X_1$, and $X_2$.}
  \label{even_wheel}
\end{figure}
\begin{proof}
  Label the vertices of $W_k$ as $v_0,\ldots,v_{k-1}$ so that $v=v_0$.  Then the sets $X_1:=\{v_0, x\}$, $X_2:=\{v_{2i-1}:i\in\{1,\ldots,k/2\}\}$, and $X_0:=\{v_{2i}:i\in\{1,\ldots, k/2-1\}\}$ satisfy the requirements of the lemma.
\end{proof}


The following lemma, illustrated in \cref{critical_colouring}, drops the requirement that the critical graph be biconnected and applies even if some of the biconnected components of $H$ are even wheels.

\begin{figure}
  \centering
  \includegraphics[page=2]{figs/two_critical}
  \caption{\cref{critical}: Partitioning the vertices of a connected critical graph into dominating sets $X_0$, $X_1$, and $X_2$.}
  \label{critical_colouring}
\end{figure}

\begin{lem}\label{critical}
  Let $H$ be a connected critical generalized near-triangulation with at least $3$ vertices, no vertices of degree $1$ and not isomorphic to $W_k$ for any even integer $k$.  Then there exists a partition $\{X_0,X_1,X_2\}$ of $V(H)$ such that
  \begin{compactenum}[(i)]
    \item for each edge $vw$ of $H[B(H)]$, $v\in X_i$ and $w\in X_j$ for some $i\neq j$;\label[p]{proper_2}
    % \item for each marked face $f$ of $H[B(H)]$ and each $i\in\{0,1,2\}$, $V(f)\cap X_i\neq\emptyset$; and \label[p]{colourful_2}
    % that contains $x_f\in I(H)$ in its interior, exactly one vertex $v_f$ of $f$ is in the set $X_i$ that contains $x$ and $v_f$ is incident to a chord of $H[B(H)]$. \pat{Define chord.}
    \item for each $i\in\{0,1,2\}$, $X_i$ dominates $H$,  \label[p]{dominates_h_minus_l}
  \end{compactenum}
\end{lem}

\begin{proof}
  The proof is by induction on $|H|$.  First, suppose $|H|=3$. Since $H$ is connected and has no vertices of degree $1$ then $H$ is a triangle $v_0v_1v_2$. We take $X_i:=\{v_i\}$ for each $i\in\{0,1,2\}$.  Clearly these sets satisfy the requirements of the lemma.

  % If $L\neq\emptyset$ then let $z$ be a vertex in $L$ and let $z_r,\ldots,z_0$ the maximal path in $H$ such that $z=z_r$ and each of $z_1,\ldots,z_{r-1}$ has degree $2$ in $H$. Consider the graph $H':=H-\{z_1,\ldots,z_{r}\}$.
  % \begin{enumerate}
  %   \item If $z_0\in L$, then $H'$ is the graph that contains only $z_0$ and $H=z_r,\ldots,z_0$ is a path.  In this case $L:=\{z_0,z_r\}$. We take $X_i:=\{z_j:j\equiv i\pmod 3\}$ for each $i\in\{0,1,2\}$, $L_2=\{z_0\}$.  Starting with $L_0:=L_1:=\emptyset$ and $L_2:=\{z_0\}$, we then set
  %   $L_{(r-2)\bmod 3}\gets L_{(r-2)\bmod 3} \cup \{z_r\}$.  This produces sets that satisfy the requirements of the lemma, and we are done.
  %
  %   \item If $H'$ is isomorphic to $W_k$ for some even integer $k$ then we apply \cref{wheelie} with $v=z_0$ to obtain sets $X_0'$, $X_1'$, $X_2'$, $L_0'$, $L_1'$, and $L_2'$.
  %
  %   \item Otherwise, we apply the inductive hypothesis on $H'$ to obtain sets $X_0'$, $X_1'$, $X_2'$, $L_0'$, $L_1'$, and $L_2'$.
  % \end{enumerate}
  % \begin{figure}
  %   \centering
  %   \includegraphics{figs/tail}
  %   \caption{Dealing with a degree-$1$ vertex, $z_r$, in the proof of \cref{critical}.}
  %   \label{critical_degree_one}
  % \end{figure}
  % See \cref{critical_degree_one}.
  % In the latter two cases, we may assume, without loss of generality (by renaming colours), that $z_0\in X_{1}'$.  Since $z_0\not\in L$, $\deg_H(z_0)\ge 3$ so $\deg_{H'}(z_0)\ge 2$.  Therefore we may assume, without loss of generality, that $X_1'$ and $X_2'$ each dominate $z_0$. Redefine $L_{(2-r)\bmod 3}'\gets L_{(2-r)\bmod 3}'\cup\{z_r\}$ so that $L_0'$, $L_1'$ and $L_2'$ is a partitition of $L$.  Then we take $X_i:=X_{i}'\cup\{z_j:1-j\equiv i\pmod 3\}$, $L_0:=L_0'$, $L_1:= L_1'$, and $L_2:=L_2'$.  These sets satisfy the requirements of the lemma.  Indeed, the only concern would be that $z_0$ is not dominated by $X_0$ because $z_0$ is not necessarily dominated by $X_0'$, but this does not occur because $X_0$ includes $z_1$.

  % Finally, we arrive at the case in which $L=\emptyset$.
  If $H$ is biconnected then, since $H$ is not an even wheel, we can immediately apply \cref{biconnected_critical} and we are done.  Otherwise, $H$ contains a cut vertex $v$ that separates $H$ into components $C_1,\ldots,C_k$ and such that $H':=H[V(C_1)\cup\{v\}]$ is biconnected. Refer to \cref{critical_3_colouring}. Since $L=\emptyset$, $H'$ has at least three vertices. If $H'$ is isomorphic to $W_k$ for some even integer $k$ then we apply \cref{wheelie} to $H'$ and $v$ to obtain sets $X_0'$, $X_1'$, and $X_2'$. Otherwise, we apply \cref{biconnected_critical} to $H'$ to obtain sets $X_0'$, $X_1'$, and $X_2'$.  In either case we may assume, without loss of generality that $v\in X_1'$, that $X_1'$ and $X_2'$ each dominate $H'$ and that $X_0'$ dominates $V(H')\setminus\{v\}$.

  \begin{figure}[htbp]
    \centering
    \begin{tabular}{cc}
      \includegraphics[page=1,trim={10 0 10 0},clip]{figs/critical_3_colouring} &
      \includegraphics[page=2]{figs/critical_3_colouring}
    \end{tabular}
    \caption{Two cases in the proof of \cref{critical}.}
    \label{critical_3_colouring}
  \end{figure}

  Let $H'':=H-V(C_1)$. First, suppose that $\deg_{H''}(v)>1$.
  If $H''$ is isomorphic to $W_k$ for some even integer $k$ then we apply \cref{wheelie} to $H''$ and $v$ to obtain sets $X_0''$, $X_1''$, $X_2''$. Otherwise, we apply the inductive hypothesis to $H''$ to obtain sets $X_0''$, $X_1''$, $X_2''$ that each dominate $H''$. In either case we may assume, without loss of generality (by renaming) that $v\in X_1''$, that $X_1''$ and $X_2''$ each dominate $H'$ and that $X_0''$ dominates $V(H'')\setminus\{v\}$.  Then the sets $X_0:=X_0'\cup X_2''$, $X_1:=X_1'\cup X_1''$ and $X_2:=X_2'\cup X_0''$ satisfy the requirements of the lemma.  (The only concern is whether each set dominates $v$, but this is guaranteed by the fact that $v\in X_1$, and that $X_2'\subseteq X_2$ and $X_2''\subseteq X_0$ each dominate $v$.)

  Finally, if $\deg_{H''}(v)=1$ then we consider the maximal path $v,v_1,v_2,\ldots,v_{r-1},v_r$ such that $\deg_{H''}(v_i)=2$ for each $i\in\{1,\ldots,r-1\}$.  Let $H''':=H''-\{v,v_1,\ldots,v_{r-1}\}$ and we treat $H'''$ exactly as we treated $H''$ in the previous paragraph to obtain sets $X_0'''$, $X_1'''$ and $X_2'''$.  Without loss of generality, we assume that $v_r\in X_{(r-1)\bmod 3}$, that $X_{(r-1)\bmod 3}$ and $X_{(r\bmod 3)}$ each dominate $H'''$ and that $X_{(r-2)\bmod 3}$ dominates  $V(H''')\setminus\{v_r\}$. Let $X_0'':=X_0'''\cup\{v_i:i\equiv 2\pmod 3\}$, $X_1'':=X_1'''\cup\{v\}\cup\{v_i:i\equiv 0\pmod 3\}$, and  $X_2'':=X_2'''\cup\{v_i:i\equiv 1\pmod 3\}$.  Then $v\in X_1''$, $X_1''$ and $X_2''$ each dominate $H''$, and $X_0''$ dominates $V(H'')\setminus\{v\}$.  We can now define the sets $X_0$, $X_1$, and $X_2$ exactly as we did in the previous paragraph.
\end{proof}


At last, the following lemma, illustrated in \cref{two_critical_colouring}, shows how we combine everything to find three dominating sets of a $2$-critical graph $H$ whose total size is at most $2|B(H)| + I(H)$.

\begin{figure}
  \centering
  \includegraphics[page=4]{figs/two_critical}
  \caption{\cref{three_sets_coverage}: Finding three sets $X_0$, $X_1$, and $X_2$ that dominate $I(H)$ in a $2$-critical graph $H$.}
  \label{two_critical_colouring}
\end{figure}


\begin{lem}\label{three_sets_coverage}
  Let $H$ be a $2$-critical generalized near-triangulation.  Then there exists $X_0,X_1,X_2\subseteq V(H)$ such that
  \begin{compactenum}[(i)]
    \item $|X_0|+|X_1|+|X_2| \le 2|B(H-B(H))|+|I(H-B(H))|$; \label[p]{total_size}
    \item for each $i\in\{0,1,2\}$, $X_i$ dominates $I(H)$ in $H$; and \label[p]{dominates_i}
    \item for each $i\in\{0,1,2\}$, each component of $H[X_i]$ contains at least one vertex in $B(H)$. \label[p]{connectivity}
  \end{compactenum}
\end{lem}

\begin{proof}
  % Refer to \cref{covers}.
  Let $\mathcal{C}$ be the set of components of $H-B(H)$
  and let $\mathcal{C}_{\boxtimes}$ be the set of components in $\mathcal{C}$  that are even wheels.

  For each component $C$ in $\mathcal{C}_{\boxtimes}$ we choose the vertex $x$ that dominates $C$, some vertex $w$ in $B(C)$ and some vertex $v\in B(H)$ adjacent to $w$.  We add $\{v,w,x\}$ to each of $X_0$, $X_1$, and $X_2$. The vertex $x$ ensures that each $X_i$ dominates $C$ and the vertices $v$ and $w$ ensure that the component of $H[X_i]$ that contains $x$ contains at least one vertex in $B(H)$.  Doing this for every component in $\mathcal{C}_{\boxtimes}$ contributes a total of $9|\mathcal{C}_{\boxplus}|$ vertices to $X_0$, $X_1$, and $X_2$. On the other hand, $|B(C)|\ge 4$ and $|I(C)|\ge 1$ for each $C\in \mathcal{C}_{\boxplus}$, so $\sum_{C\in \mathcal{C}_{\boxplus}} (2|B(C)| + I(C))\ge (2\cdot 4+1)|\mathcal{C}_{\boxplus}| = 9|\mathcal{C}_{\boxplus}|$.

  For each component $C$ in $\mathcal{C}\setminus(\mathcal{C}_{\boxplus}\cup\mathcal{C}_1)$, we apply \cref{critical} to obtain sets $X_0'$, $X_1'$, $X_2'$. For each $i\in\{0,1,2\}$ and each $w\in X_i'\cap B(H-B(H))$ we choose a vertex $v\in B(H)$ adjacent to $w$ and add both $v$ and $w$ to $X_i$. \Cref{critical} ensures that each $X_i$ dominates $C$ and the vertex $v$ ensures that the component of $X_i$ that contains $w$ contains at least one vertex of $B(H)$.
  Doing this for each component $\mathcal{C}\setminus\mathcal{C}_\boxplus$ contributes a total of at most  $\sum_{C\in\mathcal{C}\setminus(\mathcal{C}_{\boxplus}\cup\mathcal{C}_1)}(|C|+|B(C)|)=\sum_{C\in\mathcal{C}\setminus(\mathcal{C}_{\boxplus}\cup\mathcal{C}_1)}(2|B(C)|+|I(C)|)$ to $X_0$, $X_1$, and $X_2$.

  The resulting sets $X_1$, $X_2$, and $X_3$ each dominate $\bigcup_{C\in\mathcal{C}} V(C)=I(H)$ and have total size at most $\sum_{C\in\mathcal{C}} (2|B(C)|+|I(C)|) = 2|B(H-B(H))| + |I(H-B(H))|$.
\end{proof}

\begin{proof}[Proof of \cref{two_critical_handler}]
  Take $X$ to be the smallest of the three sets $X_0$, $X_1$, and $X_2$ guaranteed by \cref{three_sets_coverage}.
\end{proof}


\subsection{The Algorithm}

All of this has been leading up to a variant  $\textsc{SimpleGreedy}(G)$ that we call $\textsc{BetterGreedy}(G)$.  Suppose we have already chosen $\Delta_0,\ldots,\Delta_{i-1}$ for some $i\ge 0$ and we now want to choose $\Delta_i$.  Let $X_i:=\bigcup_{j=0}^{i-1}\Delta_j$, let $G_i$ be a dom-preserving subminor of $G-X_i$ that is dom-minimal, and let $v_i$ be a vertex in $B(G_i)$ that maximizes $\deg^+_{G_i}(v_i)$.  During iteration $i\ge 0$, there are now more cases to consider:
\begin{compactenum}[{[}bg1{]}]
    \item If $\deg^+_{G_i}(v_i)\ge 3$ then we set $\Delta_i\gets\{v_i\}$.
    \label[bg]{bg_high_degree}
    \item If $G_i-B(H_i)$ contains a vertex of degree $1$ we set $\Delta_i:=\{v_i\}$ where $v_i$ is the vertex $v$ guaranteed by \cref{leaf_killer}.
    \item If $G_i-B(H_i)$ contains a vertex of degree $0$ we set $\Delta_i:=\{v_i\}$ where $v_i$ is the vertex $v$ guaranteed by \cref{degree_zero_killer}.
    \item If there exists distinct $u,v\in B(G_i)$ and $w\in B(G_i-B(G_i))$ such that $\deg^+_{G_i}(v)=2$, $\deg^+_{G_i-v}(w)\ge 3$, and $N^+_{G_i}(u)\subseteq N_{G_i}(w)$ then set $\Delta_i:=\{v,w\}$.
    \label[bg]{bg_two_three}
    \item Otherwise, $G_i$ is $2$-critical and $i+1=r$.  By \cref{two_critical_handler}, there exists $\Delta_{r-1}\subseteq V(G_i)$ of size at most $|B(G_i-B(G_i))|/3 + |I(G_i-B(G_i))|$ that dominates $I(G_i)$.
    \label[bg]{bg_two_critical}
\end{compactenum}


\begin{proof}
  By \cref{leaf_killer,degree_zero_killer,really_good} during each of the first $r-1$ steps, one of the following occurs:
  \begin{compactitem}
    \item[$x_t$:] For some $t\ge 3$, we can add a single vertex $v_i$ that increases the size of the dominated set $D_{i+1}:=N[X_{i+1}]$ by $t$ and increases the size of the boundary set $B_{i+1}:=N_G(I(G-D_{i+1}))$ by at most $t-1$.
    \item[$a$:] We can add a vertex $v_i$ that increases the size of the dominated set $D_{i+1}$ by $2$ and \emph{decreases} the size of the boundary set $B_{i+1}$ by at least $1$.
    \item[$b$:] We can add a vertex $v_i$ that increases the size of the dominated set $D_{i+1}$ by $1$ and \emph{decreases} the size of the boundary set $B_{i+1}$ by at least $3$.
    \item[$c$:] We can add a pair of vertices $\{v_i,w_i\}$ that increase the size of the dominated set $D_{i+1}$ by at least $5$ and increases the size of the boundary set $B_{i+1}$ by at most $2$.
    \item[$\bullet$:] We can directly complete the connected dominating set $X_r=X_{i+1}$ by adding a set $\Delta_{r-1}=\Delta_i$ of at most $(2|B(G_{i}-B(G_i))|+|I(G_i-B(G_i))|)/3$ additional vertices where, as before $G_i:=G[B_i\cup (V(G)\setminus D_i)]$ and $r=i+1$.
  \end{compactitem}

  Let $a$, $b$, $c$, and $\langle x_t\rangle_{t\ge 3}$ denote the number of times each of these cases occurs in the first $r-1$ steps, and let $D:=D_{r-1}$, $B:=B_{r-1}$, and $X:=X_{r-1}$.  Then,
  \begin{align}
    |D| & \ge 3 + \sum_{t\ge 3}tx_t + 2a + b + 5c \label{dd_size} \\
    |B| & \le 3 + \sum_{t\ge 3}(t-1)x_t - a - 3b + 2c \enspace \label{bb_size} \\
    |X| & \le \sum_{t\ge 3}x_t + a + b + 2c \label{x_size} \enspace .
  \end{align}

  Let $R:=B(G_{r-1}-B(G_{r-1}))$ and $S:=I(G_{r-1}-B(G_{r-1}))$.  Since $\{D,R,S\}$ is a partition of $V(G)$,
  \begin{equation}
    |D|+|R|+|S|= n \enspace . \label{ttotal_size}
  \end{equation}
  As part of this section, we will show that $|B_{r-1}|\ge |B(G_i-B(G_i))|$, i.e., $|B|\ge |R|$.

  Putting everything together we get the constraints:
  \begin{align}
        3 + \sum_{t\ge 3}tx_t + 2a + b + 5c + |R| + |S| \le n&
          & \text{(by \cref{dd_size} and \cref{ttotal_size})}  \label{a} \\
        3 + \sum_{t\ge 3}(t-1)x_t - a - 3b + 2c \ge |R|
          & & \text{(by \cref{bb_size} and since $|B|\ge |R|$)} \label{b} \\
  \end{align}
  with all values non-negative.  The size of the final connected dominating set $X_r$ is then at most
  \begin{equation}
    |X_r| \le \sum_{t\ge 3}x_t + a + b + 2c + 2|R|/3 + |S|/3  \enspace . \label{oobjective}
  \end{equation}

  \begin{clm}\label{simplify}
    If $(a,b,c,|R|,|S|,x_3,x_4,\ldots)$ are non-negative and satisfy \cref{a,b}, then setting $x_3\gets x_3+\sum_{t\ge 4}tx_t/2$ and $x_t\gets 0$ for all $t\ge 4$ also satisfy \cref{a,b} and do not decrease \cref{oobjective}.
  \end{clm}

  \begin{clmproof}
    Suppose $x_t>0$ for some integer $t\ge 4$, otherwise there is nothing to prove.  Let $i:=\min\{t\ge 4: x_t>0\}$ and set $x_3\gets x_3+(t-1)x_t/3$ and $x_t\gets 0$. This change causes the left-hand-side of \cref{a} to decrease by $x_t$. This change does not affect the left-hand-side of \cref{b}.  This change increases the value of \cref{oobjective} by $(t-1)x_t/3-x_t\ge 0$.
  \end{clmproof}

  By \cref{simplify}, maximizing \cref{oobjective} subject to the constraints given by \cref{a,b} is a linear program in six variables $(x_3,a,b,c,|R|,|S|)$ which can be done easily.  The maximum is achieved when $x_3=a=b=c=|S|=0$, $c=(n-6)/7$ and $|R|=(2n+9)/7$, at which point \cref{oobjective} evaluates to $(10n-18)/21$.
\end{proof}


%%%%%%%% The application section %%%%

\section{Proof of \cref{one_bend_collinear}}
\label{one_bend}
This section demonstrates an application of a connected dominating set in graph drawing. We establish that each planar graph with a small connected dominating set has a 1-bend drawing with a large collinear set. We start with introducing a topological equivalent of 1-bend collinear sets as in $\cite{DBLP:journals/jocg/LozzoDFMR18}$.


% We are interested in finding bounds on the size of $h$-bend collinear set. For $0$-bend collinear set, constructions of non-Hamiltonian cubic triconnected planar graphs \cite{DBLP:journals/jct/GrunbaumW73,DBLP:conf/wg/RavskyV11,DBLP:journals/jocg/LozzoDFMR18} imply that planar triangulation has a 0-bend collinear set of size at most $O(n^{\sigma})$ where $\sigma < 0.986$. Moreover, every planar triangulation has a 0-bend collinear set of size at least $\Omega(\sqrt{n})$ \cite{DBLP:journals/dcg/BoseDHLMW09} \cite{DBLP:journals/jgaa/Dujmovic17}. In a recent work, Dujmovic and Morin \cite{DBLP:conf/compgeom/DujmovicM19} show that every planar triangulation with maximum degree $\Delta$ has $0$-bend collinear set of size at least $\Omega(\frac{n^{0.8}}{\Delta^4})$.

% For the $1$-bend drawing of planar triangulation with $n$ vertices,  Everett et al. \cite{DBLP:conf/gd/EverettLLW07} show the existence of a set $\mathcal{U}$ of $n$ distinct points in the plane such that every $n$-vertex planar graph admits an embedding on vertex set $\mathcal{U}$ with at most one bend along each edge. Furthermore, Giacomo et al. \cite{DBLP:journals/comgeo/GiacomoDLW05} show that for a linear ordering $L$ of vertices of a planar triangulation $G$ and strictly convex curve $\lambda$, there is a $1$-bend planar drawing of $G$ such that vertices of $G$ appear on $\lambda$ with the same order in $L$.

% With further relaxation on the drawing of the edges, de Fraysseix et al. \cite{DBLP:journals/combinatorica/FraysseixPP90} show that any set of $n$ points in the plane is a universal set for the two-bend drawing of planar graphs. Furthermore, They show that the planar embedding problem of any $n$-vertex graph onto an $n$ collinear points in $\mathbb{R}^2$ with at most one bend along each edge is NP-complete.

\subsection{Characterisation of 1-Bend Collinear Sets}
A curve $C$ is a continuous mapping from $[0, 1]$ to $\mathbb{R}^2$. We usually call $C(0)$ and $C(1)$ the endpoints of $C$. If these two endpoints coincide, the curve is closed. Otherwise, it is open. A curve $C$ is called simple if $C$ is $C(x) \neq C(y)$ for all $x \neq y$ with exception for the endpoints of $C$ i.e $\{0, 1\}$. $C$ is a \defin{Jordan Curve} if it is simple and closed.

Let $G$ be plane graph, a Jordan curve $C$ is a \defin{$k$-proper good curve} if it contains a point in the interior of some face of $G$ (\textit{good}), and the intersection between $C$ and each edge $e$ of $G$ is empty, or at most $k$ points, or the entire edge $e$ (\textit{proper}).

Da Lozzo et al. \cite{DBLP:journals/jocg/LozzoDFMR18} characterize the collinear sets in the straight line drawing of a planar graph using 1-proper good curves.

\begin{thm}[\cite{DBLP:journals/jocg/LozzoDFMR18} ] \label{straightline-topological}
Let $G$ be a plane graph. A set $S \subseteq V(G)$ is a collinear set if and only if there exists a $1$-proper good curve that contains $S$.
\end{thm}

Similarly, we bring a sufficient condition for $1$-bend collinear sets.

\begin{observation} \label{1-bend-topological}
    Let $G$ be a plane graph. A set $S \subseteq V(G)$ is a $1$-bend collinear set if $G$ has a $2$-proper good curve $C$ that contains $S$.
\end{observation}

\begin{proof}
Let $C$ be a 2-proper good curve that contains $S$. For each edge $e \in E(G)$ such that $|C \cap e| = 2$, we introduce a new subdivision vertex $u_e$ between the two intersection points of $C$ and $e$. By adding these new vertices, we obtain a planer drawing of a subdivision of $G$, denoted as $G^+$. Since every edge of $G^+$ is intersected by $C$ at most one, $C$ is $1$-proper good curve for $G^+$. Thus, by \cref{straightline-topological}, $S$ is collinear set for $G^+$. Note that a straight line drawing of $G^+$ is a $1$-bend drawing for $G$. Therefore, $S$ is a $1$-bend collinear set for $G$.
\end{proof}

\subsection{From a Spanning Tree to a 1-bend Collinear Set}

We prove that the leaves of a spanning tree of a planar graph induce a 1-bend collinear set. Precisely, we prove the following theorem.

\begin{lem} \label{spanning_tree_to_collinear_set}

Let $G$ be a planar graph and $T$ be a spanning tree of $G$. Then, the leaves of $T$ form a $1$-bend collinear set for $G$.

\end{lem}

\begin{proof}

    Let $\Gamma$ be a straight-line drawing of $G$.
    By \cref{1-bend-topological}, it is enough to introduce a 2-proper good curve $\ell$ on $\Gamma$ containing all the leaves of $T$. To navigate the curve $\ell$ on the drawing $\Gamma$, we construct an envelope around $\Gamma$ as follows. For each vertex $v \in V(G)$, we draw a small circle, $C_v$, centered at $v$. We make the radii of the circles small enough such that each vertex $v \in V(G)$, $C_v$ intersects only the edges incident to $v$ and it is disjoint from all the other circles that correspond to the other vertices. Moreover, for each edge $uv \in E(G)$, we draw two parallel segments on both sides of $uv$ with endpoints on the boundary of corresponding circles of $u$ and $v$. These parallel segments are close enough to the corresponding edges such that no two of them intersect. (see \cref{proof:main} \saeed{a figure is missing!}). Note that each edge $uv \in E(G)$ crosses the envelope exactly twice, once at $C_u$ and once at $C_v$.

    Assume $T$ is rooted at an arbitrary vertex of degree at least 2. We build the curve $\ell$ on the envelope of $\Gamma$ as follows. Starting from the root, we traverse the tree in \textit{depth first search} order. For each edge $uv \in E(T)$, we add the segment on the right side of the traversal direction of $uv$ into the curve $\ell$.

    For each leaf $u$ of $T$, let $v_u$ be its neighbor in $T$. To include the leaves of $T$ on the curve $\ell$, we join $u$ to the endpoint of segments around the edge $uv_u \in E(T)$ on $C_u$. To keep the curve $\ell$ closed, for each non-leaf vertex $u \in V(T)$, we append to $\ell$ the circular arcs from $C_u$ between the segments in $\ell$ in the order of the traversal. By the properties of the depth first traversal, $\ell$ is a closed curve. By construction, $\ell$ contains all the leaves of $T$ and all the other vertices of $T$ are inside $\ell$. Moreover, for each edge $uv \in E(G)$:

    \begin{enumerate}
        \item [(P1)] If $uv \in E(T)$ and neither $u$ nor $v$ is a leaf, then $|uv \cap \ell| = 0$,

        \item [(P2)] If $uv \in E(T)$ and either $u$ or $v$ is a leaf of $T$, then $|uv \cap \ell| = 1$, and

        \item [(P3)] If $uv \notin E(T)$, then $|uv \cap \ell| = 2$.
    \end{enumerate}

    Properties P1-P3 guarantee that $\ell$ is a 2-proper curve. Since the tree $T$ is not empty, $\ell$ on the circle of a vertex in $T$ touches a face of $\Gamma$. Therefore, $\ell$ is $2$-proper good curve and by \cref{1-bend-topological}, there exists a $1$-bend collinear set for $G$ formed by the leaves of $T$.
\end{proof}

Let $G$ be an $n$-vertex planar graph. \cref{main_result2} implies that $G$ has a spanning tree with at least $\frac{11}{21}n$ leaves. Using the maximum leaf spanning tree of $G$ in \cref{spanning_tree_to_collinear_set} establishes \cref{one_bend_collinear}.

% \begin{thm}
%     Let $G$ be a planar graph on $n$ vertices, then $G$ has a $1$-bend collinear set of size at least $n(1 - \frac{10}{21})$.
% \end{thm}

% \begin{proof}
% We first triangulate $G$ by adding edges. So, $G$ becomes an edge maximal planar graph. We apply Theorem 4 to obtain a dominating set $X$ of size at most $10n/21$. Next, we compute a spanning tree $T$ on $G[X]$. we append every vertex $v \in V(G)\setminus X$ to $T$ as leaf. This operation is possible because every $v \in V(G) \setminus X$ is dominated by $X$. Thus, $T$ has at least $n(1 - 10/21)$ leaves. By Theorem 6, we obtain the desired 1-bend collinear set.
% \end{proof}

\section{Discussion}

\Cref{two_critical_handler} is the best result we have right now for handling $2$-critical graphs.  This is a shame, because it is known that maximal outerplanar graphs have connected dominating sets of size at most $4n/11$. Unfortunately our critical graphs are not maximal outerplanar---they may not even be connected.  We can end up in a situation where we have a $2$-critical graph that is a collection of disjoint $6$-vertex graphs (a triangle inside a triangle), each of which requires $2$ vetices to dominate.  In this setting $Q=R=3k$ and we need $2k$ vertices in our dominating set, so the analysis is tight.

\hussein{One approach is : why we ended with a graph where all the boundary vertices has $deg^+(v) = 2$? I think before reaching such a graph we had a sequence of vertices $v_i,...,v_j$  such that $\sum_i deg^+(v_i) = |\text{the boundary of the current graph}|$ and $deg^+(v_i) \geq 3$.}

If we wanted to make use of the results for outerplanar graphs, then we would need to be way more careful in the way we choose vertices to delete.  In fact, I don't see any reasonable way to do this without using some variant of the canonical ordering (so that each intermediate graph $G_i$ is a near-triangulation).  Even if we could do that we'll still need something more because the process of making $G_i$ dom-minimal removes useless vertices and edges.

% because we could we reach a step where $G_{r-1}$ is a $2$-critical near-triangulation but $G_{r-1}-B(G_{r-1})$ has many components.  (Think of a serpentine triangulation where one out of every $4$ faces contains three vertices inside it.)

% \section{Moonshot}
%
% We say that a near-triangulation $H$ is \defin{dom-minimal} if
% \begin{compactenum}[({DM}1)]
%     \item each vertex $v\in B(H)$ has $\deg^+_H(v)\ge 1$; \label[dm]{minimum_degree}
%     \item if there exists edges $uv$ and $vw$ on the outer face of $H$ and  $\deg^+_H(v)\ge 1$ then $uw$ is an edge of $H$; and \label[dm]{one_vertex}
%     \item each edge $vw$ on the boundary of the outer face of $H$ is on the boundary an inner face $vxw$ of $H$ for some $x\in I(H)$. \label[dm]{bad_edge}
% \end{compactenum}
% We say that a generalized near-triangulation $H$ is \defin{dom-minimal} if each of its biconnected components are dom-minimal.
%
% \begin{obs}\label{bridgeless}
%     Any dom-minimal generalized near-triangulation $H$ is bridgeless.
% \end{obs}
%
% \begin{proof}
%    If $vw$ is a bridge in $H$ then both $v$ and $w$ are in $B(H)$.  Since $vw$ is a bridge in $H$, there is no path $vxw$ in $H$ and therefore no inner face $vwx$ in $H$. Thus $H$ does not satisfy \cref{bad_edge}.
% \end{proof}
%
% \begin{lem}\label{chord_incident}
%   let $H$ be a (biconnected) dom-minimal non-critical near-triangulation and let $v_0$ be a vertex in $B(H)$ with $|N_H(v_0)\cap B(H)|\ge 3$.  Then $\deg^+_H(v_0)\ge 2$.  In other words, if $v_0$ is incident to a chord of the outerplane graph $H[B(H)]$, then $v_0$ is incident to at least $2$ inner vertices of $H$.
% \end{lem}
%
% \begin{figure}[htbp]
%   \centering
%     \includegraphics{figs/chord_incident}
%
%   \caption{The proof of \cref{chord_incident}}
%   \label{chord_incident_fig}
% \end{figure}
% \begin{proof}
%   Refer to \cref{chord_incident_fig}
%   Since $H$ is a near-triangulation its outer face is bounded by a cycle $v_0,\ldots,v_{k-1}$.  Let $a:=\min\{i\in\{2,\ldots,k-2\}:v_0v_i\in E(H)\}$ and $b:=\max\{i\in\{2,\ldots,k-2\}:v_0v_i\in E(H)\}$. (Possibly $a=b$, but both $a$ and $b$ are well-defined since $|N^+_H(v_0)|\ge 3$.)   Since $H$ is dom-minimal, the edge $v_0v_1$ is on the boundary of an inner face $v_0v_1x$ of $H$ where $x$ is an inner vertex of $H$.  Since $H$ is dom-minimal, the edge $v_{k-1}v_0$ is on the boundary of an inner face $v_{k-1}v_0y$ of $H$ where $y$ is an inner vertex of $H$.  Then $x$ is in the interior of the face of $H[B(H)]$ bounded by the cycle $v_0,v_1,\ldots,v_a$ and $y$ is in the interior of the face of $H[B(H)]$ bounded by the cycle $v_0,v_b,\ldots,v_{k-1}$.  Therefore, $x\neq y$ and $N^+_H(v_0)\supseteq\{x,y\}$ so $\deg^+_H(v_0)\ge 2$.
% \end{proof}
%
%
%
% Let $K_5^-$ be the complete graph on five vertices with one edge removed.
%
% \begin{lem}
%   Let $H$ be a near-triangulation with $\deg^+_H(z)\le 2$ for all $z\in B(H)$. If $\deg^+_H(v)=1$ for some $v\in B(H)$, then $H$ is isomorphic to $K_4$ or $K_5^-$.
% \end{lem}
%
% \begin{proof}
%   Since $H$ is dom-minimal each vertex in $B(H)$ has inner-degree at least $1$, by \cref{minimum_degree}.  Therefore the outer face of $H$ is a cycle with at least three vertices.  Let $u$ and $w$ be the two neighbours of $v$ on the outer face.  Since $H$ is dom-minimal, $H$ contains an inner face $vxu$ with $x\in I(H)$, by \cref{bad_edge}.  Since $H$ is dom-minimal, $H$ contains an inner face $wyv$ with $y\in I(H)$.  Since $\deg^+_H(v)=1$, $x=y$.  Since $H$ is dom-minimal $uw$ is an edge of $H$, by \cref{one_vertex}.  Therefore $uvx
% \end{proof}
%
% A subgraph $H'$ of a generalized near-triangulation $H$ is \defin{dom-preserving} if
% % every outer-domatic set $X\subseteq V(H')$ of $H'$ is an outer-domatic set of $H$.
%
% \begin{compactenum}[({DP}1)]
%   \item $B(H')\subseteq B(H)$; \label[dp]{smaller_boundary}
%   \item $N^+_{H'}(v)=N^+_H(v)$ for all $v\in B(H')$; \label[dp]{interior_preserving}
%   \item $I(H')=I(H)$; and \label[dp]{inner_neighbourhood_preserving}
%   \item $N_{H'}(v)=N_H(v)\cap V(H')$ for all $v\in I(H)$. \label[dp]{outer_neighbour_preserving}
% \end{compactenum}
%
% \begin{obs}
%   Let $H$ be a generalized near-triangulation, let $H'$ be a dom-preserving subgraph of $H$, and let $\Delta$ be a subset of $V(H')$ that dominates $I(H')$ in $H'$.  Then $\Delta$ dominates $I(H)$ in $H$.
% \end{obs}
%
% % \begin{proof}
% %   Each vertex $v\in I(H')$ is adjacent to some vertex $w\in \Delta$.  Since $N_{H'}(v)=N_H(v)$, $w\in\Delta\cap V(H')$, so $v$ is dominated by $\Delta\cap V(H')$.  Since this is true for each $v\in I(H)=I(H')$, $\Delta\cap V(H')$ dominates $I(H)$.
% % \end{proof}
%
% \begin{lem}\label{dom_minimal}
%   For any generalized near-triangulation $H$, there exists a dom-preserving subgraph $H'$ of $H$ that is dom-minimal.
% \end{lem}
%
% \begin{proof}
%   The proof is by induction on $|V(H)|+|E(H)|$.  If $H$ is already dom-minimal, then setting $H'=H$ satisfies the requirements of the lemma, so assume that $H$ is not dom-minimal.  It is straightforward to verify that the dom-preserving subgraph relationship is transitive, so if $H$ has a dom-preserving subgraph $H^*$ and $H^*$ has a dom-preserving subgraph $H'$ then $H'$ is a dom-preserving subgraph of $H$.  Therefore, it is sufficient to show the existence of a dom-preserving subgraph $H^*$ of $H$ with fewer edges or fewer vertices than $H$. Then the inductive hypothesis provides the desired dom-minimal dom-preserving subgraph $H'$ of $H$.
%
%   If $H$ contains a vertex $v\in B(H)$ with $\deg^+_H(v)=0$ then $H-v$ is a dom-preserving subgraph of $H$ with fewer vertices than $H$.  We now assume that $\deg^+_H(v)\ge 1$ for all $v\in B(H)$.  Since $H$ is not dom-minimal then $H$ contains a biconnected component $C$ that is not dom-minimal.
%   \begin{compactenum}
%     \item If there exists an edge $vw$ on the outer face of $C$ that is not incident to any inner face $vxw$ with $x\in I(C)$ then $B(H-vw)=B(H)$ and $I(H-vw)=I(H)$, and $H-vw$ is a is dom-preserving subgraph of $H$ that has fewer edges than $H$.
%
%     \item If there exists a vertex $v\in B(C)$ with $\deg^+_C(v)=0$ then $v$ is incident to an edge $vw$ that is on the outer face of $C$ and on the outer face of $H$. Since $\deg^+_C(v)=0$, $vw$ is not incident to any inner face $vwx$ with $x\in I(C)$ and we can proceed as in the previous case. \qedhere
%   \end{compactenum}
% \end{proof}
%


\bibliographystyle{plainurlnat}
\bibliography{main}

\appendix

\section{Sage Code}

\lstinputlisting[basicstyle=\ttfamily\scriptsize,language=Python]{cds.py}


\end{document}
